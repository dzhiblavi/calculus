\section{Диффеоморфизмы}

\begin{definition}
    \textit{Областью} называют открытое связное множество.
\end{definition}

\begin{definition}
    Топологические пространства $X$, $Y$ \textit{гомеоморфны}, если
    существует обратимое и в обе стороны непрерывное $f \colon X \to Y$.
    $f$ называют \textit{гомеоморфизмом}.
\end{definition}

\begin{definition}
    \textit{Диффеоморфизмом} гладких многообразий $M, N$ называется
    обратимое и в обе стороны гладкое отображение $f \colon M \to N$.
\end{definition}

\begin{definition}
    Пусть $\Od$ --- область в $\Rm$. Тогда отображение $f \colon
    \Od \to \Rm$ называется \textit{диффеоморфизмом}, если оно обратимо
    и в обе стороны дифференцируемо.
\end{definition}

\begin{lemma}(О почти локальной инъективности)

    Пусть $f \colon \Od \subseteq \Rm \to \Rm$, $\x_0 \in \Od$, $f$
    дифференцируемо в $x_0$, $\det{f'(\x_0)} \neq 0$, тогда \\ $\exists c, \delta
    > 0$ такие, что $\forall \h \colon \norm{\h} < \delta$~
$
    \norm{f(\x + \h) - f(\x)} \geqslant c \norm{\h}
$
\end{lemma}
\begin{proof}
    TBD
\end{proof}

\begin{theorem}(О сохранении области)

    Пусть $f \colon \Od \subseteq \Rm \to \Rm$, $\forall \x \in \Od~ \det{f'(\x)} \neq 0$, тогда $f$ открыто.
\end{theorem}
\begin{proof}
    TBD
\end{proof}

\begin{corollary}

    Пусть $f \colon \Od \subseteq \Rm \to \R^{l \leqslant m}$, $\forall \x \in
    \Od~ \rank{f'(\x)} = l$, тогда $f$ открыто.
\end{corollary}
\begin{proof}
    TBD
\end{proof}

\begin{theorem}(О гладкости обратного отображения)

    Пусть $\Od \subseteq \Rm$ --- область, $f \in C^r(\Od, \Rm)$, $r \in \N \cup
    \{\,+\infty\,\}$, $\forall \x \in \Od~ \det{f'(\x)} \neq 0$, \\ $f$ обратимо,
    тогда $f^{-1} \in C^r$ и $(f^{-1})'(\y_0) = (f'(\x_0))^{-1}$, при $\y_0 =
    f(\x_0)$.
\end{theorem}
\begin{proof}
    TBD
\end{proof}

\begin{lemma}(О приближении оботражения его линеаризацией)

    Пусть $f \in C^1(\Od, \Rm)$, $\x_0 \in \Od$, тогда $\forall \h$
\[
    \norm{f(\x_0 + \h) - f(\x_0) - f'(\x_0)\h} \leqslant M \norm{\h}
\]
    где
\[
    M = \sup_{\z \in [\x_0, \x_0 + \h]}{\norm{f'(\z) - f'(\x_0)}}
\]
\end{lemma}
\begin{proof}
    TBD
\end{proof}

\begin{theorem}(О локальной обратимости)

    Пусть $f \in C^1(\Od, \Rm)$, $\x_0 \in \Od$, $\det{f'(\x_0)} \neq 0$,
    тогда $\exists U(\x_0)$ такая, что $f\big|_U$ --- диффеоморфизм.
\end{theorem}
\begin{proof}
    TBD
\end{proof}

\begin{theorem}(О неявном отображении)

    Пусть $\Od$ открыто, $f \colon \Od \subseteq \R^{m + n} \to \Rn$, $(\x \in
    \Rm, \y \in \Rn) \underset{f}{\mapsto} f(\x, \y)$, $f \in
    C^r$, \\ $(\ela, \elb) \in \Od \colon~ f(\ela, \elb) = 0$,
    $\det{f'_{\y}(\ela, \elb)} \neq 0$, тогда
    \begin{itemize}
        \item $\exists U(\ela)$, $\exists U(\elb)$, $\exists! \f \colon U(\ela)
        \to U(\elb) \in C^r$ такое, что $\forall \x \in U(\ela)~ f(\x, \f(\x))
        = 0$
        \item $\f'(\x) = -(f'_{\y}(\x, \f(\x)))^{-1} \cdot f'_{\x}(\x, \f(\x))$
    \end{itemize}
\end{theorem}
\begin{proof}
    TBD
\end{proof}

\begin{definition}
    $M \subseteq \Rm$ называют $k$\textit{-мерным многообразием} в $\Rm$,
    если оно локально гомеоморфно $\Rk$. Иными словами,
    $\forall \x \in M~ \exists U(\x)~ \exists \f \text{ --- гомеоморфизм}\colon
    \\ U(\x) \underset{\f}{\simeq} \Rk$.
\end{definition}

\begin{definition}
    $k$-мерное многообразие $M \subseteq \Rm$ называют \textit{простым}, если
    оно гомеоморфно $\Rk$. Иными словами, в предыдущем определении можно
    выбрать \\ $U(\x) = M$.
\end{definition}

\begin{definition}
    Пара $\langle U(\x), \f \rangle$ из определения называется \textit{картой},
    или \textit{параметризацией} многообразия в точке $\x$. Набор карт, который
    покрывает все $M$, называется \textit{атласом}.
\end{definition}

\begin{definition}
    Простое $k$-мерное многобразие $M$ называют $C^r$\textit{-гладким}, если \\
    $\f \in C^r$ --- параметризация $M$ и $\forall \x \in
    \Od~ \rank{f'(\x)} \neq 0$.
\end{definition}

\begin{theorem}(О задании гладкого многообразия системой уравнений)

    Пусть $M \subseteq \Rm$, $1 \leqslant k < m$, $r \in \N \cup
    \{\,+\infty\,\}$, тогда $\forall \elp \in M$ эквивалентны утверждения:
    \begin{itemize}
        \item $\exists U(\elp) \subseteq \Rm$ --- открытое такое, что
        $M \cap U$ --- простое $k$-мерное $C^r$-гладкое многообразие.
        \item $\exists \tilde{U}(\elp) \subseteq \Rm$ --- открытое такое,
        что $M \cap \tilde{U}$ можно задать системой $C^r$-гладких уравнений,
        иначе говоря: $\exists f_1, \ldots, f_{m - k} \colon \tilde{U} \to \R
        \in C^r$ такие, что \\ $\x \in M \cap \tilde{U} \Llra \forall i~
        f_i(\x) = 0$, причем $\{\,\grad{f_i}{\elp}\,\}$ линейно независимы.
    \end{itemize}
\end{theorem}
\begin{proof}
    TBD
\end{proof}

\begin{corollary}(О двух параметризациях)

    Пусть $M$ --- $k$-мерное простое $C^r$-гладкое многообразие, $\elp \in M$,
    причем $C^r \ni \f_1 \colon \Od_1 \subseteq \Rk \to U \cap M$,
    $C^r \ni \f_2 \colon \Od_2 \subseteq \Rk \to U \cap M$ --- параметризации
    $U(\elp) \cap M$. Тогда $\f_1$ и $\f_2$ отличаются на диффеоморфизм, а
    именно, $\exists \psi \colon \Od_1 \to \Od_2$ --- диффеоморфизм, причем
    $\f_1 = \f_2 \circ \psi$.
\end{corollary}
\begin{proof}
    TBD
\end{proof}

\begin{definition}

    Пусть $M$ --- $C^r$-гладкое $k$-мерное многообразие в $\Rm$, $\elp \in M$,
    \\ $\f \colon \Od \subseteq \Rk \to \Rm$ --- параметризация окрестности
    $U(\elp)$, причем $\f(\ela) = \elp$. Тогда \textit{касательным
    пространством} к $M$ в точке $\elp$ называется $T_p(M) = \im{\f'(\ela)}$.
\end{definition}

\begin{theorem}(О корректности определения касательного пространства)

    Касательное пространство не зависит от выбора параметризации.
\end{theorem}
\begin{proof}
    TBD
\end{proof}

\begin{theorem}(О касательном пространстве к гладкому пути)

    Пусть $M$ --- гладкое многообразие. Тогда $\v \in T_p(M) \Llra \exists
    \text{ гладкий путь } \gamma \colon [-1, 1] \to \Rm \colon~ \gamma([-1, 1])
    \subseteq M$.
\end{theorem}
\begin{proof}
    TBD
\end{proof}

\begin{theorem}(О касательном пространстве к графику функции)

    Касательное пространство к графику $C^r \ni f \colon \Od \subseteq \Rm \to
    \R$ в точке $\elp = (\x_0, f(\x_0))$ задается уравнением
\[
    y - f(\x_0) = f'_1(\x_1)(\x - \x_1) + \ldots + f'_m(\x_m)(\x - \x_m)
\]
\end{theorem}
\begin{proof}
    TBD
\end{proof}

\begin{theorem}(О касательном пространстве к поверхности уровня)

    Касательное пространство к поверхности уровня функции $f \colon \R^3 \to \R$
    задается уравнением
\[
    f'_x(x_0)(x - x_0) + f'_y(y_0)(y - y_0) + f'_z(z_0)(z - z_0) = 0
\]
\end{theorem}
\begin{proof}
    TBD
\end{proof}
