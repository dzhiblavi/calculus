\chapter{Теория меры}

\section{Системы множеств}

\begin{definition}
    \textit{Полукольцом подмножеств} множества $X$ называют $\PP \subseteq 2^X$,
    удовлетворяющее условиям
    \begin{itemize}
        \item[1.] $\varnothing \in \PP$.
        \item[2.] $A, B \in \PP \Lra A \cap B \in \PP$.
        \item[3.] $\displaystyle \forall A, B \in \PP~ \exists B_1, \ldots,
        B_k \in \PP\colon~ A \setminus B = \bigsqcup_{i = 1}^{k}{B_i}$.
    \end{itemize}
\end{definition}

\begin{definition}
    \textit{Ячейкой} в $\Rm$ называется множество вида
\[
    [\ela, \elb) = \{\, \x \in \Rm \mid \ela_i \leqslant \x_i < \elb_i \,\}
\]
\end{definition}

\begin{theorem}(Свойства полуколец)
    \begin{itemize}
        \item[1.] $A \in \PP \nRightarrow \bar{A} \in \PP$
        \item[2.] $A, A' \in \PP \nRightarrow A~@~A' \in \PP$, $@ \in
        \{\, \cup, \setminus, \triangle \,\}$
        \item[3.] $\displaystyle A_1, \ldots, A_n \in \PP \Lra A \setminus
        \left(\bigcup_{i = 1}^{n}{A_i}\right) = \bigsqcup_{fin}{D_j}$
    \end{itemize}
\end{theorem}

\begin{definition}
    \textit{Алгеброй подмножеств} множества $X$ называется множество $\cA \in
    2^X$ такое, что выполнены аксиомы:
    \begin{itemize}
        \item[1.] $X \in \cA$
        \item[2.] $A, B \in \cA \Lra A \setminus B \in \cA$
    \end{itemize}
\end{definition}

\begin{theorem}(Свойства алгебр)
    \begin{itemize}
        \item[1.] $\varnothing = X \setminus X \in \cA$
        \item[2.] $A \cap B = A \setminus (A \setminus B) \in \cA$
        \item[3.] $\overline{A} = X \setminus A \in \cA$
        \item[4.] $A \cup B \in \cA$
        \item[5.] $\displaystyle A_1, \ldots A_n \in \cA \Lra \bigcup_{i = 1}^{n}{A_i} \in
        \cA$, $\displaystyle \bigcap_{i = 1}^{n}{A_i} \in \cA$
        \item[6.] Алгебра подмножеств является полукольцом подмножеств
    \end{itemize}
\end{theorem}

\begin{definition}
    \textit{$\sigma$-Алгеброй подмножеств} множества $X$ называется алгебра
    подмножеств $\cA$, удовлетворяющая дополнительной аксиоме: \\
    $\displaystyle \{\,A_n\,\} \in \cA \Lra \bigcup_{n = 1}^{+\infty}{A_n} \in \cA$.
\end{definition}

\section{Объём}

\begin{definition}
    Пусть $\PP$ --- полукольцо, $\mu \colon \PP \to \Rbar$ называется
    \textit{конечно-аддитивной}, если
    \begin{itemize}
        \item[1.] $\mu$ принимает не более одного значения из $\{\,+\infty, -\infty\,\}$
        \item[2.] $\mu(\varnothing) = 0$
        \item[3.] $A_1, \ldots, A_n \in \PP$, $A_i \cap A_{j \neq i} = \varnothing$,
        тогда если оказалось, что $\displaystyle A = \bigsqcup_{i = 1}^{n}{A_i} \in \PP$,
        то \\ $\displaystyle \mu(A) = \sum_{i = 1}^{n}{\mu(A_i)}$
    \end{itemize}
\end{definition}

\begin{definition}
    Пусть $\mu \colon \PP \to \Rbar$ называется \textit{объёмом}, если
    \begin{itemize}
        \item[1.] $\mu$ конечно-аддитивна
        \item[2.] $\mu \geqslant 0$
    \end{itemize}
\end{definition}

\begin{definition}
    Объём называется \textit{конечным}, если $\mu(X) < +\infty$.
\end{definition}

\begin{definition}
    \textit{Классическим объёмом} в $\Rm$ называется объём, заданный на
    полукольце ячеек в $\Rm$, вычисляющийся по формуле
    $\displaystyle \mu([\ela, \elb)) = \prod_{k = 1}^{m}{(\elb_k - \ela_k)}$.
\end{definition}

\begin{lemma}(Монотонность объёма)
    Для объёма $\mu,$ $A, B \in \PP$, $A \subseteq B$ выполено $\mu(A) \leqslant \mu(B)$.
\end{lemma}

\begin{theorem}(Свойства объёма)
    \begin{itemize}
        \item[1.] $\displaystyle \forall A, \text{ дизъюнктных } A_1 \ldots, A_n \in \PP
        \colon~ \bigsqcup_{i = 1}^{n}{A_i} \subseteq A \Lra
        \sum_{i = 1}^{n}{\mu(A_i)} \leqslant \mu(A)$ \\ (усиленная монотонность)
        \item[2.] $\displaystyle \forall A, A_1 \ldots, A_n \in \PP
        \colon~ A \subseteq \bigcup_{i = 1}^{n}{A_i} \Lra
        \mu(A) \leqslant \sum_{i = 1}^{n}{\mu(A_i)}$ \\ (конечная полуаддитивность)
    \end{itemize}
\end{theorem}

\section{Мера}

\begin{definition}
    \textit{Мерой} называется объём $\mu \colon \PP \to \Rbar$, обладающий свойством
    счётной аддитивности.
\end{definition}

\begin{theorem}
    Пусть $\mu \colon \PP \to \Rbar$ --- объём. Тогда эквивалентны утверждения:
    \begin{itemize}
        \item[1.] $\mu$ счетно-аддитивен
        \item[2.] $\mu$ счетно-полуаддитивен
    \end{itemize}
\end{theorem}

\begin{theorem}
    Пусть $\cA$ --- алгебра, $\mu \colon \cA \to \Rbar$ --- объём. Тогда
    эквивалентны утверждения:
    \begin{itemize}
        \item[1.] $\mu$ счетно-аддитивно
        \item[2.] $\mu$ \textit{непрерывно снизу}, то есть
        $\displaystyle A,~A_1, A_2 \ldots~ \in \cA\colon A_1 \subset A_2 \subset \ldots$;
        $\displaystyle A = \bigcup_{i = 1}^{+\infty}{A_i} \Lra \\ \mu(A) = \lim_{n \to
        +\infty}{\mu(A_i)}$
    \end{itemize}
\end{theorem}

\begin{theorem}
    Пусть $\cA$ --- алгебра, $\mu \colon \cA \to \Rbar$ --- \textbf{конечный} объём.
    Тогда эквивалентны утверждения:
    \begin{itemize}
        \item[1.] $\mu$ счетно-аддитивен
        \item[2.] $\mu$ непрерывно снизу
        \item[3.] $\mu$ \textit{непрерывно сверху}, то есть
        $\displaystyle A,~A_1, A_2 \ldots~ \in \cA\colon A_1 \supset A_2 \supset \ldots$;
        $\displaystyle A = \bigcap_{i = 1}^{+\infty}{A_i} \Lra \\ \mu(A) = \lim_{n \to
        +\infty}{\mu(A_i)}$
    \end{itemize}
\end{theorem}

\section{О стандартном продолжении меры}

\begin{definition}
    \textit{Пространством с мерой} называется тройка $\langle X, \cA, \mu \rangle$,
    где $\cA$ --- $\sigma$-алгебра, $\mu \colon \cA \to \Rbar$ --- мера.
\end{definition}

\begin{definition}
    $\langle X, \cA, \mu \rangle$ называется \textit{полным} (соответственно мера
    называется \textit{полной}), если $\forall E \in \cA \colon~ \mu(E) = 0 \Lra
    \forall A \subseteq E~ A \in \cA$ и $\mu(A) = 0$.
\end{definition}

\begin{definition}
    $\langle X, \PP, \mu \rangle$ называется \textit{$\sigma$-конечным} (соответственно
    мера называется \textit{$\sigma$-конечной}), если $\displaystyle
    X = \bigcup_{i = 1}^{+\infty}{B_k}$, где $\mu(B_k) < +\infty$.
\end{definition}

\begin{theorem}(О стандартном продолжении меры)

    $\langle X, \PP, \mu_0 \rangle$, $\mu_0$ --- $\sigma$-конечный объём. Тогда
    $\exists~ \sigma$-алгебра $\cA$ и мера $\mu \colon \cA \to \Rbar\colon$
    \begin{itemize}
        \item[1.] $\PP \subseteq \cA$, $\mu\big|_{\PP} = \mu_0$
        \item[2.] $\mu$ полная
        \item[3.] Если $\cA' \supseteq \PP$, $\mu'\big|_{\PP} = \mu_0$,
        $\mu'$ --- полная, тогда $\cA \subseteq \cA'$ и $\mu'\big|_{\cA} = \mu$
        \item[4.] Если $\PP'$ --- полукольцо, $\mu'$ --- мера на $\PP'$,
        $\PP \subseteq \PP' \subseteq \cA$, тогда $\mu' = \mu\big|_{\PP'}$
        \item[5.] $\displaystyle \forall A \in \cA~ \mu(A) =
        \inf{\left(\sum_{k = 1}^{+\infty}{\mu_0(P_k)} ~\bigg|~ A \subseteq \bigcup_{k = 1}
        ^{+\infty}{P_k}, P_k \in \PP\right)}$
    \end{itemize}
\end{theorem}

\section{Мера Лебега}

\begin{theorem}
    Классический объём в $\Rm$ является $\sigma$-конечной мерой.
\end{theorem}

\begin{definition}
    \textit{Мерой Лебега} называется стандартное продолжение классического объёма.
\end{definition}

\begin{definition}
    Алгебра, на которой определена мера Лебега, обозначается $\EuFrak{M}$.
\end{definition}

\begin{definition}
    \textit{Измеримыми по Лебегу} называются множества $A \in \EuFrak{M}$.
\end{definition}

\begin{theorem}(Свойства меры Лебега)
    \begin{itemize}
        \item[1.] Объединения и пересечения измеримых множеств измеримы.
        \item[2.] Все открытые и замкнутые множества измеримы.
    \end{itemize}
\end{theorem}

\begin{lemma}(О структуре открытых множеств)
    \begin{itemize}
        \item[1.] $\Od \subseteq \Rm$ открыто $\Lra~ \exists Q_i$ --- ячейки в $\Rm$
        такие, что $\displaystyle \Od = \bigsqcup_{i}{Q_i}$, причем можно
        дополнительно считать, что выполнено что-либо из нижеперечисленного:
        \begin{itemize}
            \item[(a)] Ячейки имеют рациональные (двоично-рациональные) координаты
            \item[(b)] $\Cl(Q_i) \subseteq \Od$
            \item[(c)] $Q_i$ --- кубы
        \end{itemize}
        \item[2.] Пусть $E$ измеримо в $\Rm$, $\lambda(E) = 0$, тогда
        $\forall \e > 0~\exists Q_i$ --- ячейки в $\Rm$ такие, что
        $\displaystyle E \subseteq \bigcup_{i}{Q_i}$ и $\displaystyle
        \sum_{i}{\mu(Q_i)} < \e$.
    \end{itemize}
\end{lemma}

\begin{theorem}(Свойства меры Лебега)
    \begin{itemize}
        \item[3.] [Канторово множество TBD]
        \item[4.] [Пример неизмеримого множества TBD]
        \item[5.] \begin{itemize}
                    \item $A$ ограничено, тогда $\lambda(A) < +\infty$
                    \item $A$ открыто, тогда $\lambda(A) > 0$
                    \item $\lambda(A) = 0 \Lra$ У $A$ нет внутренних точек
                  \end{itemize}
        \item[6.] $A$ измеримо, тогда $\forall \e > 0$
                \begin{itemize}
                    \item $\exists G_{\e}$ открытое такое, что
                    $A \subset G_{\e}$, $\lambda(G_{\e} \setminus A) < \e$
                    \item $\exists F_{\e}$ замкнутое такое, что
                    $F_{\e} \subset A$, $\lambda(A \setminus F_{\e}) < \e$
                \end{itemize}
    \end{itemize}
\end{theorem}

\begin{definition}
    Пусть $\cA \subseteq 2^X$, тогда \textit{борелевской оболочкой} множества $\cA$
    называют минимальную по включению $\sigma$-алгебру, содержащую $\cA$.
\end{definition}

\begin{definition}
    \textit{Борелевской $\sigma$-алгеброй} называется борелевская оболочка
    всех открытых множеств.
\end{definition}

\begin{corollary}
    $A$ измеримо, тогда $\exists$ борелевские $B, C\colon$ $B \subset A \subset C$
    такие, что $\lambda(C \setminus B) = 0$.
\end{corollary}

\begin{corollary}
    $A$ измеримо, тогда $A = B \cup \EuFrak{N}$, $B$ --- борелевское,
    $\lambda(\EuFrak{N}) = 0$.
\end{corollary}

\begin{corollary}(Регулярность меры Лебега)

    Пусть $A$ измеримо, тогда
\[
    \lambda(A) = \inf_{\substack{G \supset A \\ G \text{ открыто}}}{\lambda(G)}
    = \sup_{\substack{F \subset A \\ F \text{ замкнуто}}}{\lambda(F)}
    = \sup_{\substack{K \subset A \\ K \text{ компакт}}}{\lambda(K)}
\]
\end{corollary}

\begin{lemma}

    Пусть $\langle X', \cA', \mu' \rangle$ --- пространство с мерой.
    $\langle X, \cA, \_ \rangle$ --- заготовка для пространства с мерой.
    $T \colon X \to X'$ --- биекция, $\forall A \in \cA~ T(A) \in \cA'$,
    $T(\varnothing) = \varnothing$. Положим $\mu(A) = \mu'(T(A))$. Тогда
    $\mu$ --- мера на $\cA$.
\end{lemma}

\begin{lemma}
    $T \colon \Rm \to \Rn \in C(\Rm)$, $\forall E \in \EuFrak{M}~ \lambda(E) = 0
    \Lra \lambda(T(E)) = 0$, \\ тогда $\forall A \in \EuFrak{M}~ T(A) \in \EuFrak{M}$.
\end{lemma}
