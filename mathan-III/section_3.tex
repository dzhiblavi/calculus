\chapter{Функциональные последовательности и ряды}

\section{Поточечная и равномерная сходимости последовательностей функций}

\begin{remark}
    \textit{Здесь и далее запись вида $f \to \bot$ будет означать, что $f$
    сходится. Знак $\bot$ используется, если не важно (или не известно), к чему
    сходится $f$.}
\end{remark}

\begin{definition}
    $f_n \colon E \to \R$ \textit{сходится поточечно} к $f \colon E \to \R$ на
    $E$, если
\[
    \forall x_0 \in E~~ f_n(x_0) \to f(x_0)
\]
    иными словами, раскрывая определение сходимости последовательности:
\[
    \forall x_0 \in E~ \left[ \forall \e > 0~ \exists N \in \mathbb{N}\colon~
    \forall n > N~~ |f_n(x_0) - f(x_0)| < \e \right]
\]
    Обозначение: $f_n \to f$.
\end{definition}

\begin{examples}
    TBD
\end{examples}

\begin{definition}
    $f_n \colon E \to \R$ \textit{сходится равномерно} к $f \colon E \to \R$ на
    $E$, если
\[
    \sup_{\x \in E}|f_n(x) - f(x)| \xrightarrow[n \to +\infty]{} 0
\]
    или, раскрывая описание супремума
\[
    \forall \e > 0~ \exists N \in \mathbb{N}\colon~ [\forall x \in E~~
    |f_n(x) - f(x)| < \e]
\]
    Обозначение: $f_n \rcon f$.
\end{definition}

\begin{remark}
    Из равномерной сходимости очевидным образом следует поточечная:
\[
    f_n \rcon f \Lra f_n \to f
\]
\end{remark}

\textit{Про сходимость мы значем очень многое для случая метрических пространств.
А нельзя ли переформулировать новые определения так, чтобы они оказались обычной
сходимостью, просто в хитром метрическом пространстве?}
\begin{present}(Метрическое пространство ограниченных функций)

    Положим
\[
    \mathcal{F} \defeq \{\, X \to \mathbb{R} \mid f \text{ограничено} \,\}
\]
    На этом множестве тривиально задается структура линейного пространства:
\begin{gather*}
    (f + g)(x) = f(x) + g(x) \\
    (\lambda f)(x) = \lambda f(x)
\end{gather*}

    Оказывается, можно ввести \textbf{метрику} на $\mathcal{F}$, сходимость
    по которой есть равномерная сходимость. Для $f\!, \! g \in \mathcal{F}$
    положим
\[
    \r(f, g) \defeq \sup_{x \in X}|f(x) - g(x)|
\]
    Проверим, что это --- метрика на $\mathcal{F}$

    \begin{itemize}
        \item[i)] Неотрицательность очевидна. Равенство нулю может выполнится
        только для равных функций.
        \item[ii)] Симметричность очевидна.
        \item[iii)] Проверим неравенство треугольника. Применим техническое
        описание супремума для $\r(f_1, f_2)$:
\[
            \forall \e > 0~ \exists x \colon~ \sup_{y \in X}|f_1(y) - f_2(y)| - \e
            \leqslant |f_1(x) - f_2(x)|
\]
        Далее
        \begin{align*}
            \forall \e > 0~ \exists x \colon~  \sup_{y \in X}|f_1(y) - f_2(y)| -
            \e &\leqslant |f_1(x) - f_2(x)| \leqslant |f_1(x) - f_3(x)| + |f_3(x)
            - f_2(x)| \\ &\leqslant \sup_{y \in X}|f_1(y) - f_3(y)| + \sup_{y \in
            X}|f_2(y) - f_3(y)| \\ &= \r(f_1, f_3) + \r(f_2, f_3)
        \end{align*}
        Получаем
\[
    \forall \e > 0~~ \r(f_1, f_2) - \e \leqslant \r(f_1, f_3)
    + \r(f_2, f_3)
\]
        Откуда непосредственно следует
\[
    \r(f_1, f_2) \leqslant \r(f_1, f_3) + \r(f_2, f_3)
\]
    \end{itemize}

    Осталось только понять, что теперь означает сходимость по этой метрике.
    Пусть $(f_n)$ --- последовательность в $\mathcal{F}$, сходящаяся к $f$ по
    метрике $\r$:
\[
    \forall \e > 0~ \exists N \in \mathbb{N}\colon~ \forall n > N~~ \r(f_n, f) <
    \e
\]
    Раскроем значение $\r$:
\[
    \forall \e > 0~ \exists N \in \mathbb{N}\colon~ \forall n > N~~ [\forall x
    \in X~~ |f_n(x) - f(x)| < \e]
\]
    А это --- обычное определение равномерной сходимости!
\end{present}

\textit{Подобную конструкцию, по всей видимости, не получится ввести для
поточечной сходимости. Зато, можно построить хаусдорфово топологическое
пространство, в котором сходимость будет означать поточечную сходимость.}
\begin{present}(Топологическое пространство ограниченных функций)

    Введем на $\mathcal{F}$ топологию, порожденную следующими множествами:
\[
    U_{\e}(f)_{x_1, \ldots, x_n} \defeq
    \{\, g \colon X \to \R \mid \forall i~~ |g(x_i) - f(x_i)| < \e \,\}
\]
    Поймем теперь, что означает сходимость в этом топологическом пространстве:
\[
        f_n \to f \Llra \forall U_{\e}(f)_{x_1, \ldots, x_n} \exists N \in
        \mathbb{N}\colon~ \forall n > N~~ f_n \in U_{\e}(f)_{x_1, \ldots, x_n}
\]
    Что означает
\[
    \forall \e > 0~ \exists N \in \mathbb{N}\colon~ \forall n > N~~ \forall i~~
    |f_n(x_i) - f(x_i)| < \e
\]
    Что как раз и есть поточечная сходимость! Просто запись вида
\[
    [\forall x_0 \in X~ \forall \e > 0]~ \exists N \in \mathbb{N}\colon~ \forall
    n > N~~ |f_n(x_0) - f(x_0)| < \e
\]
    В этом пространстве обретает вид
\[
    [\forall U_{\e}(f)_{x_0}]~ \exists N \in \mathbb{N}\colon~ \forall
    n > N~~ |f_n(x_0) - f(x_0)| < \e
\]
\end{present}

\begin{theorem}(Критерий Больцано-Коши равномерной сходимости)

\[
    f_n \rcon f \Llra \forall \e > 0~ \exists N \in \mathbb{N}\colon~
    \forall n, m > N~ [\forall x~ |f_n(x) - f_m(x)| < \e]
\]
\end{theorem}
\begin{proof}
    \enewline
    \begin{itemize}
        \item[$\Lra$] Обычное свойство всех последовательностей, сходящихся по
        метрике (если все $f_n$ и $f$ лежат в $\mathcal{F}$). Общее
        доказательство такое:
\[
            |f_n(x) - f_m(x)| \leqslant |f_n(x) - f(x)| + |f(x) - f_m(x)| <
            \frac{\e}{2} + \frac{\e}{2} = \e
\]

        \item[$\Lla$] (Полнота $\langle \mathcal{F}, \r \rangle$) Зафиксируем
        $x$. Тогда $f_n(x)$ --- обычная фундаментальная вещественная
        последовательность. Тогда, так как $\mathbb{R}$ --- полное, получаем
\[
        \forall x~ \exists \lim_{n \to +\infty}f_n(x) =: f(x)
\]
        Покажем, что $f_n \rcon f$. Посмотрим на фундаментальность $f_n$:
\[
        \forall \e > 0~ \exists N \in \mathbb{N}\colon~ \forall n, m > N~~
        [\forall x~ |f_n(x) - f_m(x)| < \e]
\]
        и перейдем к пределу $m \to +\infty$:
\[
        \forall \e > 0~ \exists N \in \mathbb{N}\colon~ \forall n > N~~
        [\forall x~ |f_n(x) - f_m(x)| < \e]
\]
        Что и есть равномерная сходимость.
    \end{itemize}
\end{proof}

\begin{examples}
    TBD
\end{examples}
