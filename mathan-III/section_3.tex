\section{Формула Тейлора}

\begin{definition}
    Пусть $f \colon \O \subseteq \Rm \to \Rn$, $\O$ --- область, $i_1, \ldots,
    i_k \in \{\, 1, 2, \ldots, m \,\}$. Определим частные производные высшего
    порядка по индукции:
\[
    \pderi{f}{i_1, \ldots, i_k} \defeq \pderi{(\pderi{f}{i_1, \ldots, i_{k-
    1}})}{i_k}
\]
\end{definition}

\begin{theorem}(О независимости ч.п. от порядка дифференцирования)

    Пусть $f \colon \O \subseteq \R^2 \to \R$, $\O$ --- область, $(x_0, y_0) \in
    \O$, $\exists B((x_0, y_0), r) \subseteq \O$, причем в $B((x_0, y_0), r)$
    существуют $\pderi{f}{12}$ и $\pderi{f}{21}$, непрерывные в точке $(x_0,
    y_0)$. Тогда $\dder{f}{x_0, y_0}{12} = \dder{f}{x_0, y_0}{21}$
\end{theorem}
\begin{proof}
\[
        \a(h) = f(x_0 + h, y_0 + k) - f(x_0 + h, y_0) - f(x_0, y_0 + k) + f(x_0,
        y_0)
\]
    Тогда $\a(0) = 0$:
\begin{align*}
        \a(h) = \a(h) - \a(0) &\underset{\text{Лагранж}}{=}
        \a'(\tilde{h})h = [f'_x(x_0 + \tilde{h}, y_0 + k) - f'_x(x_0 + \tilde{h},
        y_0)]h \\ &\underset{\text{Лагранж}}{=} f''_{xy}(x_0 + \tilde{h}, y_0 +
        \tilde{k})hk
\end{align*}
    Аналогично введем $\b(k)$:
\[
        \b(k) = f(x_0 + h, y_0 + k) - f(x_0 + h, y_0) - f(x_0, y_0 + k) + f(x_0,
        y_0)
\]
    Тогда
\begin{align*}
        \b(k) = \b(k) - \b(0) &\underset{\text{Лагранж}}{=}
        \b'(\bar{k})k = [f'_y(x_0 + h, y_0 + \bar{k}) - f'_y(x_0, y_0 + \bar{k})]k
        \\ &\underset{\text{Лагранж}}{=} f''_{yx}(x_0 + \bar{h}, y_0 + \bar{k})hk
\end{align*}
    Заметим, что $\a(h) = \b(k)$. Осталось перейти к пределу при $(h, k) \to (0,
    0)$ и воспользоваться непрерывностью частных производных в точке $(x_0, y_0)$.
\end{proof}

\begin{corollary}

    Пусть $f \colon \O \subseteq \Rm \to \Rn$, $i_1, \ldots, i_k \in \{\, 1, 2,
    \ldots, m \,\}$, $\x \in \O$, $\exists B(\x, r) \subseteq \O$, причем
    в $B(\x, r)$ для любой перестановки индексов $\pi \in S_k$ существуют и
    непрерывны в $\x$ частные производные $\pderi{f}{i_{\pi_1}, \ldots, 
    i_{\pi_k}}$. Тогда все они совпадают в точке $\x$.
\end{corollary}
\begin{proof}
    Доказательство сводится к координатным функциям, поэтому считаем, что $n = 1$.
    Предыдущая теорема дает возможность менять местами пары индексов. Осталось
    заметить, что группа перестановок порождается транспозициями.
\end{proof}

\begin{definition}
    Множество функций $f \colon \O \subseteq \Rm \to \Rn$, у которых все частные
    производные порядка не более $r$ существуют и непрерывны на $\O$, будем
    обозначать $C^r(\O)$
\end{definition}
