\chapter{Интеграл}

\section{Измеримые функции}

\begin{definition}
	\textit{Разбиением} множества $E$ называется дизъюнктный набор множеств
	$e_i$ такой, что $E = \bigsqcup{e_i}$.
\end{definition}

\begin{definition}
	Функция $f\colon X \to \R$ называется \textit{ступенчатой}, 
	если существует конечное разбиение $X = \bigsqcup{e_i}$ (в контексте 
	мер множества $e_i$ должны быть измеримыми, то есть $e_i \in \cA$) такое,
	что на элементах разбиения $f$ постоянно:
\[
	f\big|_{e_i} = c_i
\]
	Разбиение $e_i$ в таком случае называется \textit{допустимым}.
\end{definition}

\begin{definition}
	\textit{Характеристической функцией} множества $E$ называется 
	фукнция
\begin{align*}
	\chi_E \colon X &\to \R \\
	x &\mapsto \begin{cases}
		1,~ x \in E \\
		0,~ x \notin E
	\end{cases}
\end{align*}
\end{definition}

\begin{remark}
	Если множество $E$ измеримо, то его характеристическая функция 
	является ступенчатой. Подойдет разбиение $X = E \bigsqcup \overline{E}$.
\end{remark}

\begin{remark}
	Ступенчатую функцию можно представить в виде:
\[
	f(x) = \sum_{i}{\chi_{e_i}(x) \cdot c_i}
\]
\end{remark}

\begin{lemma}(Свойства ступенчатых функций)
	\begin{itemize}
		\item Пусть $f$, $g$ ступенчатые. Тогда существует 
			общее допустимое разбиение.
		\item $f$, $g$ ступенчатые, тогда $\forall \a \in \R$ функции 
			$f + g$, $f \cdot g$, $\a f$, $\max(f, g)$, $|f|$, $\displaystyle
			\frac{f}{g}$ (при $g \neq 0$) ступенчатые.
	\end{itemize}
\end{lemma}
\begin{proof}
	\enewline
	\begin{itemize}
		\item Пусть $a_i$ и $b_i$ допустимые разбиения $f$ и $g$ соответственно. 
			Тогда в качестве их общего допустимого разбиения, очевидно,
			подойдет разбиение $a_i \cap b_j$.
		\item Очевидно. Если фукнция составлена из двух, можно рассмотреть 
			общее допустимое разбиение.
	\end{itemize}
\end{proof}

\textit{Контекст: $\langle X, \cA, \mu \rangle$ --- пространство с мерой.}

\begin{definition}
	Пусть $f \colon E \subseteq X \to \Rbar$. Тогда \textit
	{лебеговскими множествами} называются множества вида
\[
	E(f < a) \defeq \{\, x \in E \mid f(x) < a \,\}
\]
	Где $a \in \R$.
\end{definition}

\begin{remark}
	\enewline
	\begin{itemize}
		\item $E(f < a) = \overline{E(f \geqslant a)}$.
		\item $\displaystyle E(f \leqslant a) = \bigcap_{b > a}{E(f < b)} = 
			\bigcap_{n \in \bN}{E\left(f < a + \frac{1}{n}\right)}$
	\end{itemize}
\end{remark}

\begin{definition}
	$f \colon X \to \Rbar$, называется \textit{измеримой на множестве} $E \in \cA$,
	если
\[
	\forall a \in \R~~ E(f < a) \in \cA
\]
\end{definition}

\begin{definition}
	$f$ называется \textit{измеримой}, если она измерима на множестве $X$.
\end{definition}

\begin{definition}
	$f$ называется \textit{измеримой по Лебегу}, если она измерима в 
	контексте $\langle \Rm, \mathfrak{M}^m, \l_m \rangle$.
\end{definition}

\begin{remark}
	В определении измеримости на множестве можно брать 
	лебеговские множества любого вида (см. замечания выше:
	все эти множества измеримы или нет одновременно).
\end{remark}

\begin{lemma}(Измеримость непрерывных функций)
	Пусть $f \colon \Rm \to \R$ непрерывна. Тогда $f$ измерима по Лебегу.
\end{lemma}
\begin{proof}
	Имеем
\[
	E(f < a) = f^{-1}((-\infty, a))
\]
	Последнее множество открыто по определению непрерывности, а значит,
	измеримо.
\end{proof}

\begin{theorem}(Свойства измеримых функций)

	Пусть $f$ измерима на $E$. Тогда 
	\begin{itemize}
		\item $\forall a \in \R~ E(f = a)$ измеримо.
		\item $\forall \a > 0~ \a \cdot f$, $-f$ измеримы.
		\item $f$ измерима на $E_1, \ldots$ тогда $f$ измерима на $E' = \bigcup{E_i}$.
		\item $E' \subseteq E$, $E'$ измеримо, тогда $f$ измерима на $E'$.
		\item $f \neq 0$, тогда $\displaystyle \frac{1}{f}$ измерима на $E$.
		\item $f \geqslant 0,~\a > 0$, тогда $f^\a$ измерима на $E$.
	\end{itemize}
\end{theorem}
\begin{proof}
	Для доказательства всех пунктов просто преобразуем лебеговские 
	множества так, чтобы все стало очевидно из определения:
	\begin{itemize}
		\item $E(f = a) = E(f \leqslant a) \cap E(f \geqslant a)$.
		\item $E(\a \cdot f < a) = E\left(f < \frac{a}{\a}\right)$.
		\item $E'(f < a) = \bigcup{E_i(f < a)}$.
		\item $E'(f < a) = E' \cap E(f < a)$.
		\item
\begin{align*}
	E\left(\frac{1}{f} < a\right) = \left(E\left(\frac{1}{f} < a\right) 
	\cap E(f > 0)\right) 									
	\cup \left(E\left(\frac{1}{f} < a\right) \cap E(f < 0)\right) \\
	= \left(E\left(f > \frac{1}{a}\right) \cap E(f > 0)\right) 
	\cap \left(E\left(f < \frac{1}{a}\right) \cap E(f < 0)\right)
\end{align*}
		\item Аналогично.
	\end{itemize}
\end{proof}

\begin{theorem}(Измеримость пределов и супремумов)

	Пусть $f_n$ измеримы на $E$, тогда
	\begin{itemize}
		\item $\displaystyle \sup_{n}{f_n(x)}$, $\displaystyle \inf_{n}{f_n(x)}$
			измеримы на $E$.
		\item $\displaystyle \varlimsup_{n \to +\infty}{f_n(x)}$, 
			$\displaystyle \varliminf_{n \to +\infty}{f_n(x)}$ измеримы на $E$.
		\item Если $\displaystyle \forall x~ \exists f(x) =
			\lim_{n \to +\infty}{f_n(x)}$, то $f$ измерима на $E$.
	\end{itemize}
\end{theorem}
\begin{proof}
	\enewline
	\begin{itemize}
		\item Пусть $\displaystyle g(x) = \sup_{n}{f_n(x)}$. Рассмотрим лебеговы 
			множества $g$:
\[
	E(g > a) = \bigcup_{n}{E(f_n > a)}
\]
	Из этого равенства очевидным образом следует первое утверждение.
	Докажем его подробно:
		\begin{itemize}
			\item[$\subseteq$]
				Пусть $y \in E(g > a)$. Тогда $g(y) > a$:
\[
	a < g(y) = \sup_{n}{f_n(y)} \Lra \exists n\colon f_n(y) > a \Lra
	y \in E(f_n > a)	
\]
				Действительно, если бы $\forall n~ f_n(y) \leqslant a$, то 
				тогда $\displaystyle g(y) = \sup_{n}{f_n(y)} \leqslant a$.
			\item[$\supseteq$]
				Пусть теперь наоборот, $y \in E(f_n > a)$ для некоторого $n$.
				Тогда $f(y) > a$, поэтому:
\[
	g(y) = \sup_{n}{f_n(y)} \geqslant f_n(y) > a \Lra y \in E(g > a)
\]
		\end{itemize}	
		\item Верхний и нижний пределы определялись в терминах огибающих. 
			Воспользуемся этим (а еще вспомним, чему равен предел 
			монотонной последовательности):
\[
	\varlimsup_{n \to +\infty}{f_n(x)} = \lim_{n \to +\infty}{\sup_{k \geqslant n}
	{f_k(x)}} = \inf_{n}{\sup_{k}{f_{n + k}(x)}}
\]
			Инфимум и супремум, как видно из первого пункта, измеримы.
		\item Если предел существует, то он совпадает и с верхним, и с нижним
			пределами, а значит, измерим.
	\end{itemize}
\end{proof}

\begin{corollary}
	Если измеримы $f$, $g$, то измеримы:
	\begin{itemize}
		\item $\max(f, g)$, $\min(f, g)$.
		\item $f_{+}$, $f_{-}$.
		\item $|f|$.
	\end{itemize}
\end{corollary}
\begin{proof}
	\enewline
	\begin{itemize}
		\item $\max(f, g) = \sup(f, g, g, \ldots)$, $\min(f, g) = \inf(f, g, g, \ldots)$.
		\item $f_{+} = \max(0, f)$, $f_{-} = \max(0, -f)$.
		\item $|f| = \max(f, -f)$.
	\end{itemize}
\end{proof}

\begin{theorem}
	Пусть $\langle X, \cA, \mu \rangle$ --- пространство с мерой, 
	$f \colon X \to \Rbar$, $f \geqslant 0$, измерима. Тогда
	$\exists f_n$ --- ступенчатые такие, что
\begin{itemize}
	\item $0 \leqslant f_1 \leqslant \ldots \leqslant f_n \leqslant f_{n + 1}
		\leqslant \ldots$
	\item $\displaystyle f(x) = \lim_{n \to +\infty}{f_n(x)}$.
\end{itemize}
\end{theorem}
\begin{proof}
	Зафиксируем $n$. Нарежем отрезок $[0, n]$ на $n^2$ равных частей. 
	Рассмотрим множества вида
\[
	e^{(n)}_k = X\left(\frac{k}{n} \leqslant f < \frac{k + 1}{n}\right)
\]
	Для $0 \leqslant k < n^2$. Для $n^2$:
\[
	e^{(n)}_{n^2} = X(f \geqslant n)
\]
	Заведем пока примерную последовательность ступенчатых функций:
\[
	g_n(x) = \sum_{k = 1}^{n^2}{\chi_{e^{(n)}_k}(x) \cdot \frac{k}{n}}
\]
	По построению имеем
\[
	0 \leqslant g \leqslant f
\]
	Поймем, почему $g_n \to f$.
	\begin{itemize}
		\item Пусть $f = +\infty$. Тогда
\[
	\forall x~ x \in e^{(n)}_{n^2} \Lra g_n(x) = n \to +\infty
\]
		\item Пусть $f < +\infty$. Тогда
\[
	\forall n > f(x)~ \exists k_n\colon~ x \in e^{(n)}_{k_n} \Lra
	|f(x) - g_n(x)| \leqslant \frac{1}{n} \xrightarrow[n \to +\infty]{} 0
\]
	\end{itemize}
	Не хватает упорядоченности. Исправим это:
\[
	f \geqslant f_n = \max(g_1, g_2, \ldots, g_n) \geqslant g_n
\]
	По теореме о сжатой последовательности сходится туда, куда надо,
	причем
\[
	0 \leqslant f_1 \leqslant \ldots \leqslant f_n \leqslant f_{n + 1} \leqslant 
	\ldots
\]
\end{proof}

\begin{corollary}
	Пусть $f$ измерима. Тогда $\exists g_n$ такие, что
\begin{itemize}
	\item $\forall n~ |g_n| \leqslant f$.
	\item $\displaystyle \forall x~ \lim_{n \to +\infty}{g_n(x)} = f(x)$.
\end{itemize}
\end{corollary}
\begin{proof}
	Разложим $f = f_{+} - f_{-}$. Срезки в свою очередь представим в 
	виде ступенчатых функций по теореме:
\begin{align*}
	0 \leqslant \ldots \leqslant g^{+}_n \leqslant \ldots \\
	0 \leqslant \ldots \leqslant g^{-}_n \leqslant \ldots 
\end{align*}
	Тогда положим
\[
	g_n = g^{+}_n - g^{-}_n
\]
	Причем в любой точке одна из $g^{+}$, $g^{-}$ равна нулю (по определению срезок).
	А значит, $g_n$ подходит.
\end{proof}

\begin{corollary}
	Пусть $f, g\colon X \to \Rbar$ измеримы. Тогда измерима $f \cdot g$.
\end{corollary}
\begin{proof}
	Сопоставим $f$ последовательность ступенчатых $f_n$, $g$ --- $g_n$. Тогда
	везде, вроме точек, в которых возникает неопределенность $0 \cdot \infty$, 
	имеем:
\[
	\forall x~ f_n(x) \cdot g_n(x) \xrightarrow[n \to +\infty]{} (f \cdot g)(x)
\]
	Если дополнительно считать, что $0 \cdot \infty = 0$, то сходимость будет и в 
	таких точках (по всей видимости, это следует из построения последовательностей
	$f_n$ и $g_n$: там в случае $0 \cdot \infty$ получатся ступени $0$ у той функции,
	которая принимает значение $0$ и $n$ у той функции, которая принимает значение 
	$+\infty$. Поэтому получается $\forall n~ f_n(x) \cdot g_n(x) = 0 \to 0 
	= (f \cdot g)(x)$).
	Из чего получаем, что $f \cdot g$ --- предельная фукнция последовательности
	измеримых функций, то есть измерима.
\end{proof}

\begin{corollary}
	Пусть $f, g\colon X \to \Rbar$ измеримы, причем в выражении 
	$f + g$ не возникает неопределенности $\infty - \infty$.
	Тогда $f + g$ измерима.
\end{corollary}
\begin{proof}
	Абсолютно аналонично предыдущему следствию.
\end{proof}

\begin{remark}
	Если положить $\infty - \infty = 0$, измеримость сохранится и для таких 
	случаев.
\end{remark}

\begin{definition}
	Предикат $w(x),~ x \in E$ \textit{выполняется почти везде на множестве} $E \in \cA$, 
	если $\exists e \in \cA\colon~$
\begin{itemize}
	\item $\mu(e) = 0$.
	\item $\forall x \in E \setminus e~ w(x)$.
\end{itemize}
\end{definition}

\begin{theorem}
	Пусть $\mu$ --- полная мера. Тогда измеримая на $E \in \cA$ почти везде функция $f$
	измерима.
\end{theorem}
\begin{proof}
	Пусть $e \in \cA$ --- такое множество меры $0$, что $f$ измерима на $E \setminus e$.
	Тогда множества Лебега для $E$ выглядят так:
\[
	E(f < a) = (E \setminus e)(f < a) \cup e(f < a)
\]
	Измеримы, потому что $e(f < a) \subseteq e$ --- измеримо по полноте $\mu$.
\end{proof}

\begin{corollary}(Теорема об измеримости функции, непрерывной на множестве полной меры)

	\textit{Множество полной меры --- множество, дополнение которого в данном 
	контексте имеет нулевую меру} \\

	Пусть $f \colon E \to \R$, 
	$e \subseteq E$, $\l(e) = 0$, $E' = E \setminus e$,
	$f\big|_{E'} \in C(E')$. Тогда $f$ измерима по Лебегу.
\end{corollary}
\begin{proof}
	На $E'$ $f$ непрерывна, а поэтому измерима. Значит, $f$ почти всюду измерима на $E$,
	что по теореме дает измеримость.
\end{proof}

\begin{corollary}
	Пусть $\mu$ --- полная мера, $f_n \to f$ почти везде, $f_n$ измеримы. 
	Тогда $f$ измерима.
\end{corollary}
\begin{proof}
	Из сходимости почти везде получаем, что $f$ почти везде измерима (как предельная 
	функция). Значит, $f$ измерима везде.
\end{proof}

\begin{corollary}
	Пусть $\mu$ --- полная мера, $g$ измерима, $f$ не совпадает с 
	$g$ только на множестве нулевой меры. Тогда $f$ измерима.
\end{corollary}
\begin{proof}
	Аналогично предыдущим двум следствиям.
\end{proof}

\begin{remark}
	Что делать, если $\mu$ --- не полная? Проблема возникнет в тот момент, 
	когда мы будем рассматривать множество $e(f < a)$, которое в случае
	полной меры обязательно измеримо, если $\mu(e) = 0$. Но мы знаем,
	что $e$ измеримо. Если мы сделаем так, чтобы $e(f < a) = e$, либо $e(f < a) = 
	\varnothing$, все будет работать. Все, что надо для этого сделать --- 
	переопределить $f$ на $e$ так, чтобы $f$ на $e$ оказалась константой.
	Понятно, что в таком случае либо $e(f < a) = e$, либо $e(f < a) = \varnothing$.
\end{remark}

\begin{definition}
	Функции, совпадающие почти всюду, называются \textit{эквивалентными}.
\end{definition}

\begin{theorem}
	Пусть $\forall n~ w_n(x)$ верно почти везде. Тогда 
	$\bigwedge{w_n(x)}$ верно почти везде.
\end{theorem}
\begin{proof}
	Возьмем множества $e_n$ для всех $w_n$ из определения верности почти везде.
	Тогда $\bigwedge{w_n(x)}$ верно на $\displaystyle X \setminus 
	\left(\bigcup{e_n}\right)$. По счетной полуаддитивности меры имеем
\[
	\l\left(\bigcup{e_n}\right) \leqslant \sum{\l(e_n)} = 0
\]
	Поэтому $\bigwedge{w_n(x)}$ верно почти везде.
\end{proof}

\section{Сходимость по мере}

\begin{definition}
	Пусть $f_n$, $f$ измеримы и почти везде конечны. Тогда 
	$f_n$ \textit{сходится к $f$ по мере}, если
\[
	\forall \e > 0~ \mu(X(|f - f_n| \geqslant \e)) \xrightarrow[n \to +\infty]{} 0
\]
	Обозначают $f_n \underset{\mu}{\Lra} f$.
\end{definition}

\begin{remark}
	Определение сходимости по мере устойчиво к изменениям фукнций на множетсве меры 0.
\end{remark}

\begin{theorem}(Лебега о сходимости почти всюду и по мере)

	Пусть $\mu(X) < +\infty$, $f_n, f \colon X \to \Rbar$ измеримы и почти всюду конечны,
	причем  $f_n \to f$ почти везде. Тогда $f_n \underset{\mu}{\Lra} f$.
\end{theorem}
\begin{proof}
	\enewline
	\begin{itemize}
		\item Для начала поправим $f_n$ и $f$ на том множестве, 
			где сходимость не наблюдается: заменим $f_n$ и $f$ на 0.
			Теперь имеем сходимость везде, при этом на сходимость по мере 
			это никак не повлияло.
		\item Разберем частный случай. Пусть $f_n \to 0 = f$, $\forall x~ f_n$
			монотонна по $n$. В таком случае, $|f_n|$ монотонно убывает.
			Тогда рассмотрим множества вида:
\[
	X(|f_n| \geqslant \e) \supseteq X(|f_{n + 1}| \geqslant \e) \supseteq \ldots
\]
			И множество из опредeления $f_n \underset{\mu}{\Lra} f$:
\[
	0 = \mu(\varnothing) = \l\left(\bigcap{X(|f_n| \geqslant \e)}\right) 
	= \lim_{n \to +\infty}{\l(X(|f_n| \geqslant \e))}
\]
			Последнее равенство --- непрерывность сверху.
		\item Докажем общий случай, сведя его к частному. Рассмотрим последовательность
			функций
\[
	\f_n(x) = \sup_{k \geqslant n}{|f_k(x) - f(x)|} \xrightarrow[n \to +\infty]{} 0
\]
			Очевидно, что $\f_n(x)$ убывает по $n$. Тогда к $\f$ применим частный случай:
\[
	\l(X(|\f_n| \geqslant \e)) \to 0
\]
			Но тогда, поскольку $X(|f_n - f| \geqslant \e) 
			\subseteq X(|\f_n| \geqslant \e)$:
\[
	\l(X(|f_n - f| \geqslant \e)) \leqslant \l(X(|\f_n| \geqslant \e)) \to 0
\]
			Что и требовалось.
	\end{itemize}		
\end{proof}

\begin{theorem}(Рисса о сходимости по мере и почти всюду)
			
	Пусть $f_n, f \colon X \to \Rbar$ --- измеримые и почти везде конечные функции, причем
	$f_n \underset{\mu}{\Lra} f$. Тогда
\[
	\exists n_k\colon~ f_{n_k} \xrightarrow[k \to +\infty]{} f
\]
	Почти всюду.
\end{theorem}
\begin{proof}
	Воспользуемся определением сходимости по мере:
\[
	\forall k~\exists n'_k\colon~ \forall n \geqslant n'_k~
	\mu\left(X\left(|f_n - f| \geqslant \frac{1}{k}\right)\right) < \frac{1}{2^k}
\]
	Заменим $n'_k$ на $n_k$ так, чтобы $n_k$ возрастало (от этого ничего не изменится).
	Положим 
\[
	E_k = \bigcup_{j = k}^{+\infty}{X\left(|f_{n_j} - f| \geqslant \frac{1}{j}\right)}	
\]
	Получаем:
	\begin{itemize}
		\item $E_1 \supseteq E_2 \supseteq \ldots$
		\item
\[
	\mu(E_k) \leqslant \sum_{j = k}^{+\infty}{\frac{1}{2^j}} = \frac{2}{2^k}
\]
	Пусть тогда $E = \bigcap{E_k}$, тогда по непрерывности сверху имеем:
\[
	\mu(E) = \lim_{k \to +\infty}{\mu(E_k)} = 0
\]
	Проверим, что везде, кроме $E$ $f_n \to f$, что и закончит доказательство.
	Проверим: $\forall x \notin E~ f_n(x) \xrightarrow[n\to +\infty]{} f(x)$. Раз 
	$x \notin E$, то $\exists N\colon~ \forall n > N~ x\notin E_n$, что по определению
	$E_k$ означает:
	\begin{align*}
		|f_{n_N}(x) - f(x)| &< \frac{1}{N} \\
		|f_{n_{N + 1}}(x) - f(x)| &< \frac{1}{N + 1} \\
		&\ldots
	\end{align*}
	То есть $f_n(x) \to f(x)$.
	\end{itemize}
\end{proof}

\begin{corollary}(Аналог предельного перехода в неравенствах для сходимости по мере)
		
	Пусть $f_n \underset{\mu}{\Lra} f$, $|f_n(x)| \leqslant g(x)$ почти везде, 
	тогда $|f(x)| \leqslant g(x)$ почти везде.
\end{corollary}
\begin{proof}
	Выделим подпоследовательность $n_k$, вдоль которой $f_{n_k} \to f$ почти везде.
	Выполним в неравенстве $|f_{n_k}(x)| < g(x)$ предельный переход. Получим, 
	что $|f(x)| \leqslant g(x)$ почти везде, что и требовалось.
\end{proof}

\begin{theorem}(Егорова о сходимости почти везде и почти равномерной сходимости)

	Пусть $\mu(X) < +\infty$, $f_n, f \colon X \to \Rbar$ измеримы и почти везде конечны,
	$f_n \to f$ почти везде, тогда
\[
	\forall \e > 0~ \exists e\colon~ \mu(e) < \e,~ f_n \rcon f \text{ на } X \setminus e
\]
\end{theorem}
