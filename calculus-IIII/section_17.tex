\section{Формула обращения в $\R$}

\begin{definition}
    \textit{Интегралом Фурье} называется
    \[
        f(x) = \intl_{-\infty}^\infty{\hat{f}(y) e^{2i\pi yx} \dd y}
    .\]
\end{definition}

\begin{definition}
    \textit{Частичным интегралом Фурье} называется
    \[
        I_A(f, x) = \intl_{-A}^A{\hat{f}(y) e^{2i\pi yx} \dd y}
    .\]
\end{definition}

\begin{lemma}(О ядре Дирихле)
   
    Пусть $f \in L^1(\R)$, $x \in \R$. Тогда $\forall A > 0$
    \[
        I_A(f, x) = \intl_{-\infty}^\infty{f(x - t) \frac{\sin{2\pi At}}{\pi t} \dd t} 
    .\]
\end{lemma}
\begin{proof}
    Обозначим $\chi_A = \chi_{[-A, A]}$. Тогда
    \begin{align*}
        I_A(f, x) 
        &= \intl_{-\infty}^\infty{\hat{f}(y) \parens*{\chi_A(y) e^{2i\pi xy}} \dd y}
        = \intl_{-\infty}^\infty{f(y) \parens*{\chi_A(y) e^{2i\pi xy}}^\wedge(y) \dd y}
        = \intl_{-\infty}^\infty{f(y) \hat{\chi}_A(y - x) \dd y} \\
        &= \intl_{-\infty}^\infty{f(x - t) \frac{\sin{2\pi At}}{\pi t}}
    .\end{align*}
\end{proof}

\begin{corollary}
    $\forall \delta > 0~ I_A(f, x) = \intl_{-\delta}^\delta{f(x - t)
    \frac{\sin{2\pi At}}{\pi t} \dd t} + o(1),~ A \to +\infty$.
\end{corollary}
\begin{proof}
    Учитывая, что $\frac{1}{\pi t} \leqslant \frac{1}{\delta}$, по теореме 
    Римана-Лебега имеем:
    \[
        \intl_{|t| \geqslant \delta}{f(x - t) \frac{\sin{2\pi At}}{\pi t}}
        \xrightarrow[A \to +\infty]{} 0
    .\]
\end{proof}

\begin{remark}
    Частичные суммы ряда Фурье представимы через ядро Дирихле:
    \[
        S_n(f, x) = \intl_{-\pi}^\pi{f(x - t) D_n(t) \dd t} =
        \intl_{-\pi}^\pi{f(x - t) \frac{\sin{nt}}{\pi t} \dd t} + o(1),~ n \to +\infty
    .\]
\end{remark}

