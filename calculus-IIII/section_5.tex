\section{Функции распределения}

\begin{definition}
    Пусть $h \colon X \to \Rbar$ -- измеримая и почти везде конечная функция,
    причем $\forall t \in \R~ \mu X(h < t) < +\infty$. Тогда
    функция $H(t) = \mu X(h < t)$ называется \textit{функцией распределения
    $h$ по мере $\mu$}.
\end{definition}

\begin{remark}
    $H(t)$ не убывает.
\end{remark}

\begin{remark}
    Пусть $h$ измерима, тогда для любого борелевского $B \in \cB(\R)$ $h^{-1}(B)$ измерим.
\end{remark}

\begin{definition}
    Стандартное продолжение $\mu_H([a, b)) = H(b - 0) - H(a - 0)$
    называется \textit{мерой Бореля-Стилтьеса}.
\end{definition}

\begin{definition}
    В текущем контексте обозначим $h(\mu) = \nu\colon \nu{A} = \mu{h^{-1}(A)}$.
\end{definition}

\begin{lemma}
    Пусть $h \colon X \to \Rbar$ -- измеримая и почти везде конечная функция, $H$
    -- её функция распределения. Тогда на $\cB(\R)$ $\mu_H = h(\mu)$.
\end{lemma}
\begin{proof}
    Проверим равенство на ячейках, чего будет достаточно для совпадения функций на $\cB(\R)$.
    \[
        \mu_H{[a, b)} = H(b - 0) - H(a - 0) = H(b) - H(a) = \mu{X(a \leqslant h < b)}
        = \mu{h^{-1}([a, b))}
    .\]
    Второе равенство выполнено как следствие непрерывности меры снизу:
    \[
        \bigcup_{t < b}{X(h < t)} = X(h < b) \Lra H(b - 0) = \lim_{t \to b_-}{H(t)} = \mu{X(h < b)} = H(b) 
    .\]
\end{proof}

\begin{theorem}
    Пусть $0 \leqslant f \colon \R \to \R$ -- функция, измеримая по Борелю,
    $h \colon X \to \Rbar$ -- измеримая и почти везде конечная функция,
    $H$ -- её функция распределения, $\mu_H$ -- мера Бореля-Стилтьеса для $H$.
    Тогда
\[
    \intl_{X}{f \circ h \dd\mu} = \intl_{\R}{f \dd\mu_H}
.\] 
\end{theorem}
\begin{proof}
    Пусть $\Phi \colon X \to \R$, $\Phi = h$, $w = 1$. Кроме того, нам по определению известно,
    что для $\nu = h(\mu)$ верно
    \[
        \nu{A} = \mu{h^{-1}(A)}
    .\]
    Это значит, что $\nu$ есть взвешенный (с весом $1$) образ меры $\mu$ при отображении $h$.
    Поэтому справедлива теорема об интегрировании по взвешенному образу меры:
    \[
        \intl_{\R}{f \dd \nu} = \intl_{X}{f \circ \Phi \dd\mu}
    .\]
    Заменив $\nu$ на $\mu_H$ по предыдущей лемме и вернув $h = \Phi$ получаем
    \[
        \intl_{\R}{f \dd\mu_H} = \intl_{X}{f \circ h \dd\mu}
    .\]
\end{proof}

