\section{Функции распределения}

\begin{definition}
    Пусть $h \colon X \to \Rbar$ -- измеримая и почти везде конечная функция,
    причем $\forall t \in \R~ \mu X(h < t) < +\infty$. Тогда
    функция $H(t) = \mu X(h < t)$ называется \textit{функцией распределения
    $h$ по мере $\mu$}.
\end{definition}

\begin{remark}
    $H(t)$ не убывает.
\end{remark}

\begin{remark}
    Пусть $h$ измерима, тогда для любого борелевского $B \in \cB(\R)$ $h^{-1}(B)$ измерим.
\end{remark}

\begin{definition}
    Стандартное продолжение $\mu_H([a, b)) = H(b - 0) - H(a - 0)$
    называется \textit{мерой Бореля-Стилтьеса}.
\end{definition}


\begin{lemma}
    Пусть $h \colon X \to \Rbar$ -- измеримая и почти везде конечная функция, $H$
    -- её функция распределения. Тогда на $\cB(\R)$ $\mu_H = H(\mu)$.
\end{lemma}

\begin{theorem}
    Пусть $0 \leqslant f \colon \R \to \R$ -- функция, измеримая по Борелю,
    $h \colon X \to \Rbar$ -- измеримая и почти везде конечная функция,
    $H$ -- её функция распределения, $\mu_H$ -- мера Бореля-Стилтьеса для $H$.
    Тогда
\[
    \intl_{X}{f \circ h \dd\mu} = \intl_{\R}{f \dd\mu_H}
.\] 
\end{theorem}

