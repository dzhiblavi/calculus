\section{Поверхностный интеграл II рода}

\begin{definition}
    \textit{Поверхностью} будем называть простое гладкое двумерное многообразие.
\end{definition}

\begin{definition}
    \textit{Стороной поверхности} называется непрерывное векторное поле единичных нормалей
    к этой поверхности.
\end{definition}

\begin{definition}
    Поверхность называется \textit{двусторонней}, если для неё существует непрерывное
    поле нормалей. Иначе она называется \textit{односторонней}.
\end{definition}

\begin{example}
    Лента Мебиуса -- односторонняя поверхность.
\end{example}

\begin{definition}
    \textit{Репером} называется пара линейно независимых касательных векторов.
\end{definition}

\begin{example}
    \enewline
    \begin{itemize}
        \item Пусть поверхность задается гладкой параметризацией $\Phi \colon \cO \to M$.
            Тогда можно задать поле нормалей, пользуясь репером:
            \[
                n = \Phi'_u \times \Phi'_v
            .\]
            Этот вектор пока не является единичной нормалью. Исправим это:
            \[
                n_0 = \frac{n}{\norm{n}}
            .\]
        \item Сторону поверхности можно задать другим способом. Рассмотрим петлю $\gamma$
            на нашей поверхности $M$. У этой петли есть вектор скорости $\gamma'$, а так
            же вектор, нормальный петле $\tau$. Перемножая эти векторы, получаем нормаль:
            \[
                n = \gamma' \times \tau
            .\]
            Таким способом удобно задавать стороны поверхностей, ограниченных кривой.
            В таких ситуациях предполгается, что задано направление пути, и нормаль должна быть
            направлена таким образом, чтобы при обходе пути по заданному направлению 
            поверхность оставалась слева, если смотреть на картинку сверху по отношению
            к нормали.
    \end{itemize}       
\end{example}

\begin{remark}
    Для гладкой параметризации способ задания поля нормалей действительно задаёт корректное
    поле в том смысле, что нормаль всегда направлена в одну сторону относительно поверхности.
    Действительно, поскольку параметризация гладкая, соответственно её производные непрерывны,
    отображение, сопоставляющее точке нормаль, непрерывно. Пусть случился разворот репера (как
    на рисунке). Тогда понятно, что между этими состояниями было состояние, в котором
    нормали были направлены в противоположные стороны, то есть линейно зависимы, чего не может
    быть в случае гладкой параметризации (мы требуем, чтобы ранг якобиана был всегда максимальным).
\end{remark}
%:: NOTE all рисунок

\begin{figure}[ht]
    \centering
    \incfig{reper}
    \caption{Задание нормали через касательные векторы.}
\end{figure}

\begin{definition}
    Пусть $\O$ -- двусторонняя поверхность в $\R^3$, $F \colon \O \to \R^3$, 
    $n_0 \colon \R^3 \to \R^3$ -- сторона поверхности. Тогда \textit{
    интегралом II рода функции $F$ по поверхности $\O$} назовем интеграл
\[
    \intl_{\O}{\langle F, n_0 \rangle \dd S}
.\] 
\end{definition}

\begin{remark}
    \enewline
    \begin{itemize}
        \item Смена стороны на противоположную влечет замену знака.
        \item Интеграл II рода не зависит от параметризации.
        \item Пусть $F = \langle P, Q, R \rangle$. Тогда
            интеграл II рода записывают так:
\[
    \intl_{\O}{\langle F, n_0 \rangle \dd S} =
    \intl_{\O}{P \dd y \d z + Q \dd z \d x + R \dd x \d y}
.\] 
        \item Пусть поверхность задана параметризацией $x(u, v), y(u, v), z(u, v)$.
            Получим нормальный вектор, перемножив векторно касательные векторы:
\[    
    n = \begin{pmatrix}
        x'_u \\
        y'_u \\
        z'_u \\
    \end{pmatrix} \times 
    \begin{pmatrix}
        x'_v \\
        y'_v \\
        z'_v \\
    \end{pmatrix} =
    \begin{pmatrix} 
        \begin{vmatrix} y'_u y'_v \\ z'_u z'_v \end{vmatrix},
        \begin{vmatrix} z'_u z'_v \\ x'_u x'_v \end{vmatrix},
        \begin{vmatrix} x'_u x'_v \\ y'_u y'_v \end{vmatrix}
    \end{pmatrix}^T
.\]
            Мера $S$ выгдялит следующим образом:
\[
    \d S = \norm{\f'_u \times \f'_v} \dd u \d v = \norm{n} \dd u \d v
.\] 
            Вычислим интеграл:
    \begin{align*}            
        \intl_{\O}{\langle F, n_0 \rangle \dd S} = 
        &\iintl_{\widetilde{\O}}{\left(P \begin{vmatrix} y'_u y'_v \\ z'_u z'_v \end{vmatrix} 
                           + Q \begin{vmatrix} z'_u z'_v \\ x'_u x'_v \end{vmatrix} 
                           + R \begin{vmatrix} x'_u x'_v \\ y'_u y'_v \end{vmatrix}\right)
                       \cdot \frac{1}{\norm{n}} \cdot \norm{n} \dd u \d v} = \\
        &\iintl_{\widetilde{\O}}{\left(P \begin{vmatrix} y'_u y'_v \\ z'_u z'_v \end{vmatrix} 
                           + Q \begin{vmatrix} z'_u z'_v \\ x'_u x'_v \end{vmatrix}
                           + R \begin{vmatrix} x'_u x'_v \\ y'_u y'_v \end{vmatrix}\right) \dd u \d v}
    \end{align*}                           
    \end{itemize} 
\end{remark}

\begin{remark}
    Посчитаем интеграл поля $\langle 0, 0, R \rangle$ по поверхности
    $\O$, заданной графиком (то есть, имеющей параметризацию вида $x, y, z(x, y)$).
%:: NOTE all рисунок
    \begin{align*}    
        \iintl_{\O^+}{R \dd x \d y} = 
        \iintl_{\widetilde{\O}}{\left(P \begin{vmatrix} y'_u y'_v \\ z'_u z'_v \end{vmatrix} 
                       + Q \begin{vmatrix} z'_u z'_v \\ x'_u x'_v \end{vmatrix}
                       + R \begin{vmatrix} x'_u x'_v \\ y'_u y'_v \end{vmatrix}\right) \dd u \d v} =
        \iintl_{\widetilde{\O}}{R(x, y, z(x, y)) \dd u \d v} 
    \end{align*}
\end{remark}

\begin{remark}
    Попробуем посчитать объём фигуры $\O$, ограниченной графиками $z_1, z_2$:
%:: NOTE all рисунок
    \begin{align*}
        \l_3(\O) = &\iintl_{\widetilde{\O}}{(z_1(x, y) - z_2(x, y)) \dd x \d y} =
        \iintl_{\widetilde{\O}}{z_1(x, y) \dd x \d y} - \iintl_{\widetilde{\O}}{z_2(x, y) \dd x \d y} \\ = 
            &\iintl_{\partial \O^+}{z \dd x \d y}
    \end{align*}
    В последнем переходе мы воспользовались предыдущим замечанием и тем, что у нижней
    части фигуры (ограниченной $z_1$) нормали направлены в другую сторону.
\end{remark}

\begin{remark}
    Пусть $\gamma$ -- гладкая кривая в $\R^2$ (лежит в плоскости $xy$), $\O$ -- цилиндр над $\gamma$. Тогда
\[
    \iintl_{\O}{R \dd x \d y} = 0
.\] 
\end{remark}
\begin{proof}
    \enewline

    \textit{Первое доказательство:} по формуле из первого замечания мы собираемся интегрировать
    какую-то функцию по носителю пути по двумерной мере. Носитель гладкого пути по такой мере
    всегда имеет меру 0. \\
    \textit{Второе доказательство:} мы пытаемся интегрировать $\langle F, n_0 \rangle$.
    Заметим, что у $F$ не равна нулю только третья координата ($R$), тогда как
    вектор нормали к цилиндру над $xy$ всегда имеет $z = 0$. Таким
    образом, мы интегрируем функцию, тождественно равную нулю.
\end{proof}

