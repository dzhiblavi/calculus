\section{Тригонометрические ряды Фурье}

\begin{definition}
    \textit{Тригонометрическим полиномом порядка не выше $n$} называется
    \[
        T_n(x) = \frac{a_0}{2} + \sum_{k = 1}^n{a_k \cos{kx} + b_k \sin{kx}}
    .\]
\end{definition}

\begin{definition}
    \textit{Тригонометрическим рядом} называется
    \[
        T(x) = \frac{a_0}{2} + \sum_{k = 1}^{+\infty}{a_k \cos{kx} + b_k \sin{kx}}
    .\]
\end{definition}

\begin{remark}
    Эквивалентной записью тринометрических полиномов или рядов являются записи
    \[
        S_n(x) = \sum_{k = -n}^n{c_k e^{ikx}},~~ S(x) = \sum_{k \in \bZ}{c_k e^{ikx}}
        \defeq \lim_{n \to +\infty}{S_n(x)}
    .\]
\end{remark}

\begin{lemma}
    Пусть $f$ -- тригонометрический ряд: $S_n \to f$ в $L^1[-\pi, \pi]$. Тогда
    \[
        a_k = \frac{1}{\pi}\intl_{-\pi}^\pi{f(t) \cos{kt} \dd t},~
        b_k = \frac{1}{\pi}\intl_{-\pi}^\pi{f(t) \sin{kt} \dd t},~
        c_k = \frac{1}{2\pi}\intl_{-\pi}^\pi{f(t) e^{-ikt} \dd t}
    .\]
\end{lemma}
\begin{proof}
    Докажем первую формулу. Пусть $n > k$:
    \begin{align*}
        &\intl_{-\pi}^\pi{S_n(t) \cos{kt} \dd t} = \intl_{-\pi}^\pi{
            \parens*{\frac{a_0}{2} + \sum_{l = 1}^n{\parens*{a_l \cos{lt} + b_l 
        \sin{lt}}}} \cos{kt} \dd t} \\
        &= \intl_{-\pi}^\pi{a_k \cos^2{kt} \dd t} = \pi a_k
    .\end{align*}
    \begin{align*}
        \abs*{\intl_{-\pi}^\pi{S_n(t) \cos{kt} \dd t} - \intl_{-\pi}^\pi{f(t) \cos{kt} \dd t}}
        &\leqslant \intl_{-\pi}^\pi{\abs*{\cos{kt} (S_n(t) - f(t))} \dd t} =
        \intl_{-\pi}^\pi{|\cos{kt}| |S_n(t) - f(t)| \dd t} \\
        &= \intl_{-\pi}^\pi{|S_n(t) - f(t)| \dd t} = \norm{S_n - f} \to 0
    .\end{align*}
    Мы только что показали, что
    \[
        \pi a_k = \intl_{-\pi}^\pi{S_n(t) \cos{kt} \dd t} \to
        \intl_{-\pi}^\pi{f(t) \cos{kt} \dd t}
    ,\]
    что и требовалось.
\end{proof}

\begin{definition}
    Пусть $f \in L^1[-\pi, \pi]$. Тогда \textit{коэффициентами Фурье} функции $f$
    называются $a_k(f), b_k(f)$, получаемые из предыдущей леммы.
\end{definition}

\begin{definition}
    Пусть $f \in L^1[-\pi, \pi]$. Тогда \textit{рядом Фурье} функции $f$
    называется ряд
    \[
        \frac{a_0(f)}{2} + \sum_{k = 1}^{+\infty}{a_k(f) \cos{kt} + b_k(f) \sin{kt}}
    .\]
\end{definition}

\begin{remark}
    Далее будем обозначать
    \[
        L_k = L^k[-\pi, \pi]
    .\]
\end{remark}

\begin{remark}
    Для четных функций верно, что $b_k(f) = 0$.
\end{remark}

\begin{remark}
    Функциям из $L^1[0, \pi]$ можно сопоставлять ряды
    \[
        \frac{a_0}{2} + \sum_k{a_k \cos{kx}},~~ \frac{a_0}{2} + \sum_k{b_k \sin{kx}}
    .\]
\end{remark}

\begin{definition}
    Введем обозначение
    \[
        A_k(f, x) = \begin{cases}
            \frac{a_0(f)}{2},~ k = 0 \\
            a_k(f) \cos{kx} + b_k(f) \sin{kx},~ k > 0
        \end{cases}
    .\]
\end{definition}

\begin{lemma}
    \[
        A_k(f, x) = \begin{cases}
            \frac{1}{2\pi} \intl_{-\pi}^\pi{f(x + t) \dd t},~ k = 0 \\
            \frac{1}{\pi} \intl_{-\pi}^\pi{f(x + t) \cos{kt} \dd t},~ k > 0 \\
        \end{cases}
    .\]
\end{lemma}
\begin{proof}
    Покажем вторую формулу.
    \begin{align*}
        a_k(f) \cos{kx} + b_k(f) \sin{kx} 
        &= \frac{1}{\pi} \intl_{-\pi}^\pi{\parens*{
        f(t) \cos{kt} \cos{kx} + f(t) \sin{kt} \sin{kx}} \dd t} \\
        &= \frac{1}{\pi} \intl_{-\pi}^\pi{f(t) \cos{(k(t - x))} \dd t} =
        \frac{1}{\pi} \intl_{-\pi}^\pi{f(x + t) \cos{kt} \dd t}
    .\end{align*}
\end{proof}

\begin{lemma}
    \begin{align*}
        |a_k(f)|, |b_k(f)|, 2|c_k(f)| \leqslant \frac{1}{\pi} \norm{f}_1 \\
        |A_k(f, x)| \leqslant \begin{cases}
            \frac{1}{2\pi} \norm{f}_1,~ k = 0 \\
            \frac{1}{\pi} \norm{f}_1,~ k > 0
        \end{cases}
    .\end{align*}
\end{lemma}
\begin{proof}
    Докажем первое неравенство:
    \[
        \pi|a_k(f)| = \abs*{\intl_{-\pi}^\pi{f(t) \cos{kt} \dd t}} \leqslant
        \intl_{-\pi}^\pi{|f(t)| |\cos{kt}| \dd t} \leqslant
        \intl_{-\pi}^\pi{|f(t)| \dd t} = \norm{f}_1
    .\]
\end{proof}

\begin{remark}(Дюбуа-Реймон)
    Существует $f \in \widetilde{C}[-\pi, \pi]$ и такая точка $x_0$, что
    ряд Фурье $f$ расходится в $x_0$.
\end{remark}

\begin{remark}(Колмогоров)
    Существует $f \in L_1$ такая, что её ряд Фурье расходится всюду.
\end{remark}

\begin{theorem}(Карлесон)
    $f \in L^2 \Lra$ ряд Фурье $f$ сходится почти всюду.
\end{theorem}

\begin{theorem}(Хант)
    $f \in L^p$, $1 < p < +\infty \Lra$ ряд Фурье $f$ сходится почти всюду.  
\end{theorem}

\begin{theorem}(Риман, Лебег)
   
    Пусть $E$ -- измеримое множество в $\R$ по мере Лебега, $f \in L^1(E, \lambda_1)$,
    $\lambda \neq 0$ Тогда
    \[
        \intl_E{f(t) e^{i \lambda t} \dd t} \xrightarrow[\lambda \to +\infty]{} 0,~~
        \intl_E{f(t) \cos{\lambda t} \dd t} \xrightarrow[\lambda \to +\infty]{} 0,~~
        \intl_E{f(t) \sin{\lambda t} \dd t} \xrightarrow[\lambda \to +\infty]{} 0
    .\]
\end{theorem}
\begin{proof}
    Занулим $f$ на дополнении $E$, это никак не скажется на интеграле по $E$.
    Теперь гарантированно $f \in L^1(\R, \lambda_1)$.
    \[
        \intl_\R{f(t) e^{i \lambda t} \dd t} = \intl_\R{f\parens*{\tau + \frac{\pi}{\lambda}}
        e^{i \lambda \tau + i \pi} \dd \tau} = -\intl_\R{f\parens*{\tau + \frac{\pi}{\lambda}} 
        e^{i \lambda \tau}\dd \tau}
    .\]
    Поэтому
    \begin{align*}
        &2 \intl_\R{f(t) e^{i \lambda t} \dd t} = \intl_\R{\parens*{f(t) -
        f\parens*{t + \frac{\pi}{\lambda}}} e^{i \lambda \tau} \dd t} \\
        &\Lra \abs*{\intl_\R{f(t) e^{i \lambda t} \dd t}} \leqslant
        \frac{1}{2} \intl_\R{\abs*{f(t) - f\parens*{t + \frac{\pi}{\lambda}}}
        |e^{i \lambda t}| \dd t} = \frac{1}{2} \norm{f - f_{\frac{\pi}{\lambda}}}_1 
        \xrightarrow[\frac{\pi}{\lambda} \to 0]{} 0
    .\end{align*}
    Предельный переход выполнен в силу теоремы о непрерывности сдвига.
\end{proof}

\begin{remark}
    $f \in L^1(\Rm), y \in \Rm \Lra I(y) = \intl_{\Rm}{f(x) e^{i \scp{x}{y}} \dd x}
    \xrightarrow[|y| \to +\infty]{} 0$. Это утверждение тоже называется теоремой
    Римана-Лебега и доказывается примерно так же, как и только что доказанная
    теорема, только сдвиг делается такой:
    \[
        x = x + \frac{\pi y}{\norm{y}^2} 
    .\]
\end{remark}

\begin{definition}
    \textit{Модулем непрерывности} называется функция
    \[
        \omega(f, h) \defeq \sup_{\substack{x, y \in E \\ \norm{x - y} < h}}{|f(x) - f(y)|}
    .\]
\end{definition}

\begin{remark}
    Если $f$ равномерно непрерывна, то $\omega(f, h) \xrightarrow[h \to 0]{} 0$.
\end{remark}

\begin{remark}
    Если $f$ дифференцируема на $[a, b]$, то $\omega(f, h) \leqslant \max_{[a, b]}{f'} \cdot h$.
\end{remark}

\begin{lemma}
    Пусть $f \in \widetilde{C}[-\pi, \pi]$. Тогда
    \[
        |a_k(f)|, |b_k(f)|, 2|c_k(f)| \leqslant \omega\parens*{f, \frac{\pi}{k}}
    .\]
    Повторим выкладки из теоремы Римана-Лебега, только вместо интегрирования по $\R$
    и экспоненты будем писать интергирование по $[-\pi, \pi]$ и косинус. Получим
    \[
        \pi a_k(f) = \abs*{\intl_{-\pi}^\pi{f(t) \cos{kt} \dd t}} \leqslant \frac{1}{2}
        \intl_{-\pi}^\pi{\abs*{f(t) - f\parens*{t + \frac{\pi}{k}}}} \leqslant
        \frac{1}{2} \intl_{-\pi}^\pi{\omega\parens*{f, \frac{\pi}{k}}\dd t} =
        \pi \omega\parens*{f, \frac{\pi}{k}}
    .\]
\end{lemma}

\begin{definition}
    Пусть $E = \langle a, b \rangle$, $M \in \R$, $\a \in (0, 1)$. Тогда
    \textit{классом Липшица с константой $M$ и показателем $\a$}
    называется класс функций
    \[
        \Lip_M(\a) \defeq \{\, f\colon E \to \R \mid \forall x, y \in E~
        |f(x) - f(y)| \leqslant M |x - y|^\a \,\} 
    .\]
\end{definition}

\begin{remark}
    $f$ дифференцирума, $|f'| \leqslant M \Lra f \in \Lip_M(1)$.
\end{remark}

\begin{remark}
    Если $f \in \Lip_M(\a)$, $\a > 1$, то $f \equiv C$.
\end{remark}

\begin{remark}
    $f \in \Lip_M(\a) \Lra \omega(f, h) \leqslant M h^\a$.
\end{remark}

\begin{corollary}
    Пусть $0 < \a \leqslant 1$, $f \in \Lip_M(\a)$, тогда при $k \neq 0$
    справедливы оценки
    \[
        |a_k(f)|, |b_k(f)|, 2|c_k(f)| \leqslant \frac{M \pi^\a}{k^\a}
    .\]
\end{corollary}
\begin{proof}
    Очевидно в свете последней леммы и определений.
\end{proof}

\begin{lemma}
    Пусть $f \in \widetilde{C}^1[a, b]$, тогда
    \[
        a_k(f') = k b_k(f),~~ b_k(f') = -k a_k(f),~~ c_k(f') = ik c_k(f)
    .\]
\end{lemma}
\begin{proof}
    Проверим последнее равенство.
    \[
        2 \pi c_k(f') = \intl_{-\pi}^\pi{f'(x) e^{-ikx}} = f(x) e^{-ikx}\big|_{-\pi}^\pi +
        ik \intl_{-\pi}^\pi{f(x) e^{-ikx}} = 0 + ik \cdot 2 \pi c_k(f)
    .\]
\end{proof}

\begin{corollary}
    Пусть $f \in \widetilde{C}^r[-\pi, \pi]$. Тогда
    \begin{itemize}
        \item $|a_k(f)|, |b_k(f)|, 2|c_k(f)| \leqslant \frac{C}{|k|^r}$.
        \item Если $f^{(r)} \in \Lip_M(\a)$, то $|a_k(f)|, |b_k(f)|, 2|c_k(f)|
            \leqslant \frac{M \pi^\a}{|k|^{r + \a}}$.
    \end{itemize}
\end{corollary}
\begin{proof}
    $f^{(r)}$ непрерывна, поэтому её коэффициенты Фурье стремятся к нулю, а значит,
    ограничены константой $C$. Кроме того, каждый раз при увеличении порядка
    производной ``выскакивает'' очередное $k$. Поэтому
    \[
        a_k(f) \leqslant \frac{1}{k^r} C 
    ,\]
    откуда следует первая оценка. Вторая оценка получается из первой с учетом
    более строгой оценки на коэффициенты Фурье производной:
    \[
        a_k(f^{(r)}) \leqslant \frac{M \pi^\a}{k^\a}
    .\]
\end{proof}

\begin{corollary}
    Если $f \in C^2$, то ряд Фурье $f$ сходится.
\end{corollary}

\begin{definition}
    \textit{Ядром Дирихле} называется функия
    \[
        D_n(t) = \frac{1}{\pi}\parens*{\frac{1}{2} + \sum_{k = 1}^n{\cos{kt}}},~ n \geqslant 0
    .\]
\end{definition}

\begin{definition}
    \textit{Ядром Фейера} называется функция
    \[
        \Phi_n(t) = \frac{1}{n + 1} \sum_{m = 0}^n{D_m(t)}
    .\]
\end{definition}

\begin{lemma}
    \[
        D_n(t) = \frac{\sin{\parens*{n + \frac{1}{2}} t}}{2 \pi \sin{\frac{t}{2}}},~~
        \Phi_n(t) = \frac{1}{2\pi (n + 1)} \frac{\sin^2{\frac{n + 1}{2} t}}{\sin^2{\frac{t}{2}}}
    .\]
\end{lemma}
\begin{proof}
    Используя тот факт, что
    \[
        2 \sin{\frac{t}{2} \cos{kt}} = \sin{\parens*{k + \frac{1}{2} t}}
        - \sin{\parens*{k - \frac{1}{2} t}}
    \]
    получаем телескопическую сумму:
    \begin{align*}
        2 \pi \sin{\frac{t}{2}} \cdot D_n(t) = 2 \pi \sin{\frac{t}{2}} \cdot
        \frac{1}{\pi} \parens*{\frac{1}{2} + \sum_{k = 1}^n{\cos{kt}}} =
        \sin{\frac{t}{2}} + \sum_{k = 1}^n{2 \sin{\frac{t}{2}} \cos{kt}} =
        \sin{\parens*{n + \frac{1}{2}} t}
    .\end{align*}
    Для ядра Фейера имеем
    \[
        2\pi (n + 1) \Phi_n(t) = \sum_{k = 0}^n{\frac{\sin{\parens*{k + \frac{1}{2}} t}}
        {\sin{\frac{t}{2}}}}
    .\]
    Сравним последнюю сумму с выражением
    \[
        \frac{\sin^2{\frac{n + 1}{2} t}}{\sin^2{\frac{t}{2}}}
    .\]
    Для этого домножим обе части на $\sin^2{\frac{t}{2}}$.
    \[
        2 \sum_{k = 0}^n{\sin^2{\frac{n + 1}{2} t} \sin{\frac{t}{2}}} =
        \sum_{k = 0}^n{\cos{kt} - \cos{(k + 1) t}} = (1 - \cos{(n + 1) t}) =
        2 \sin^2{\frac{n + 1}{2} t}
    .\]
\end{proof}

\begin{corollary}
    \enewline
    \begin{itemize}
        \item $D_n, \Phi_n$ -- четные функции.
        \item $\Phi_n \geqslant 0$.
        \item
            \[
                \intl_{-\pi}^\pi{D_n(t) \dd t} = 1,~~
                \intl_{-\pi}^\pi{\Phi_n(t) \dd t} = 1
            .\]
        \item $f \in L_1$, тогда
            \[
                S_n(f, x) = \intl_{-\pi}^\pi{f(x + t) D_n(t) \dd t} 
            .\]
    \end{itemize}
\end{corollary}
\begin{proof}
    Покажем последнее утверждение. Вычислим частичную сумму:
    \[
        S_n(f, x) = \sum_{k = 0}^n{A_k(f, x)} 
    .\]
    Помня про представление $A_k(f, x)$ в виде
    \[
        A_k(f, x) = \begin{cases}
            \frac{1}{2\pi} \intl_{-\pi}^\pi{f(x + t) \dd t},~ k = 0 \\
            \frac{1}{\pi} \intl_{-\pi}^\pi{f(x + t) \cos{kt} \dd t},~ k > 0
        \end{cases}
    ,\]
    получаем требуемое
\end{proof}

\begin{definition}
    \textit{Интегралом Дирихле} называется интеграл
    \[
        \intl_{-\pi}^\pi{f(x + t) D(t) \dd t}
    .\]
\end{definition}

\begin{theorem}(Принцип локализации Римана)

    Пусть $f, g \in L_1$, $x_0 \in \R$, $\delta > 0$, причем
    \[
        \forall x \in (x_0 - \delta, x_0 + \delta)~ f(x) = g(x)
    .\]
    Тогда ряды Фурье этих функций ведут себя одинаково в $x_0$, то есть
    \[
        \abs*{S_n(f, x_0) - S_n(g, x_0)} \xrightarrow[n \to +\infty]{} 0
    .\]
\end{theorem}
\begin{proof}
    Достаточно проверить, что если $h = 0$ на $(x_0 - \delta, x_0 + \delta)$,
    то $S_n(g, x_0) \to 0$. Для начала заметим
    \[
        \frac{\sin{\parens*{n + \frac{1}{2}} t}}{\sin{\frac{t}{2}}} =
        \cot{\frac{t}{2}} \sin{nt} + \cos{nt}
    .\]
    Далее имеем
    \begin{align*}
        S_n(h, x_0) 
        &= \intl_{-\pi}^\pi{h(x_0 + t) D_n(t) \dd t} =
        \frac{1}{\pi} \intl_{-\pi}^\pi{(\underbrace{h(x_0 + t) \cot{\frac{t}{2}}}_{h_1(t)}
        \sin{nt} + \underbrace{h(x_0 + t)}_{h_2(t)} \cos{nt}) \dd t} \\ 
        &= b_n(h_1) + a_n(h_2) \xrightarrow[n \to +\infty]{} 0
    .\end{align*}
    Чтобы все стало корректно, нужно проверить, что $h_1, h_2 \in L_1$. В случае
    с $h_2$ это очевидно, проверим, что $h_1 \in L_1$. Для этого промажорируем
    её функцией из $L_1$:
    \[
        |h_1(t)| = \abs*{h(x_0 + t) \cot{\frac{t}{2}}} \leqslant \begin{cases}
            0,~ |t| < \delta \\
            |h(x_0 + t)| \cot{\frac{\delta}{2}},~ |t| \geqslant \delta
        \end{cases}
    .\]
\end{proof}

\begin{corollary}
    С условиях теоремы сходимости рядов Фурье функций $f$ и $g$ в точке $x_0$
    эквивалентны, причем в случае сходимости суммы одинаковы.
\end{corollary}

\begin{remark}
    Для определения всего ряда Фурье нужны значения функции на $[-\pi, \pi]$.
    Для определения же значения ряда Фурье в конкретной точке нужны значения
    всего лишь в окрестности этой точки.
\end{remark}

