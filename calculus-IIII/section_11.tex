\section{Гильбертово пространство}

\begin{definition}
    \textit{Гильбертовым пространством} называется линейное пространство со
    скалярным произведением, полное как линейное нормированное пространство.
\end{definition}

\textit{Далее $\cH$ -- гильбертово пространство.}

\begin{definition}
    Пусть $a_n \in \cH$, тогда ряд
    \[
        \sum_{n = 1}^{+\infty}{a_n}
    \]
    называется \textit{сходящимся}, если существует предел последовательности
    \[
        S_N = \sum_{n = 1}^N{a_n} 
    .\]
    Если ряд сходится, то соответствующий предел называется \textit{суммой ряда}.
\end{definition}

\begin{definition}
    $x, y \in \cH$ называются \textit{ортогональными}, если их скалярное
    произведение равно нулю:
    \[
        x \bot y \Llra \scp{x}{y} = 0
    .\]
\end{definition}

\begin{definition}
    Элемент $x \in \cH$ называется \textit{ортогональным множеству} $A \subset \cH$,
    если
    \[
        \forall y \in A~ x \bot y
    .\]
\end{definition}

\begin{definition}
    Ряд $\sum{a_n}$ называется \textit{ортогональным}, если
    \[
        \forall i, j \neq i~ a_i \bot a_j 
    .\]
\end{definition}

\begin{theorem}(Свойства сходимости в гильбертовом пространстве) 

    \label{th:pytha} Пусть $x_i, y_i \in \cH$. Тогда
    \begin{itemize}
        \item $x_n \to x_0, y_n \to y_0 \Lra \scp{x_n}{y_n} \to \scp{x_0}{y_0}$.
        \item $\sum{x_k}$ сходится, тогда
            \[
                \forall y \in \cH~ \scp{\sum{x_k}}{y} = \sum{\scp{x_k}{y}}
            .\]
        \item $\sum{x_k}$ -- ортогональный ряд. Тогда сходимость этого ряда
            эквивалентна условию
            \[
                \sum{\norm{x_k}^2} < +\infty
            ,\]
            и при этом
            \[
                \norm{\sum{x_k}}^2 = \sum{\norm{x_k}^2}
            .\]
    \end{itemize}
\end{theorem}
\begin{proof}
    \enewline
    \begin{itemize}
        \item 
            \begin{align*}
                |\scp{x_n}{y_n} - \scp{x_0}{y_0}| 
                &\leqslant |\scp{x_n}{y_n} - \scp{x_n}{y_0}| + 
                |\scp{x_n}{y_0} - \scp{x_0}{y_0}| \\
                &= |\scp{x_n}{y_n - y_0}| + |\scp{x_n - x_0}{y_0}| \\
                &\leqslant \norm{x_n} \norm{y_n - y_0} + \norm{x_n - x_0} \norm{y_0}
                \xrightarrow[n \to +\infty]{} 0
            .\end{align*}
        \item Обозначим $S_N = \sum_{n = 1}^N{x_n}$, $S = \sum_{n = 1}^{+\infty}{x_n}$.
            Тогда
            \[
                \xymatrix{
                    \scp{S_N}{y} \ar[d]_{n \to +\infty} & = & \sum_{k = 1}^N
                                                        {\scp{x_k}{y}} \ar[d]_{n \to +\infty} \\
                    \scp{S}{y} & \overset{!}{=} & \sum_{k = 1}^{+\infty}{\scp{x_k}{y}}
                }
            .\]
        \item Обозначим $S_N = \sum_{k = 1}^N{x_k}$. Тогда
            \begin{align*}
                \norm{S_N}^2 = \scp{S_N}{S_N} = \scp{\sum_{k = 1}^N{x_k}}{\sum_{k = 1}^N{x_k}} =
                \sum_{k, l = 1}^N{\scp{x_k}{x_l}} = \sum_{k = 1}^N{\scp{x_k}{x_k}} = 
                \sum_{k = 1}^N{\norm{x_k}^2} = \Sigma_N
            .\end{align*}
            Аналогично:
            \[
                \norm{S_N - S_M}^2 = |\Sigma_N - \Sigma_M|
            .\]
            Таким образом, последовательности $S_n$, $\Sigma_n$ фундаментальны
            (или не фундаментальны) одновременно. Имея в виду полноту как $\cH$,
            так и $\R$, получаем, что эти последовательности сходятся (или
            расходятся) одновременно.
    \end{itemize}
\end{proof}

\begin{definition}
    Система векторов $\{\,e_k\,\}$ называется \textit{ортогональной},
    если
    \[
        \forall i, j \neq i~ e_i \bot e_j
    .\]
\end{definition}

\begin{definition}
    Система векторов $\{\,e_k\,\}$ называется \textit{ортонормированной},
    если она ортогональна, причем $\forall k~ \norm{e_k} = 1$.
\end{definition}

\begin{theorem}(О свойствах разложения по ортогональной системе)
    
    Пусть $E = \{\,e_k\,\}$ -- ортогональная система в $\cH$, $x \in \cH$,
    $x = \sum_{k = 1}^{+\infty}{c_k e_k}$. Тогда
    \begin{itemize}
        \item $E$ -- линейно независимая система.
        \item \label{th:Hortho:2} $c_k = \frac{\scp{x}{e_k}}{\norm{e_k}^2}$.
        \item $c_k e_k$ есть ортогональная проекция $x$ на прямую
            $\{\,t e_k\,\}_{t \in \R}$, то есть
            \[
                x = c_k e_k + z,~~ z \bot e_k
            .\]
    \end{itemize}
\end{theorem}
\begin{proof}
    \enewline
    \begin{itemize}
        \item Пусть
            \[
                \sum_{k = 1}^N{\a_k e_k} = 0
            .\]
            Домножим это равенство на $e_j$:
            \[
                0 = \sum_{k = 1}^N{\a_k \scp{e_k}{e_j}} = \a_j \norm{e_j} \Lra
                \a_j = 0
            .\]
        \item
            \begin{align*}
                \scp{x}{e_m} = \scp{\sum{c_k e_k}}{e_m} = \sum{c_k \scp{e_k}{e_m}} =
                c_k \scp{e_m}{e_m} \Lra c_k = \frac{\scp{x}{e_m}}{\norm{e_m}^2}
            .\end{align*}
        \item Надо проверить, что $z = x - c_k e_k \bot e_k$.
            \[
                \scp{x - c_k e_k}{e_k} = \scp{x}{e_k} - c_k \scp{e_k}{e_k} =
                \scp{x}{e_k} - c_k \norm{e_k}^2 = 0
            .\]
    \end{itemize}
\end{proof}

\section{Ряды Фурье}

\begin{definition}
    Пусть $\{\,e_k\,\}$ -- ортонормированная система, $x \in \cH$. Тогда
    \[
        c_k(x) \defeq \frac{\scp{x}{e_k}}{\norm{e_k}^2}    
    \]
    называется \textit{коэффициентом Фурье $x$ по системе $e_k$}.
\end{definition}

\begin{definition}
    Ряд
    \[
        \sum_k{c_k(x) e_k}
    \]
    называется \textit{рядом фурье $x$ по системе $e_k$}.
\end{definition}

\begin{remark}
    Ряд Фурье не меняется при перенормировке $e_k$.
\end{remark}

\begin{theorem}(О свойствах частичных сумм ряда Фурье)

    Пусть $\{\,e_k\,\}$ -- ортонормированная система в $\cH$, $n \in \bN$,
    $S_n = \sum_{k = 1}^n{c_k(x) e_k}$, $\cL = \Lin{\parens*{e_1, \ldots, e_n}}$.
    Тогда
    \begin{itemize}
        \item $S_n = \cP_\cL(x)$, то есть $x = S_n + z,~~ z \bot \cL$.
        \item $S_n$ -- елемент наилучшего приближения $x$ в $\cL$, то есть
            \[
                \forall y \in \cL~ \norm{x - S_n} \leqslant \norm{x - y}
            .\]
        \item $\norm{S_n} \leqslant \norm{x}$.
    \end{itemize}
\end{theorem}
\begin{proof}
    \enewline
    \begin{itemize}
        \item Как и в предыдущей теореме, надо проверить, что $z = x - S_n \bot \cL$.
            Для этого проверим, что $\forall k = 1..n~ e_k \bot z$.
            \begin{align*}
                \scp{z}{e_k} 
                &= \scp{x}{e_k} - \scp{S_n}{e_k} = \scp{x}{e_k} -
                \sum_{i = 1}^n{\scp{c_i(x) e_i}{e_k}} = \scp{x}{e_k} -
                \sum_{i = 1}^n{c_i(x) \scp{e_i}{e_k}} \\
                &= \scp{x}{e_k} - \scp{x}{e_k} = 0
            .\end{align*}
        \item Пусть $y \in \cL$:
            \begin{align*}
                \norm{x - y}^2 = \norm{S_n + z - y}^2 = \norm{S_n - y}^2 + \norm{z}^2
                \geqslant \norm{z}^2 = \norm{S_n - x}^2
            .\end{align*}
        \item $\norm{x}^2 = \norm{S_n}^2 + \norm{z}^2 \geqslant \norm{S_n}^2$.
    \end{itemize}
\end{proof}

\begin{corollary}(Неравенство Бесселя)
    В условиях предыдущей теоремы выполняется
    \[
        \sum_{k = 1}^{+\infty}{|c_k(x)|^2 \norm{e_k}^2} \leqslant \norm{x}^2
    .\]
\end{corollary}
\begin{proof}
    Из последнего пункта предыдущей теоремы для любого $n$ имеем
    \[
        \norm{x}^2 \geqslant \sum_{k = 1}^n{|c_k(x)|^2 \norm{e_k}^2}
    .\]
    Переходя к пределу при $n \to +\infty$, получаем требуемое.
\end{proof}

\begin{theorem}(Рисс, Фишер)
    
    Пусть $\{\,e_k\,\}$ -- ортогональая система в $\cH$, $x \in \cH$. Тогда
    \begin{itemize}
        \item Ряд Фурье $x$ сходится в $\cH$.
        \item $x = \sum_{k = 1}^{+\infty}{c_k(x) e_k} + z,~~ \forall k~ z \bot e_k$.
        \item $x = \sum_{k = 1}^{+\infty}{c_k(x) e_k} \Llra \norm{x}^2 =
            \sum_{k = 1}^{+\infty}{|c_k(x)|^2 \norm{e_k}^2}$.
    \end{itemize}
\end{theorem}
\begin{proof}
    \enewline
    \begin{itemize}
        \item Сходимость ортогонального ряда эквивалентная сходимости ряда
            из квадратов норм его элементов. То есть, сходимость ряда Фурье $x$
            эквивалентна сходимости ряда
            \[
                \sum_{k = 1}^{+\infty}{|c_k(x)|^2 \norm{e_k}^2}
            .\]
            Этот ряд сходится потому, что
            \[
                \sum_{k = 1}^{+\infty}{|c_k(x)|^2 \norm{e_k}^2} \leqslant \norm{x}^2
                < +\infty
            .\]
        \item $\scp{z}{e_n} = \scp{x}{e_n} - \sum{\scp{c_k(x) e_k}{e_n}} =
            \scp{x}{e_n} - \scp{x}{e_n} = 0$.
        \item \begin{itemize}
            \item[$\Lra$] Это -- третий пункт теоремы \ref{th:pytha}.
            \item[$\Lla$] Пусть выполнено строгое неравенство. Тогда
                \[
                    \norm{\sum{\ldots}} = \norm{x}^2 = \norm{\sum{\ldots}} 
                    + \norm{z}^2 > \norm{\sum{\ldots}}
                .\]
        \end{itemize}
    \end{itemize}
\end{proof}

\begin{definition}
    \textit{Равенством Парсиваля}, или \textit{уравнением замкнутости} называется
    равенство
    \[
        \sum_{k = 1}^{+\infty}{|c_k(x)|^2 \norm{e_k}^2} = \norm{x}^2
    .\]
\end{definition}

\begin{definition}
    Ортогональная система $\{\,e_k\,\}$ называется \textit{базисом}, если
    \[
        \forall x \in \cH~ x = \sum_{k = 1}^{+\infty}{c_k(x) e_k}
    .\]
\end{definition}

\begin{definition}
    Ортогональная система $E$ называется \textit{полной}, если
    \[
        \not \exists x \in \cH\colon~ E \cup \{x\} \text{ -- ортогональная система}
    .\]
\end{definition}

\begin{definition}
    Ортогональная система называется \textit{замкнутой}, если в ней для любого
    элемента $\cH$ выполняется равенство Парсиваля.
\end{definition}

\begin{theorem}(О характеристике базиса)
    
    Пусть $\{\,e_k\,\}$ -- ортогональная система. Тогда эквивалентны
    утверждения
    \begin{enumerate}
        \item $\{\,e_k\,\}$ -- базис.
        \item $\forall x, y \in \cH~ \scp{x}{y} = \sum_{k = 1}^{+\infty}{c_k(x)
            \overline{c_k(y)} \cdot \norm{e_k}^2}$.
        \item $\{\,e_k\,\}$ замкнута.
        \item $\{\,e_k\,\}$ полна.
        \item $\Lin{\parens*{e_1, \ldots}}$ плотно в $\cH$, то есть
            \[
                \Cl(\Lin{\parens*{e_1, \ldots}}) = \cH
            .\]
    \end{enumerate}
\end{theorem}
\begin{proof}
    \enewline
    \begin{itemize}
        \item[$1 \Lra 2$] Так как система является базисом,
            \[
                \scp{x}{y} = \scp{\sum{c_k(x) e_k}}{y}
            .\]
            Далее нетрудно получить требуемое:
            \[
                \scp{\sum{c_k(x) e_k}}{y} = \sum{c_k(x) \scp{e_k}{y}} =
                \sum{c_k(x) \cdot \norm{e_k}^2 \overline{c_k(y)}}
            .\]
        \item[$2 \Lra 3$] Проверим равенство Парсиваля:
            \[
                \norm{x}^2 = \scp{x}{x} = \sum_{k = 1}^{+\infty}{c_k(x) c_k(x) 
                \norm{e_k}^2} = \sum_{k = 1}^{+\infty}{|c_k(x)| \norm{e_k}^2}
            .\]
        \item[$3 \Lra 4$] Надо проверить, что $\forall k~ z \bot e_k \Lra z = 0$.
            Для начала заметим, что
            \[
                \forall k~ c_k(z) = \frac{\scp{z}{e_k}}{\norm{e_k}^2} = 0
            .\]
            Воспользуемся условием замкнутости:
            \[
                \norm{z}^2 = \sum_{k = 1}^{+\infty}{|c_k(z)|^2 \norm{e_k}^2} = 0
                \Lra z = 0
            .\]
        \item[$4 \Lra 1$] По теореме Рисса-Фишера имеем
            \[
                x = \sum_{k = 1}^{+\infty}{c_k(x) e_k} + z,~~ \forall k~ z \bot e_k
            .\]
            Система полна, поэтому в $\cH$ нет нелулевых элементов, ортогональных
            сразу всем $e_k$. То есть, $z = 0$.
        \item[$1 \Lra 5$] Очевидно, что $\Cl(\cL) \subset \cH$. Проверим, что
            $x \in \cH \Lra x \in \Cl(\cL)$. По определению базиса имеем
            \[
                x = \sum_{k = 1}^{+\infty}{c_k(x) e_k} = \lim_{N \to +\infty}
                {\sum_{k = 1}^N{c_k(x) e_k}} \in \Cl(\cL)
            .\]
        \item[$5 \Lra 4$] Пусть $\forall k~ y \bot e_k$. Тогда
            \[
                y \bot \cL \Lra y \bot \cH \Lra y \bot y \Lra \scp{y}{y} = 0 \Lra y = 0
            .\]
    \end{itemize}
\end{proof}

