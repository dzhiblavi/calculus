\section{Гильбертово пространство}

\begin{definition}
    \textit{Гильбертовым пространством} называется линейное пространство со
    скалярным произведением, полное как линейное нормированное пространство.
\end{definition}

\textit{Далее $\cH$ -- гильбертово пространство.}

\begin{definition}
    Пусть $a_n \in \cH$, тогда ряд
    \[
        \sum_{n = 1}^{+\infty}{a_n}
    \]
    называется \textit{сходящимся}, если существует предел последовательности
    \[
        S_N = \sum_{n = 1}^N{a_n} 
    .\]
    Если ряд сходится, то соответствующий предел называется \textit{суммой ряда}.
\end{definition}

\begin{definition}
    $x, y \in \cH$ называются \textit{ортогональными}, если их скалярное
    произведение равно нулю:
    \[
        x \bot y \Llra \scp{x}{y} = 0
    .\]
\end{definition}

\begin{definition}
    Элемент $x \in \cH$ называется \textit{ортогональным множеству} $A \subset \cH$,
    если
    \[
        \forall y \in A~ x \bot y
    .\]
\end{definition}

\begin{definition}
    Ряд $\sum{a_n}$ называется \textit{ортогональным}, если
    \[
        \forall i, j \neq i~ a_i \bot a_j 
    .\]
\end{definition}

\begin{theorem}(Свойства сходимости в гильбертовом пространстве) 

    Пусть $x_i, y_i \in \cH$. Тогда
    \begin{itemize}
        \item $x_n \to x_0, y_n \to y_0 \Lra \scp{x_n}{y_n} \to \scp{x_0}{y_0}$.
        \item $\sum{x_k}$ сходится, тогда
            \[
                \forall y \in \cH~ \scp{\sum{x_k}}{y} = \sum{\scp{x_k}{y}}
            .\]
        \item $\sum{x_k}$ -- ортогональный ряд. Тогда сходимость этого ряда
            эквивалентна условию
            \[
                \sum{\norm{x_k}^2} < +\infty
            ,\]
            и при этом
            \[
                \norm{\sum{x_k}}^2 = \sum{\norm{x_k}^2}
            .\]
    \end{itemize}
\end{theorem}
\begin{proof}
    \enewline
    \begin{itemize}
        \item 
            \begin{align*}
                |\scp{x_n}{y_n} - \scp{x_0}{y_0}| 
                &\leqslant |\scp{x_n}{y_n} - \scp{x_n}{y_0}| + 
                |\scp{x_n}{y_0} - \scp{x_0}{y_0}| \\
                &= |\scp{x_n}{y_n - y_0}| + |\scp{x_n - x_0}{y_0}| \\
                &\leqslant \norm{x_n} \norm{y_n - y_0} + \norm{x_n - x_0} \norm{y_0}
                \xrightarrow[n \to +\infty]{} 0
            .\end{align*}
        \item Обозначим $S_N = \sum_{n = 1}^N{x_n}$, $S = \sum_{n = 1}^{+\infty}{x_n}$.
            Тогда
            \[
                \xymatrix{
                    \scp{S_N}{y} \ar[d]_{n \to +\infty} & = & \sum_{k = 1}^N
                                                        {\scp{x_k}{y}} \ar[d]_{n \to +\infty} \\
                    \scp{S}{y} & \overset{!}{=} & \sum_{k = 1}^{+\infty}{\scp{x_k}{y}}
                }
            .\]
        \item Обозначим $S_N = \sum_{k = 1}^N{x_k}$. Тогда
            \begin{align*}
                \norm{S_N}^2 = \scp{S_N}{S_N} = \scp{\sum_{k = 1}^N{x_k}}{\sum_{k = 1}^N{x_k}} =
                \sum_{k, l = 1}^N{\scp{x_k}{x_l}} = \sum_{k = 1}^N{\scp{x_k}{x_k}} = 
                \sum_{k = 1}^N{\norm{x_k}^2} = \Sigma_N
            .\end{align*}
            Аналогично:
            \[
                \norm{S_N - S_M}^2 = |\Sigma_N - \Sigma_M|
            .\]
            Таким образом, последовательности $S_n$, $\Sigma_n$ фундаментальны
            (или не фундаментальны) одновременно. Имея в виду полноту как $\cH$,
            так и $\R$, получаем, что эти последовательности сходятся (или
            расходятся) одновременно.
    \end{itemize}
\end{proof}

\begin{definition}
    Система векторов $\{\,e_k\,\}$ называется \textit{ортогональной},
    если
    \[
        \forall i, j \neq i~ e_i \bot e_j
    .\]
\end{definition}

\begin{definition}
    Система векторов $\{\,e_k\,\}$ называется \textit{ортонормированной},
    если она ортогональна, причем $\forall k~ \norm{e_k} = 1$.
\end{definition}

\begin{theorem}(О свойствах разложения по ортогональной системе)
    
    Пусть $E = \{\,e_k\,\}$ -- ортогональная система в $\cH$, $x \in \cH$,
    $x = \sum_{k = 1}^{+\infty}{c_k e_k}$. Тогда
    \begin{itemize}
        \item $E$ -- линейно независимая система.
        \item \label{th:Hortho:2} $c_k = \frac{\scp{x}{e_k}}{\norm{e_k}^2}$.
        \item $c_k e_k$ есть ортогональная проекция $x$ на прямую
            $\{\,t e_k\,\}_{t \in \R}$, то есть
            \[
                x = c_k e_k + z,~~ z \bot e_k
            .\]
    \end{itemize}
\end{theorem}
\begin{proof}
    \enewline
    \begin{itemize}
        \item Пусть
            \[
                \sum_{k = 1}^N{\a_k e_k} = 0
            .\]
            Домножим это равенство на $e_j$:
            \[
                0 = \sum_{k = 1}^N{\a_k \scp{e_k}{e_j}} = \a_j \norm{e_j} \Lra
                \a_j = 0
            .\]
        \item
            \begin{align*}
                \scp{x}{e_m} = \scp{\sum{c_k e_k}}{e_m} = \sum{c_k \scp{e_k}{e_m}} =
                c_k \scp{e_m}{e_m} \Lra c_k = \frac{\scp{x}{e_m}}{\norm{e_m}^2}
            .\end{align*}
        \item Надо проверить, что $z = x - c_k e_k \bot e_k$.
            \[
                \scp{x - c_k e_k}{e_k} = \scp{x}{e_k} - c_k \scp{e_k}{e_k} =
                \scp{x}{e_k} - c_k \norm{e_k}^2 = 0
            .\]
    \end{itemize}
\end{proof}

