\chapter{Интегралы, зависящие от параметра}

\section{Несобственный интеграл}

\begin{definition}
    Пусть $f \colon [a, b) \to \Rbar$ \textit{локально суммируема} на $[a, b)$, то
    есть $\forall a \leqslant B < b~ f$ суммируема на $[a, B]$. Тогда \textit{несобственным
    интегралом} $f$ на $[a, b)$ называется
    \[
        \intl_a^{\to b}{f \dd \lambda_1} \defeq \lim_{B \to b_-}{\intl_a^B{f \dd \lambda_1}}
    .\]
\end{definition}

\begin{theorem}
    $\displaystyle \intl_a^{\to b}{f \dd \lambda_1}$ сходится абсолютно тогда и только тогда,
    когда $f$ суммируема на $[a, b)$.
\end{theorem}
\begin{proof}
    \enewline
    \begin{itemize}
        \item[$\Lla$] Интеграл неотрицательной функции монотонен по множеству, поэтому:
            \[
                \intl_a^B{|f| \dd \lambda_1} \leqslant \intl_{[a, b)}{|f| \dd \lambda_1} < +\infty
            .\]
            По этой же причине $\displaystyle \intl_a^B{|f| \dd \lambda_1}$ монотонен по $B$.
            Поэтому существует предел
            \[
                \lim_{B \to b_-}{\intl_a^B{|f| \dd \lambda}} \leqslant  
                \intl_{[a, b)}{|f| \dd \lambda_1} < +\infty
            .\]
        \item[$\Lra$] По непрерывности меры $\displaystyle E \mapsto \intl_E{|f| \dd \lambda_1}$ снизу
            имеем
            \[
                +\infty > \lim_{B \to b_-}{\intl_a^B{|f| \dd \lambda_1}} = \intl_{[a, b)}{|f| \dd \lambda_1}
            .\]
    \end{itemize}
\end{proof}

\section{Дейстия над интегралами с параметром}

\textit{Далее общий контекст такой: $f \colon X \times Y \to \Rbar$,
    $X$ -- пространство с мерой, $Y$ -- метризуемое пространство, $\forall y_0~ f(x, y_0)$ суммируема на $X$.}

\begin{theorem}(О предельном переходе при равномерной сходимости)

    Пусть $\mu{X} < +\infty$, $y_0$ -- предельная точка $Y$, $\exists \f \colon X \to \R\colon~
    f(x, y) \rcon \f(x)$ при $y \to y_0$. Тогда $\f$ суммируема на $X$ и более того
    \[
        \lim_{y \to y_0}{\intl_X{f(x, y) \dd \mu}} = \intl_X{\f(x) \dd \mu}
    .\]
\end{theorem}
\begin{proof}
    \enewline
    \begin{itemize}
        \item Из равномерной сходимости следует, что 
            \[
                \exists U(y_0)\colon~ \forall x~ \forall y \in U~ |f(x, y) - \f(x)| < 1
            .\]
            Тогда $\f$ суммируема:
            \[
                \intl_X{|\f| \dd \mu} \leqslant \intl_X{(|f| + 1) \dd \mu} 
                \leqslant \intl_X{|f| \dd \mu} + \mu{X} < +\infty
            .\]
        \item Проверим, что интеграл сходится туда, куда мы ожидаем:
            \[
                \abs*{\intl_X{f \dd \mu} - \intl_X{\f \dd \mu}} \leqslant \intl_X{\abs*{f - \f} \dd \mu}
                \leqslant \sup_X{\abs{f - \f}} \cdot \mu{X} \xrightarrow[y \to y_0]{} 0 
            .\]
    \end{itemize}
\end{proof}

\begin{definition}
    $f \colon X \times Y \to \Rbar$, $y_0$ -- предельная точка $Y$. Тогда $f$ 
    \textit{удовлетворяет условию $L$-локальности в точке $y_0$} ($f \in L_{loc}(y_0)$), если 
    $\exists g \colon X \to \Rbar$ -- суммируемая на $X$, $\exists U(y_0)\colon$
    для почти всех $x \in X$, $\forall y \in U(y_0) \cap Y~ \abs*{f(x, y)} \leqslant g(x)$.
\end{definition}

\begin{theorem}(О предельном переходе при условии $L_{loc}$)

    Пусть $f \colon X \times Y \to \Rbar$, $\f \colon X \to \Rbar$, $\lim_{y \to y_0}{f(x, y)} = \f(x)$
    при почти всех $x$, $f \in L_{loc}(y_0)$. Тогда $\f$ суммируема на $X$ и более того
    \[
        \lim_{y \to y_0}{\intl_X{f(x, y) \dd \mu}} = \intl_X{\f \dd \mu}
    .\]
\end{theorem}
\begin{proof}
    \enewline
    \begin{itemize}
        \item Будем доказывать по Гейне. Пусть $g$ -- функция из условия $L_{loc}$ для $f$.
            Тогда вдоль любой последовательности $y_n \to y_0$:
            \[
                |f(x, y_n)| \leqslant g(x)
            .\]
            Что при предельном переходе по $n \to \infty$ влечет
            \[
                |\f(x)| \leqslant g(x) 
            .\]
            Отсюда получаем, что $\f$ суммируема на $X$.
        \item Положим $f_n = f(x, y_n)$. Имеем $f_n \to \f$, $|f_n| \leqslant g$, $g$
            суммируема. Применяя теорему Лебега о мажорированной сходимости в случае
            сходимости почти везде, получаем:
            \[
                \lim_{y \to y_0}{\intl_X{f \dd \mu}} = \lim_{n \to +\infty}{\intl_X{f_n \dd \mu}} 
                = \intl_X{\f \dd \mu}
            .\]
    \end{itemize}
\end{proof}

\begin{corollary}(Теорема Лебега о непрерывности интеграла по параметру)

    В условиях предыдущей теоремы, если $f$ непрерывно по $y$ в точке $y_0$, то функция
    \[
        \displaystyle J(y) = \intl_X{f(x, y) \dd \mu(x)} 
    .\] 
    непрерывна в $y_0$.
\end{corollary}

\begin{example}
    $\displaystyle \Gamma(y) = \intl_0^{+\infty}{x^{y - 1} e^{-x} \dd x}$ непрерывна по $y$ на $(0, +\infty)$.
\end{example}
\begin{proof}
    Проверим условие $\Gamma \in L_{loc}(y_0)$. Пусть $0 < \a < y_0 < \b < +\infty$. Тогда
    \[
        \forall x \in (0, +\infty)~ \forall y \in (\a, \b)~ \abs*{x^{y - 1}e^{-x}} 
        \leqslant g(x) = \begin{cases}
            x^{\b - 1} e^{-x},~ x > 1 \\
            x^{\a - 1} e^{-x},~ x \leqslant  1
        \end{cases}
    .\]
    $g(x)$, очевидно, суммируема.
\end{proof}

\begin{theorem}(Правило Лейбница дифференцирования интеграла по параметру)

    Пусть $Y \subset \R$ -- промежуток, $f \colon X \times Y \to \Rbar$, $\forall y~ f(x, y)$
    суммируема на $X$, $y_0 \in Y$, $J(y) = \intl_X{f(x, y) \dd \mu(x)}$. Кроме того:
    \begin{itemize}
        \item Для почти всех $x$, $\forall y \in Y~ \exists f'_y(x, y)$.
        \item $f'_y(x, y) \in L_{loc}(y_0)$.
    \end{itemize}
    Тогда $J(y)$ дифференцируема в $y_0$, причем
    \[
        J'(y) = \intl_X{f'_y(x, y) \dd \mu(x)}
    .\]
\end{theorem}
\begin{proof}
    Положим
    \[
        F(x, h) = \frac{f(x, y_0 + h) - f(x, y_0)}{h} \to f'_y(x, y_0)
    .\]
    Тогда выражение для производной $J(y)$ можно записать так:
    \[
        \frac{J(y_0 + h) - J(y_0)}{h} = \intl_X{F(x, h) \dd \mu(x)}
    .\]
    Проверим, что $F(x, h) \in L_{loc}(y_0)$:
    \[
        \abs*{F(x, h)} = \abs*{f'_y(x, y + \theta h) \cdot h \cdot h^{-1}} = \abs*{f'_y(x, y + \theta h)}
    .\]
    Последнее выражение мажорируется суммируемой функцией по условию $f'_y \in L_{loc}$.
    Воспользуеся теперь предыдущей теоремой:
    \[
        \frac{J(y_0 + h) - J(y_0)}{h} = \intl_X{F(x, h) \dd \mu(x)} \to \intl_X{f'_y(x, y_0) \dd \mu(X)}
    .\]
\end{proof}

\begin{example}
    $\Gamma(y) \in C^{\infty}$.
\end{example}
\begin{proof}
    Проверим, что $\Gamma(y)$ дифференцируема. Сначала формально найдем производную по $y$:
    \[
        \Gamma'(y_0) = \intl_0^{+\infty}{x^{y - 1} \ln{x} e^{-x} \dd x} 
    .\]
    Проверим условие $L_{loc}$: пусть $0 < \a < y_0 < \b < +\infty$. Тогда
    \[
        \abs*{x^{y - 1} \ln{x} e^{-x}} \leqslant g(x) = \begin{cases}
            x^{\b - 1} \ln{x} e^{-x},~ x \geqslant 1 \\
            x^{\a - 1} \abs*{\ln{x}} e^{-x},~ x < 1
        \end{cases}
    .\]
    Понятно, что $g(x)$ суммируема. Более того, она была бы суммируема, даже если бы
    логарифм был не в первой, а в любой степени, не меньшей единицы. Этот факт
    сразу даёт аналогичный способ доказательства того факта, что $\Gamma \in C^k$ для
    любого $k$, откуда получается, что $\Gamma \in C^\infty$.
\end{proof}

\begin{example}(Интеграл Дирихле)

    Пусть $\displaystyle J(y) = \intl_0^{+\infty}{e^{-xy} \frac{\sin{x}}{x} \dd x}$. Тогда
    $\displaystyle J(y) = \frac{\pi}{2} - \arctan{y}$.
\end{example}
\begin{proof}
    Продифференцируем $J(y)$:
    \[
        J'(y) = \intl_0^{+\infty}{-e^{-xy} \sin{x} \dd x} = -\frac{1}{1 + y^2}
    .\]
    Обоснуем это действие. Для этого проверим, $L_{loc}(y_0)$ для $0 < \a < y_0 < \b < +\infty$:
    \[
        \abs*{e^{-xy} \sin{x}} \leqslant e^{-\a x}
    .\]
    Мы получили, что $J(y) = C - \arctan{y}$. Чтобы найти константу, найдем предел
    \[
        \lim_{y \to \infty}{J(y)} = \lim_{y \to \infty}{\intl_0^{+\infty}{e^{-xy} \frac{\sin{x}}{x} \dd x}} =
        \intl_0^{+\infty}{\lim_{y \to \infty}{\parens*{e^{-xy} \frac{\sin{x}}{x}}} \dd x} =
        \intl_0^{+\infty}{0 \dd x} = 0
    .\]
    Чтобы обосновать предельный переход под знаком интеграла, проверим $L_{loc}$:
    \[
        \abs*{e^{-xy} \frac{\sin{x}}{x}} \leqslant e^{-\a x}
    .\]
    Таким образом, имеем равенство:
    \[
        C - \frac{\pi}{2} = 0 \Lra C = \frac{\pi}{2}
    .\]
\end{proof}

\section{Действия над несобственными интегралами с параметром}

\textit{Здесь контекст такой: $\displaystyle J(y) = \intl_a^{\to b}{f(x, y) \dd \mu(x)}$,
$f \colon \langle a, b \rangle \times Y \to \Rbar$, $f$ локально суммируема.}

\begin{definition}
    Интеграл $J(y)$ \textit{равномерно сходится на $Y$}, если
    \[
        \intl_a^t{f(x, y) \dd x} \rcon J(y),~ t \to b_-
    .\]
    Или, что то же самое:
    \[
        \sup_{y \in Y}{\abs*{\intl_a^t{f(x, y) \dd x} - J(y)}} =
        \sup_{y \in Y}{\abs*{\intl_t^{\to b}{f(x, y) \dd x}}} \xrightarrow[t \to b_-]{} 0
    .\]
\end{definition}

\begin{theorem}(Признак Вейерштрасса)

    Пусть $\forall x, y \in \langle a, b \rangle \times Y~ \abs*{f(x, y)} \leqslant g(x)$, причем
    $g(x)$ суммируема на $X$. Тогда $J(y)$ сходится равномерно.
\end{theorem}
\begin{proof}
    \[
        \intl_a^{\to b}{f(x, y) \dd x} \leqslant \intl_a^{\to b}{|g(x)| \dd x} =
        \intl_a^b{|g(x)| \dd x} < +\infty
    .\]
\end{proof}

\textit{Обобщим определение равномерной сходимости.}

\begin{definition}
    Пусть $f \colon X \times Y \to \R$, $\f \colon X \to \R$, $E \subseteq X$, $y_0$ -- предельная
    точка $y$. $X$, $Y$ -- хаусдорфовы топологические пространства. Тогда \textit{$f$ равномерно сходится к $\f$ 
    при $y \to y_0$ на $E$}, если
    \[
        \sup_{x \in E}{\abs*{f(x, y) - \f(x)}} \xrightarrow[y \to y_0]{} 0
    .\]
\end{definition}

\begin{remark}
    В определении изменилось только то, что раньше было $Y = \bN$, и $y_0 = +\infty$.
    Заметим, что ни в каких теоремах, связанных с равномерной сходимостью, 
    мы на самом деле не пользовались конкретно $Y = \bN$. Поэтому все соответстующие
    теоремы остаются справедливыми и в новой формулировке.
\end{remark}

\begin{theorem}(О перестановке предельных переходов)

    Пусть $f \colon T \times Y \to \R$, $T \subseteq \widetilde{T}$, $Y \subseteq \widetilde{Y}$ --
    метризуемые пространства, $t_0$ -- предельная точка $T$, $y_0$ -- предельная точка $Y$. Кроме того:
    \begin{itemize}
        \item $\displaystyle \forall t \in T~ \exists +\infty > L(t) = \lim_{y \to y_0}{f(t, y)}$.
        \item $\displaystyle \forall y \in Y~ \exists +\infty > J(y) = \lim_{t \to t_0}{f(t, y)}$.
        \item Хотя бы один из этих пределов равномерный.
    \end{itemize}
    Тогда существуют и конечны пределы
    \[
        \lim_{t \to t_0}{L(t)} = \lim_{y \to y_0}{J(y)}
    .\]
\end{theorem}

\begin{theorem}(О предельном переходе в несобственном интеграле)
    
    Пусть $f \colon X \times Y \to \Rbar$, $Y \subseteq \widetilde{Y}$, $y_0 \in \widetilde{Y}$ --
    предельная точка $Y$. Кроме того:
    \begin{enumerate}
        \item При почти всех $x$ $\displaystyle \exists f_0(x) = \lim_{y \to y_0}{f(x, y)}$.
        \item $f$ локально суммируема и выполняется: 
            \[
                \forall t < b~ \intl_a^t{f(x, y) \dd x} = \intl_a^t{f_0(x) \dd x}
            .\]
        \item $\displaystyle \forall y~ \exists J(y) = \intl_a^{\to b}{f(x, y) \dd x}$, причем
            сходится равномерно на $Y$.
    \end{enumerate}
    Тогда
    \[
        \intl_a^{\to b}{f(x, y) \dd x} \xrightarrow[y \to y_0]{} \intl_a^{\to b}{f_0(x) \dd x}
    .\]
\end{theorem}
\begin{proof}
    Это в точности предыдущая теорема с точностью до замен:
    \begin{itemize}
        \item $T = (a, b)$, $\widetilde{T} = \Rbar$, $t_0 = b$, $Y = Y$, $y_0 = y_0$.
        \item $\displaystyle f(t, y) = \intl_a^t{f(x, y) \dd x}$.
        \item $L(t) = \intl_a^t{f_0(x) \dd x}$.
        \item $J(y) = J(y)$, равномерен именно этот предел.
    \end{itemize}
\end{proof}

\begin{corollary}
    Если в последней теорема заменить первое условие на условие $y \mapsto f(x, y) \in C(y_0)$, 
    то получится теорема о непрерывности несобственного интеграла по параметру. 
\end{corollary}

\begin{theorem}(Об интегрировании несобственного интеграла по параметру)
    Пусть
    \begin{enumerate}
        \item $f \colon (a, b) \times Y \to \Rbar$ -- суммируемая на множествах вида $(a, t) \times Y$
            при $t < b$ функция.
        \item $\mu{Y} < +\infty$.
        \item $\displaystyle J(y) = \intl_a^{\to b}{f(x, y) \dd x}$ сходится равномерно.
    \end{enumerate}
    Тогда
    \begin{enumerate}
        \item $J(y)$ суммируема на $Y$. 
        \item Сходится интеграл
            \[
                \intl_a^{\to b}{\intl_Y{f(x, y) \dd y} \dd x} 
            .\]
        \item Выполняется
            \[
                \intl_a^{\to b}{\intl_Y{f(x, y) \dd y} \dd x} 
                = \intl_Y{\intl_a^{\to b}{f(x, y) \dd x} \dd y}
            .\]
    \end{enumerate}
\end{theorem}
\begin{proof}
    \enewline
    \begin{enumerate}
        \item Положим 
            \[ 
                J_t(y) = \intl_a^t{f(x, y) \dd x}
            .\]
            По теореме Фубини, $J_t$ суммируема на $Y$. Из равномерной сходимости $J(y)$ следует,
            что при $t$ достаточно близких к $b$ выполняется:
            \[
                \abs*{J(y) - J_t(y)} = \abs*{\intl_t^{\to b}{f(x, y) \dd x}} \leqslant 1
            .\]
            Отсюда сразу следует суммируемость $J(y)$ на $Y$:
            \[
                \intl_Y{|J(y)| \dd y} \leqslant \intl_Y{|J_t(y)|} + \mu{Y} < +\infty
            .\]
        \item Функция $\displaystyle x \mapsto \intl_Y{f(x, y) \dd y}$ суммируема на $(a, b)$ по
            теореме Фубини. По той же теореме Фубини, справедливо:
            \[
                \intl_a^t{\intl_Y{f(x, y) \dd y} \dd x} = \intl_Y{\intl_a^t{f(x, y) \dd x} \dd y} 
                = \intl_Y{\intl_a^{\to b}{f(x, y) \dd x} \dd y} 
                - \intl_Y{\intl_t^{\to b}{f(x, y) \dd x} \dd y}
            .\]
            Последнее выражение имеет смысл (а вместе с этим, первый интеграл существует) потому, что:
            \begin{enumerate}
                \item Интеграл
                    \[
                        \intl_a^{\to b}{f(x, y) \dd x} = J(y)
                    .\]
                    Является суммируемой на $Y$ функцией.
                \item Интеграл
                    \[
                        \intl_t^{\to b}{f(x, y) \dd x} = J(y) - \intl_a^t{f(x, y) \dd x}
                    .\]
                    Является суммируемой на $Y$ функцией.
            \end{enumerate}
        \item Проверим последние 2 пункта теоремы:
            \[
                \abs*{\intl_a^t{\intl_Y{f}} - \intl_Y{\intl_a^{\to b}{f}}} \leqslant
                \abs*{\intl_Y{\intl_t^{\to b}{f}}} \leqslant 
                \sup_{y \in Y}{\abs*{\intl_t^{\to b}{f}}} \cdot \mu{Y} \xrightarrow[t \to b_-]{} 0
            .\]
    \end{enumerate}
\end{proof}

\begin{theorem}(Правило Лейбница для несобственного интеграла)

    Пусть 
    \begin{enumerate}
        \item $f \colon [a, b) \times \langle c, d \rangle \to \R$, $f \in C$.
        \item $\displaystyle \forall y~ J(y) = \intl_a^{\to b}{f(x, y) \dd x}$ сходится. 
        \item $\forall x \forall y~ \exists f'_y(x, y)$, $f'_y \in C$.
        \item $\displaystyle I(y) = \intl_a^{\to b}{f'_y(x, y) \dd x}$ сходится равномерно.
    \end{enumerate}
    Тогда
    \begin{enumerate}
        \item $J(y) \in C^1(\langle c, d \rangle)$.
        \item $J'(y) = I(y)$.
    \end{enumerate}
\end{theorem}
\begin{proof}
    \enewline
    \begin{itemize}
        \item По теореме о непрерывности несобственного интеграла $I(y)$ непрерывно зависит от $y$.
        \item Зафиксируем $s_0, s \in \langle c, d \rangle$. Тогда
            \begin{align*}
                \intl_{s_0}^s{I(y) \dd y} 
                 &= \intl_{s_0}^s{\intl_a^{\to b}{f'_y \dd x} \dd y} =
                 \intl_a^{\to b}{\intl_{s_0}^s{f'_y(x, y) \dd y} \dd x} \\
                 &= \intl_a^{\to b}{\parens{f(x, s) - f(x, s_0)} \dd x} = J(s) - J(0)
             .\end{align*}
             Первый интеграл существует потому, что $I$ непрерывна и $\mu{Y} < +\infty$. Множества
             вида $[a, t] \times [s_0, s]$ компактны. $f'_y$ непрерывна, поэтому суммируема на таких множествах.
             Поэтому работает предыдущая теорема.
             Мы получили, что $J(s)$ дифференцируема. Кроме того, по теореме Барроу, последнее 
             равенство означает, что $J'(s) = I(s)$.
        \end{itemize}
\end{proof}

