\chapter{Основные интегральные формулы}
\section{Формула Грина}

\begin{remark}
     В данном контексте рассматривается $D$: компактное, связное,
     односвязное множество в $\R^2$, ограниченное кусочно-гладкой кривой.
     При этом граница $\partial D$ направлена против часовой стрелки 
     (фигура всегда находится слева).
\end{remark}

\begin{theorem}(Формула Грина)

    Пусть $P$, $Q$ -- гладкие векторные поля в $U(D)$. Тогда
\[
    \iintl_{D}{\left( \pderv{Q}{x} - \pderv{P}{y} \right) \dd x \d y}
    = \intl_{\partial D}{P \dd x + Q \dd y}
.\] 
\end{theorem}
\begin{proof}
    Докажем, что 
    \[
        \iintl_D{-\pderv{P}{y} \dd x \d y} = \intl_{\partial D}{P \dd x}
    .\]
    Из этого очевидно следует утверждение теоремы.
    Будем рассматривать простой случай. Пусть $D$ -- криволинейный четырёхугольник относительно
    осей $x$ и $y$ (как на рисунке). Тогда
    \[
        \iintl_D{\pderv{P}{y} \dd x \d y} = -\intl_a^b{\dd x \intl_{f_1(x)}^{f_2(x)}
        {\pderv{P}{y} \dd y}} = \intl_a^b{P(x, f_1(x)) \dd x} - \intl_a^b{P(x, f_2(x))}
    .\]
    С другой стороны,
    \[
        \intl_{\partial D}{P \dd x} = \intl_{\gamma_1} + \intl_{\gamma_2} + \intl_{\gamma_3}
        + \intl_{\gamma_4} = \intl_{\gamma_1} + \intl_{\gamma_3}
    .\]
    Параметризуем путь $\gamma_1$: Пусть $\gamma_1(t) = (t, f_1(t))$, $\gamma_2(t) = (b - t, f_2(b - t))$.
    Тогда имеем:
    \[
        \intl_{\partial D}{P \dd x} = \intl_{\gamma_1} + \intl_{\gamma_3} =
        \intl_a^b{P(t, f_1(t)) \dd t} - \intl_a^b{P(t, f_2(t)) \dd t} =
        \iintl_{D}{\pderv{P}{y} \dd x \d y}
    .\]
\end{proof}

\begin{remark}
    Формула "аддитивна" по фигуре.
%:: NOTE all рисунок, формулы
\end{remark}

\section{Формула Стокса}

\begin{remark}
    В данном контексте рассматривается $\O$ -- двусторонняя поверхность с границей.
    $n_0$ -- её сторона. $\partial \O$ -- кусочно-гладкая кривая, согласованная
    по ориентации со стороной поверхности.
\end{remark}

\begin{theorem}(Формула Стокса)

    Пусть $\langle P, Q, R \rangle$ -- гладкое векторное поле в $U(\O)$. Тогда
    \begin{align*}
        \intl_{\partial \O}{P \dd x + Q \dd y + R \dd z} =
        \iintl_{\O}{(R'_y - Q'_z) \dd y \d z + (P'_z - R'_x) \dd z \d x + (Q'_x - P'_y) \dd x \d y}
    ..\end{align*}
\end{theorem}
\begin{proof}
    Будем считать, что $\Omega$ -- $C^2$- гладкое. Достаточно проверить, что
    \[
        \intl_{\partial \Omega}{P \dd x} = \iintl_{\Omega}{P'_z \dd z \d x - P'_y \dd x \d y}
    .\]
    Пуст $\f(u, v) = (x(u, v), y(u, v), z(u, v))$ -- параметризация $\Omega$, причем она продолжается до
    границы $\partial \cO = L$ таким образом, что $\f(L) = \partial \Omega$. Тогда
    \begin{align*}
        \intl_{\partial \Omega}{P \dd x} 
        &= \intl_L{P(x(u, v), y(u, v), z(u, v)) \cdot
        \parenth{\pderv{x}{u} \dd u + \pderv{x}{v} \dd v}} = 
        \intl_L{P x'_u \dd u + P x'_v \dd v} \\
        &\underset{\text{Грин}}{=} \iintl_\cO{\parenth{\pderv{}{u}(P x'_v) - \pderv{}{v}(P x'_u)} \dd u \d v} \\
        &= \iintl_\cO{\parenth{(P'_x x'_u + P'_y y'_u + P'_z z'_u)x'_v + P x''_{uv} 
        - (P'_x x'_v + P'_y y'_v + P'_z z'_v)x'_u - P x''_{uv}} \dd u \d v} \\
        &= \iintl_\cO{\parenth{P'_x (x'_u x'_v - x'_v x'_u) + P'_y(y'_u x'_v - y'_v x'_u) 
        + P'_z(z'_u x'_v - z'_v x'_u)} \dd u \d v} \\
        &= \iintl_\cO{\parenth{P'_y(y'_u x'_v - y'_v x'_u) + P'_z(z'_u x'_v - z'_v x'_u)} \dd u \d v}
    ..\end{align*}
    Первое равенство легко получается, если попробовать посчитать левую и правую часть
    через параметризацию $L$. С другой стороны:
    \[
        \iintl_\Omega{P'_z \dd z \d x - P'_y \dd x \d y} = \iintl_\Omega{\left\langle
        \begin{pmatrix} 0 \\ P'_z \\ -P'_y \end{pmatrix}, n_0\right\rangle \dd S}
    .\]
    %При отображении $\f = (x, y, z)$ согласованная относительно ориетнации пара векторов переходит
    %в согласованную при порядке координат $u, v$ (такой порядок мы выбрали в самом начале
    %как порядок, определяющий ориентацию в $\R^2$. Поэтому нормаль следует выбирать именно так:
    Подберем такую параметризацию $\f$, которая отвечает согласованию стороны при попытке рассматривать
    $n = \tau_u \times \tau_v$ (направление $L$ должно соответствовать правильной ориентации):
    \[
        n = \begin{pmatrix}
            x'_u \\ y'_u \\ z'_u
        \end{pmatrix} \times \begin{pmatrix}
            x'_v \\ y'_v \\ z'_v
        \end{pmatrix} = \tau_u \times \tau_v
    .\]
    Тогда
    \[
        \iintl_\Omega{P'_z \dd z \d x - P'_y \dd x \d y} = \iintl_\Omega{\scp{
        \begin{pmatrix}
            0 \\ P'_z \\ -P'_y
        \end{pmatrix}, \tau_u \times \tau_v} \dd u \d v} =
        \intl_{\partial \Omega}{P \dd x}
    .\]
\end{proof}

\begin{remark}
    Формула "аддитивна" по фигуре.
%:: NOTE all рисунок, формулы
\end{remark}

\section{Формула Гаусса-Остроградского}

\begin{remark}
    В данном контексте рассматриваются
\[
    V = \left\{\, (x, y, z) \mid (x, y) \in \O, f(x, y) \leqslant z \leqslant F(x, y) \,\right\}
.\] 
    Здесь $\O \subseteq \R^2$ -- замкнутое множество, $\partial \O$ -- кусочно-гладкая кривая
    в $\R^2$, $f, F \in C^1(\O)$. Рассматриваем внешнюю сторону фигуры.
\end{remark}

\begin{theorem}(Формула Гаусса-Остроградского)

    Пусть $R \colon U(V) \to \R$, $R \in C^1(U(V))$. Тогда
\[
    \iiintl_{V}{\pderv{R}{z} \dd x \d y \d z} = \iintl_{\partial V^+}{R \dd x \d y}
.\] 
\end{theorem}
\begin{proof}
    \begin{align*}
        &\iiintl_{V}{\pderv{R}{z} \dd x \d y \d z} = \iintl_{\Omega}{\dd x \d y~
        {\intl_{f(x, y)}^{F(x, y)}{\pderv{R}{z} \dd z}}} \\
        &= \iintl_{\Omega}{R(x, y, F(x, y)) \dd x \d y} - \iintl_{\Omega}{R(x, y, f(x, y)) \dd x \d y} \\
        &= \iintl_{\Gamma_F}{R \dd x \d y} + \iintl_{\Gamma_f}{R \dd x \d y} + \iintl_{C}{R \dd x \d y} \\
        &= \iintl_{\partial V^+}{R \dd x \d y}
    ..\end{align*}
\end{proof}
%:: NOTE рисунок

\begin{corollary}
    В условиях формулы Гаусса-Остроградского, верно
\[
    \iiintl_{V}{\left(\pderv{P}{x} + \pderv{Q}{y} + \pderv{R}{z}\right) \dd x \d y \d z} =
    \iintl_{\partial V^+}{P \dd y \d z + Q \dd z \d x + R \dd x \d y}
.\]
\end{corollary}

\begin{corollary}
    Пусть $l$ -- фиксированное направление в $\R^3$. Тогда
\[
    \iiintl_{V}{\pderv{f}{l} \dd x \d y \d z} = 
    \iintl_{\partial V^+}{f \cdot \langle l, n_0 \rangle \dd S}
.\] 
\end{corollary}
\begin{proof}
    \begin{align*}
        &\iiintl_V{\pderv{f}{l} \dd x \d y \d z} =
        \iiintl_V{\parenth{l_1 \pderv{f}{x} + l_2 \pderv{f}{y} + l_3 \pderv{f}{z}} \dd x \d y \d z} \\
        &= \iintl_{\partial V^+}{f l_1 \dd y \d z + f l_2 \dd z \d x + f l_3 \dd x \d y}
        = \iintl_{\partial V^+}{f \cdot \scp{l}{n_0} \dd S}
    .\end{align*}
\end{proof}

\section{Примеры дифференциальных операторов}

\begin{definition}
    Пусть $C^1 \ni A = \langle P, Q, R \rangle$ -- векторное поле в $\R^3$. Тогда
    \textit{дивергенцией} $A$ называется
\[
    \diver A \defeq \pderv{P}{x} + \pderv{Q}{y} + \pderv{R}{z}
.\] 
\end{definition}

\begin{remark}(Бескоординатное определение дивергенции)

    Дивергенцию поля в точке можно вычислять так:
\[
    \diver A(a) = \lim_{r \to 0}{\frac{1}{\l_3 B} \iiintl_{B(a, r)}{\diver A~ \dd x \d y \d z}} =
    \lim_{r \to 0}{\frac{1}{\l_3 B} \iintl_{S(a, r)}{\langle A, n_0 \rangle} \dd S}
.\] 
    Последнюю формулу можно интерпретировать как величину потока, проходящего через
    сферу с центром в данной точке достаточно малого радиуса. То есть, дивергенция
    характеризует точку как ``источник'' поля.
\end{remark}

\begin{definition}
    Пусть $C^1 \ni A = \langle P, Q, R \rangle$ -- векторное поле в $\R^3$. Тогда
    \textit{ротором} $A$ называется
\[
    \rotor A \defeq \langle R'_y - Q'_z, P'_z - R'_x, Q'_x - P'_y \rangle
.\] 
\end{definition}

\begin{remark}
    $V \colon \cO \to \R^3$, $\cO$ -- односвязная область, $\rotor V = 0$. Тогда $V$ потенциально.
\end{remark}

\begin{remark}
    $V \colon \cO \to \R^3$, $\cO$ -- односвязная область, $\rotor V = 0$. Тогда
    \begin{itemize}
        \item Если $\gamma$ -- петля, то
\[
    \intl_{\gamma}{P \dd x + Q \dd y + R \dd z} = 0
.\] 
        \item Если $\gamma$ -- путь, то интеграл
\[
    \intl_{\gamma}{P \dd x + Q \dd y + R \dd z}
.\] 
            зависит только от начальной и конечной точек пути.
    \end{itemize} 
\end{remark}

\begin{remark}
    Если $\cO$ не односвязна, но $\rotor V = 0$, то все равно интеграл по пути
    не зависит от самого пути.
\end{remark}
\begin{proof}
    Зафиксируем две петли $\gamma_1$, $\gamma_2$ (как на рисунке). Обозначим фигуру, границами которой являются
    $\gamma_1$, $\gamma_2$. Тогда по формуле Стокса:
    \[
        0 = \iintl_\Omega{\rotor{V}} = \intl_{\gamma_1}{V} - \intl_{\gamma_2}{V}
    .\]
    Минус появился потому, что $\gamma_2$ не согласован по ориентации с $\gamma_1$.
\end{proof}
%:: NOTE all рисунок 

\textit{Если в поле нет источников, то откуда может взяться поток через поверхность?}
\begin{remark}
    $\diver V = 0$, тогда для любой ``разумной'' фигуры $\O$ выполнено
\[
    \iintl_{\partial \O}{\langle V, n_0 \rangle \dd S} = 0
.\] 
\end{remark}

\begin{remark}
    Пусть $U_1$, $U_2$ -- поверхности с одной границей $\partial U$. Тогда
    \[
        \iintl_{U_1^+}{\scp{V}{n_0} \dd S} = \iintl_{U_2^+}{\scp{V}{n_0} \dd S}
    .\]
\end{remark}
\begin{proof}
    Пусть $V$ -- фигура, ограниченная поверхностями $U_1$, $U_2$. Тогда
    \[
        0 = \iintl_{\partial V}{\scp{V}{n_0} \dd S} = \iintl_{U_1^+}{\scp{V}{n_0} \dd S}
        + \iintl_{U_2^-}{\scp{V}{n_0} \dd S}
    .\]
\end{proof}
%:: NOTE all рисунок

\begin{definition}
    Поле $V$ называется \textit{соленоидальным} в $\O \subseteq \R^3$, 
    если у него существует векторный потенциал, то есть
    $\exists B$ -- векторное поле такое, что $\rotor B = V$ на $\O$.
\end{definition}

\begin{theorem}(Критерий соленоидальности поля)
    
    $A$ соленоидально в $\O \Llra$ $\diver A = 0$ на $\O$
\end{theorem}
\begin{proof}
    \enewline
    \begin{itemize}
        \item[$\Lra$] $\diver{A} = \diver{\rotor{B}} = 0$, последнее равенство проверяется непосредственно.
        \item[$\Lla$] Пусть $A = (A_1, A_2, A_3)$. Будем искать $B$ в виде $(P, Q, 0)$. Зафиксируем точку
            $(x_0, y_0, z_0) \in \O$ и параллелепипед, содержащий эту точку и лежащий целиком в $\O$.
            Тогда справедлива система уравнений:
            \[
                \begin{cases}
                    R'_y - Q'_z = A_1 \\
                    P'_z - R'_x = A_2 \\
                    Q'_x - P'_y = A_3 
                \end{cases} \Llra
                \begin{cases}
                    -Q'_z = A_1 \\
                    P'_z = A_2 \\
                    Q'_x - P'_y = A_3
                \end{cases}
            .\]
            Восстановим из второго уравнения $P$:
            \[
                P(x, y, z) = \intl_{z_0}^z{A_2(x, y, t) \dd t} + \psi(x, y)
            .\]
            Здесь мы пользуемся наличием параллелепипеда (по $z$) вокруг нашей точки. Выберем $\psi(x, y) = 0$. 
            Аналогично восстановим из первого уравнения $Q$:
            \[
                Q(x, y, z) = -\intl_{z_0}^z{A_1(x, y, t) \dd t} + \f(x, y)
            .\]
            Подставим результаты в третье уравнение. Будем считать, что $A \in C^1$, 
            что позволит нам воспользоваться правилом Лейбница дифференцирования по параметру:
            \[
                -\intl_{z_0}^z{\pderv{A_1}{x} \dd z} + \f'_x - \intl_{z_0}^z{\pderv{A_2}{y} \dd z} = A_3
            .\]
            Пользуемся условием:
            \[
                \f'_x + \intl_{z_0}^z{\pderv{A}{z} \dd z} = A_3
            .\]
            Отсюда получаем:
            \begin{align*}
                \f'_x(x, y) + A_3(x, y, z) - A_3(x, y, z_0) &= A_3(x, y, z) \Lra
                \f'_x(x, y) = A_3(x, y, z_0) \\
                &\Lra \f(x, y) = \intl_{x_0}^x{A_3(t, y, z_0) \dd t}
            .\end{align*}
    \end{itemize}
\end{proof}

