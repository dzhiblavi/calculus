\chapter{Основные интегральные формулы}
\section{Формула Грина}

\begin{remark}
     В данном контексте рассматривается $D$: компактное, связное,
     односвязное множество в $\R^2$, ограниченное кусочно-гладкой кривой.
     При этом граница $\partial D$ направлена против часовой стрелки 
     (фигура всегда находится слева).
\end{remark}

\begin{theorem}(Формула Грина)

    Пусть $P$, $Q$ -- гладкие векторные поля в $U(D)$. Тогда
\[
    \iintl_{D}{\left( \pderv{Q}{x} - \pderv{P}{y} \right) \dd x \d y}
    = \intl_{\partial D}{P \dd x + Q \dd y}
.\] 
\end{theorem}
%:: NOTE all proof

\begin{remark}
    Формула "аддитивна" по фигуре.
%:: NOTE all рисунок, формулы
\end{remark}

\section{Формула Стокса}

\begin{remark}
    В данном контексте рассматривается $\O$ -- двусторонняя поверхность с границей.
    $n_0$ -- её сторона. $\partial \O$ -- кусочно-гладкая кривая, согласованная
    по ориентации со стороной поверхности.
\end{remark}

\begin{theorem}(Формула Стокса)

    Пусть $\langle P, Q, R \rangle$ -- гладкое векторное поле в $U(\O)$. Тогда
    \begin{align*}
        \intl_{\partial \O}{P \dd x + Q \dd y + R \dd z} =
        \iintl_{\O}{(R'_y - Q'_z) \dd y \d z + (P'_z - R'_x) \dd z \d x + (Q'_x - P'_y) \dd x \d y}
    \end{align*}
\end{theorem}
%:: NOTE all proof

\begin{remark}
    Формула "аддитивна" по фигуре.
%:: NOTE all рисунок, формулы
\end{remark}

\section{Формула Гаусса-Остроградского}

\begin{remark}
    В данном контексте рассматриваются
\[
    V = \left\{\, (x, y, z) \mid (x, y) \in \O, f(x, y) \leqslant z \leqslant F(x, y) \,\right\}
.\] 
    Здесь $\O \subseteq \R^2$ -- замкнутое множество, $\partial \O$ -- кусочно-гладкая кривая
    в $\R^2$, $f, F \in C^1(\O)$. Рассматриваем внешнюю сторону фигуры.
\end{remark}

\begin{theorem}(Формула Гаусса-Остроградского)

    Пусть $R \colon U(V) \to \R$, $R \in C^1(U(V))$. Тогда
\[
    \iiintl_{V}{\pderv{R}{z} \dd x \d y \d z} = \iintl_{\partial V^+}{R \dd x \d y}
.\] 
\end{theorem}
%:: NOTE all proof

\begin{corollary}
    В условиях формулы Гаусса-Остроградского, верно
\[
    \iiintl_{V}{\left(\pderv{P}{x} + \pderv{Q}{y} + \pderv{R}{z}\right) \dd x \d y \d z} = 
    \iintl_{\partial V^+}{P \dd y \d z + Q \dd z \d x + R \dd x \d y}
.\] 
\end{corollary}

\begin{corollary}
    Пусть $l$ -- фиксированное направление в $\R^3$. Тогда
\[
    \iiintl_{V}{\pderv{f}{l} \dd x \d y \d z} = 
    \iintl_{\partial V^+}{f \cdot \langle l, n_0 \rangle \dd S}
.\] 
\end{corollary}
%:: NOTE all proof

\section{Примеры дифференциальных операторов}

\begin{definition}
    Пусть $C^1 \ni A = \langle P, Q, R \rangle$ -- векторное поле в $\R^3$. Тогда
    \textit{дивергенцией} $A$ называется
\[
    \diver A \defeq \pderv{P}{x} + \pderv{Q}{y} + \pderv{R}{z}
.\] 
\end{definition}

\begin{remark}
    Дивергенцию поля в точке можно вычислять так:
\[
    \diver A(a) = \lim_{r \to 0}{\frac{1}{\l_3 B} \iiintl_{B(a, r)}{\diver A~ \dd x \d y \d z}} =
    \lim_{r \to 0}{\frac{1}{\l_3 B} \iintl_{S(a, r)}{\langle A, n_0 \rangle} \dd S}
.\] 
    Последнюю формулу можно интерпретировать как величину потока, проходящего через
    сферу с центром в данной точке достаточно малого радиуса. То есть, дивергенция
    характеризует точку как ``источник'' поля.
\end{remark}

\begin{definition}
    Пусть $C^1 \ni A = \langle P, Q, R \rangle$ -- векторное поле в $\R^3$. Тогда 
    \textit{ротором} $A$ называется
\[
    \rotor A \defeq \langle R'_y - Q'_z, P'_z - R'_x, Q'_x - P'_y \rangle
.\] 
\end{definition}

\begin{remark}
    $V \colon \cO \to \R^3$, $\cO$ -- односвязная область, $\rotor V = 0$. Тогда $V$ потенциально.
\end{remark}

\begin{remark}
    $V \colon \cO \to \R^3$, $\cO$ -- односвязная область, $\rotor V = 0$. Тогда
    \begin{itemize}
        \item Если $\gamma$ -- петля, то
\[
    \intl_{\gamma}{P \dd x + Q \dd y + R \dd z} = 0
.\] 
        \item Если $\gamma$ -- путь, то интеграл
\[
    \intl_{\gamma}{P \dd x + Q \dd y + R \dd z}
.\] 
            зависит только от начальной и конечной точек пути.
    \end{itemize} 
\end{remark}

\begin{remark}
    Если $\cO$ не односвязна, но $\rotor V = 0$, то все равно интеграл по пути
    не зависит от самого пути.
%:: NOTE all рисунок, proof
\end{remark}

\textit{Если в поле нет источников, то откуда может взяться поток через поверхность?}
\begin{remark}
    $\diver V = 0$, тогда для любой ``разумной'' фигуры $\O$ выполнено
\[
    \iintl_{\partial \O}{\langle V, n_0 \rangle \dd S} = 0
.\] 
%:: NOTE all рисунок, proof
\end{remark}

\begin{definition}
    Поле $V$ называется \textit{соленоидальным} в $\O \subseteq \R^3$, 
    если у него существует векторный потенциал, то есть
    $\exists B$ -- векторное поле такое, что $\rotor B = V$ на $\O$.
\end{definition}

\begin{theorem}(Критерий соленоидальности поля)
    
    $A$ соленоидально в $\O \Llra$ $\diver A = 0$ на $\O$
\end{theorem}
%:: NOTE all proof
