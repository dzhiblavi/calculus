\section{Предельный переход под знаком интеграла}

\begin{theorem}(Леви)

    Пусть $f_n \colon X \to \Rbar$, измеримы, $\forall n~ 0 \leqslant f_n \leqslant f_{n + 1}$
    при почти всех $x \in X$. Пусть $\displaystyle f(x) = \lim_{n \to +\infty}{f_n(x)}$ 
    при почти всех $x$. Тогда
\[
    \int_X{f ~\d\mu} = \lim_{n \to +\infty}{\int_X{f_n ~\d\mu}}
.\] 
\end{theorem}

\begin{theorem}

    Пусть $f, g \geqslant 0$, измеримы на $E$. Тогда
\[
    \int_E{(f + g) ~\d\mu} = \int_E{f ~\d\mu} + \int_E{g ~\d\mu}
.\] 
\end{theorem}

\begin{corollary}
    
    Пусть $f, g$ суммируемы на $E$. Тогда $f + g$ суммируема, причем
\[
    \int_E{(f + g) ~\d\mu} = \int_E{f ~\d\mu} + \int_E{g ~\d\mu}
.\]
\end{corollary}

\begin{definition}
    $\cL(X) = \left\{\, f \mid f \colon X \to \Rbar, \int{|f| ~\d\mu} < +\infty \,\right\}$
\end{definition}

\begin{lemma}
    $\cL(X)$ -- линейное пространство. 
\end{lemma}

\begin{theorem}
    Пусть $u_n \colon X \to \R$, $u_n \geqslant 0$ почти везде, $u_n$ измеримы на $E$.
    Тогда
\[
    \int_E{\left(\sum_{n = 1}^{+\infty}{u_n}\right) ~\d\mu} = \sum_{n = 1}^{+\infty}{\int_E{u_n ~\d\mu}}
.\]
\end{theorem}

\begin{corollary}
    Пусть $u_n$ измеримы, причем $\displaystyle \sum_{n = 1}^{+\infty}{\int_E{|u_n| ~\d\mu}} < +\infty$,
    тогда ряд $\displaystyle \sum_{n = 1}^{+\infty}{u_n ~\d\mu}$ сходится абсолютно почти везде на $E$.
\end{corollary}

\begin{theorem}(Абсолютная непрерывность интеграла)

    Пусть $f$ -- суммируемая функция. Тогда
\[
    \forall \e > 0~ \exists \delta > 0~ \forall E \in \cA\colon~ \mu(E) < \delta
    ~~ \left|\int_E{f ~\d\mu}\right| < \e
.\] 
\end{theorem}

\begin{corollary}
    Пусть $e_n \in \cA$, $\mu(e_n) \xrightarrow[n \to +\infty]{} 0$, $f$ -- суммируемая функция,
    тогда
\[
    \int_{e_n}{|f| ~\d\mu} \xrightarrow[n \to +\infty]{} 0
.\] 
\end{corollary}
