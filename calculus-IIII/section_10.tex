\section{Гильбертовы пространства}

\begin{definition}
    \textit{Гильбертовым пространством} называется линейное пространство $\cH$ со скалярным произведением,
    полное как метрическое пространство с метрикой и нормой, порожденными скалярным произведением:
    \begin{itemize}
        \item $\langle \phantom{x}, \phantom{x} \rangle \colon \cH \times \cH \to \R(\bC)$.
        \item $\norm{\phantom{x}} \colon \cH \to \R, \norm{\ex} = \sqrt{\langle \ex, \ex \rangle}$.
        \item $\langle \ex, \ex \rangle \geqslant 0$, $\langle \ex, \ex \rangle = 0 \Llra \ex = \elemvec{0}$.
        \item $\langle \ex, \ey \rangle = \overline{\langle \ey, \ex \rangle}$.
        \item $\langle \a \ex + \b \ey, \ez \rangle = \a \langle \ex, \ez \rangle + \b \langle \ey, \ez \rangle$.
    \end{itemize} 
\end{definition}

%:: NOTE all examples

\textit{Далее $\cH$ -- Гильбертово пространство.}

\begin{definition}
    Ряд $\sum_{i = 1}^{+\infty}{a_n}$, $a_n \in \cH$, называется \textit{сходящимся}, если
    $S_N \defeq \sum_{i = 1}^{N}{a_i}$ таково, что $\exists S \in \cH \colon~
    \norm{S_N - S} \to 0$. Иными словами, последовательность частичных сумм ряда сходится 
    к элементу $\cH$.
\end{definition}

\begin{definition}
    $\ex \bot \ey \Llra \scp{\ex}{\ey} = 0$.
\end{definition}

\begin{definition}
    Пусть $A \subseteq \cH$. Тогда по определению $\ex \bot A \Llra \forall \ey \in A~ \ey \bot \ex$.
\end{definition}

\begin{definition}
    Ряд называется \textit{ортогональным}, если все его элементы попарно ортогональны.
\end{definition}

\begin{theorem}(Свойства сходимости в Гильбертовых пространствах)

    Пусть $\ex_i, \ey_i \in \cH$. Тогда
    \begin{itemize}
        \item $\ex_n \to \ex_0, \ey_n \to \ey_0 \Lra \scp{\ex_n}{\ey_n} \to \scp{\ex_0}{\ey_0}$.
        \item Пусть ряд $\sum{\ex_k}$ сходится. Тогда $\forall \ey \in \cH~ \scp{\sum{\ex_k}}{\ey} 
            = \sum{\scp{\ex_k}{\ey}}$.
        \item Пусть ряд $\sum{\ex_k}$ ортогонален. Тогда $\sum{\ex_k}$ сходится тогда и только тогда,
            когда $\sum{\norm{\ex_k}^2}$ сходится. Более того, в этом случае
            $\norm{\sum{\ex_k}}^2 = \sum{\norm{\ex_k}^2}$.
    \end{itemize} 
\end{theorem}
%:: NOTE all proof

\begin{definition}
    \textit{Ортогональным семейством векторов} называется $\{\,\ee_k\,\} \subseteq \cH$
    такое, что $\ee_k \bot \ee_{j \neq k}$. Если более того $\norm{\ee_k} = 1$, то
    семейство называется \textit{ортонормированным}.
\end{definition}

\begin{definition}
    $L_2 \defeq L^2([0, 2\pi], \l_1)$.
\end{definition}

\begin{theorem}
    Пусть $\{\,\ee_k\,\}$ -- ОС, $\ex \in \cH$, $\ex = \sum_{k = 1}^{+\infty}{c_k \ee_k}$.
    Тогда
    \begin{itemize}
        \item ОС линейно независима.
        \item $\displaystyle c_k = \frac{\scp{\ex, \ee_k}}{\norm{\ee_k}^2}$.
        \item $\displaystyle c_k \ee_k = \cP^{\bot}_{\{t \ee_k\}}$, 
            то есть $\ex = c_k \ee_k + \ez$, $\ez \bot \ee_k$.
    \end{itemize} 
\end{theorem}
%:: NOTE all proof

\section{Ряды фурье}

\begin{definition}
    Пусть $\{\,\ee_k\,\}$ -- ОС, $\ex \in \cH$, тогда числа
    $c_k(\ex) = \frac{\scp{\ex}{\ee_k}}{\norm{\ee_k}^2}$ называются 
    \textit{коэффициентами фурье вектора $\ex$ по системе $\ee_k$}.
\end{definition}

\begin{definition}
    Ряд $\sum_k{c_k(\ex) \ee_k}$ называется \textit{рядом Фурье $\ex$ по $\ee_k$}. 
\end{definition}

\begin{remark}
    При перенормировке ОС ряд Фурье не меняется.
\end{remark}

\begin{theorem}(О свойстах частичных сумм ряда Фурье)

    Пусть $\cL = Lin(\ee_1, \ldots, \ee_n)$. Тогда
    \begin{itemize}
        \item $S_n = \cP^\bot_\cL(\ex)$, то есть $\ex = S_n + \ez$, $\ez \bot \cL$.
        \item $S_n$ -- элемент наилучшего приближения $\ex$ в $\cL$, то есть
            $\forall \ey \in \cL~ \norm{S_n - \ex} \leqslant \norm{\ey - \ex}$.
        \item $\norm{S_n} \leqslant \norm{\ex}$.
    \end{itemize} 
\end{theorem}
%:: NOTE all proof

\begin{corollary}(Неравенство Бесселя)
\[
    \sum_{k = 1}^{+\infty}{|c_k(\ex)|^2 \norm{\ee_l}^2} \leqslant \norm{x}^2
.\] 
\end{corollary}
%:: NOTE all proof

\begin{theorem}(Рисс, Фишер)

    Пусть $\{\,\ee_k\,\}$ -- ОС, $\ex \in \cH$. Тогда
    \begin{itemize}
        \item Ряд Фурье $\ex$ сходится в $\cH$.
        \item $\displaystyle \ex = \sum_{k = 1}^{+\infty}{c_k(\ex) \ee_k} + \ez$, $\forall k~ \ez \bot \ee_k$.
        \item $\displaystyle \ex = \sum_{k = 1}^{+\infty}{c_k(\ex) \ee_k} \Llra \norm{\ex}^2 = 
            \sum_{k = 1}^{+\infty}{|c_k(\ex)|^2 \norm{\ee_k}^2}$.
    \end{itemize} 
\end{theorem}
%:: NOTE all proof

\begin{definition}
    \textit{Равенство Парсиваля}, или \textit{уравнение замкнутости}:
\[
    \norm{x}^2 = \sum_{k = 1}^{+\infty}{|c_k(\ex)|^2 \norm{\ee_k}^2}
.\] 
\end{definition}

\section{Базис в Гильбертовом пространстве}

\begin{definition}
    \textit{Базисом} в Гильбертовом пространстве называется ОС $\{\,\ee_k\,\}$, если
    выполняется условие:
\[
    \forall \ex \in \cH~ \ex = \sum_{k = 1}^{+\infty}{c_k(\ex) \ee_k}
.\] 
\end{definition}

\begin{definition}
    ОС $\{\,\ee_k\,\}$ называется \textit{полной}, если
\[
    \forall \ex \in \cH\colon~ \{\,\ee_k\,\} \cup \ex \text{ -- не ОС}
.\] 
\end{definition}

\begin{definition}
    ОС $\{\,\ee_k\,\}$ называется \textit{замкнутой}, если для любого её
    элемента выполняется уравнение замкнутости.
\end{definition}

\begin{theorem}(Характеризация базиса)
    
    Пусть $E = \{\,\ee_k\,\}$ -- ОС. Тогда эквивалентны утверждения:
    \begin{itemize}
        \item $E$ -- базис.
        \item $\forall \ex, \ey \in \cH~ \scp{\ex}{\ey} = \sum_{k = 1}^{+\infty}{c_k(\ex) 
            \cdot \overline{c_k(\ey)} \cdot \norm{\ee_k}^2}$
        \item $E$ замкнута.
        \item $E$ полна.
        \item $Lin(\ee_1, \ee_2, \ldots)$ плотно в $\cH$.
    \end{itemize} 
\end{theorem}


