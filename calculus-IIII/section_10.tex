\chapter{Ряды Фурье}
\section{Пространство $L^p$}

\begin{definition}
    Комплексное отображение $f \colon X \to \bC$ назовем \textit{измеримым}, если
    $f(x) = g(x) + i h(x)$, $g, h \colon X \to \R$, причем $g$, $h$ измеримы.
\end{definition}

\begin{definition}
    Аналогично определим \textit{суммируемые} комплексные отображения.
\end{definition}
 
\begin{definition}
   Пусть $f \colon X \to \bC$, $f(x) = g(x) + i h(x)$, $g, h \colon X \to \R$. Тогда
   определим интеграл:
\[
    \intl_{E}{f \dd\mu} \defeq \intl_{E}{g \dd\mu} + i \intl_{E}{h \dd\mu}
.\] 
\end{definition}

\begin{remark}
\[
    \left|\intl_{E}{f \dd\mu}\right| \leqslant \intl_{E}{|f| \dd\mu}
.\] 
\end{remark}

\begin{theorem}(Интегральное неравенство Гёльдера)
    
    Пусть $p, q > 1$, $\frac{1}{p} + \frac{1}{q} = 1$, $f, g \colon X \to \bC$
    -- измеримые почти везде заданные функции.
    Тогда
\[
    \intl_{X}{|f g| \dd\mu} \leqslant \left(\intl_{X}{|f|^p \dd\mu}\right)^{\frac{1}{p}}
    \cdot \left(\intl_{X}{|g|^q \dd\mu}\right)^{\frac{1}{q}}
.\] 
\end{theorem}

\begin{theorem}(Интегральное неравенство Минковского)
    
    Пусть $f, g \colon X \to \bC$, $p \geqslant 1$, тогда
\[
    \left(\intl_{X}{|f + g|^p \dd\mu}\right)^{\frac{1}{p}} \leqslant
    \left(\intl_{X}{|f|^p \dd\mu}\right)^{\frac{1}{p}} 
    + \left(\intl_{X}{|g|^p \dd\mu}\right)^{\frac{1}{p}}
.\]  
\end{theorem}

\begin{definition}
    Пусть $\langle X, \cA, \mu \rangle$ -- пространство с мерой. Тогда для
    $1 \leqslant p < +\infty$ положим
\[
    \cL^p(X, \mu) \defeq \left\{f \colon X \to \bC / \R \mid 
    f \text{ измерима}, \intl_{X}{|f|^p \dd\mu} < +\infty\right\}
.\] 
\end{definition}

\begin{remark}
    $\cL^p(X, \mu)$ -- линейное пространство.
\end{remark}

\begin{definition}
    Зададим на $\cL^p$ отношение эквивалентности: $f \sim g$ тогда и
    только тогда, когда $f = g$ почти везде. Положим
\[
    L^p(X, \mu) \defeq \cL^p(X, \mu) / \sim
.\] 
\end{definition}

\begin{definition}
    В $L^p$ заведем норму: $\displaystyle \norm{[f]} \defeq 
    \left(\int_{X}{|f|^p \dd\mu}\right)^{\frac{1}{p}}$.
\end{definition}

\begin{definition}
    Пусть $f \colon X \to \Rbar$ задана почти везде. Тогда \textit{существенным супремумом
    $f$} называется
\[
    \esssup_{X}{f} \defeq \inf{\{A \in \Rbar \mid f(x) \leqslant A \text{ п.в.}\}}
.\] 
\end{definition}

\begin{theorem}(Свойства существенного супремума)
    \begin{itemize}
        \item $\esssup_{X}{f} \leqslant \sup_{X}{f}$.
        \item $f(x) \leqslant \esssup_{X}{f}$ при почти всех $x$.
        \item $\displaystyle \left|\int_{X}{fg \dd\mu}\right| 
               \leqslant \esssup_{X}{|f|} \cdot \int_{X}{|g|}$.
    \end{itemize} 
\end{theorem}
\begin{proof}
    \enewline
    \begin{itemize}
        \item Супремум есть в множестве, по которому берется инфимум 
            в определении существенного супремума.
        \item Пусть $M = \esssup_{X}{f}$. Тогда по определению инфимума:
            \[
                \forall n \in \bN~ f(x) \leqslant M + \frac{1}{n}
            \]
            при почти всех $x$. Объединяем все эти неравенства (их
            счетное число) и получаем требуемое.
        \item По предыдущему пункту $f$ может быть больше $\esssup_X{f}$ только
            на множестве $e$ меры $0$. Поэтому
            \[
                \abs*{\intl_X{f g \dd \mu}} \leqslant
                \intl_X{\abs*{f g} \dd \mu} \leqslant
                \underbrace{\intl_e{\abs*{f g} \dd \mu}}_{= 0} 
                + \intl_{X \setminus e}{\abs*{f g} \dd \mu} \leqslant
                \esssup_X{f} \cdot \intl_{X}{\abs*{g} \dd \mu}
            .\]
    \end{itemize}
\end{proof}

\begin{definition}
    Для $p = \infty$:
\[
    \cL^{\infty}(X, \mu) \defeq \left\{f \colon X \to \bC / \R \mid 
    f \text{ измерима}, \esssup_{X}{f} < +\infty\right\}
.\] 
\end{definition}

\begin{remark}
    $\cL^{\infty}(X, \mu)$ -- линейное пространство.
\end{remark}

\begin{definition}
    Пространство $L^{\infty}$ зададим аналогично конечному случаю.
    Нормой на этом пространстве положим $\esssup$.
\end{definition}

\begin{theorem}(О вложении пространств $L^p$)
    
    Пусть $\mu(X) < +\infty$, $1 \leqslant s < r \leqslant +\infty$.
    Тогда
    \begin{itemize}
        \item $L^r(X, \mu) \subset L^s(X, \mu)$.
        \item $\norm{f}_s \leqslant \mu(X)^{\frac{1}{s} - \frac{1}{r}}
               \cdot \norm{f}_r$.
    \end{itemize} 
\end{theorem}
\begin{proof}
    \enewline
    \begin{itemize}
        \item Первый пункт очевидным образом следует из второго.
        \item Пусть $r = +\infty$. Тогда
            \[
                \norm{f}_s = \parens*{\intl_X{\abs{f}^s}}^{\frac{1}{s}} \leqslant
                \parens*{\intl_X{\norm{f}^s_\infty}}^{\frac{1}{s}} = 
                \norm{f}_\infty \cdot {\mu(X)}^{\frac{1}{s}}
            .\]
            Неравенство выполнено потому, что функция почти везде не превосходит
            свой существенный супремум. Пусть теперь $r < +\infty$:
            \[
                \norm{f}^s_s = \intl_X{|f|^s \cdot 1 \dd \mu}
                \underset{\text{Гельдер}}{\leqslant} \parens*{\intl_X
                {\parens*{|f|^s}^{\frac{r}{s}}}}^{\frac{s}{r}} \cdot 
                \parens*{\intl_X{1}}^{\frac{r - s}{s}} =
                \parens*{\intl_X{|f|^r}}^{\frac{s}{r}} \cdot 
                {\mu(X)}^{1 - \frac{s}{r}} =
                \norm{f}_r^s \cdot {\mu(X)}^{1 - \frac{s}{r}}
            .\]
    \end{itemize}
\end{proof}

\begin{corollary}
    Пусть $\mu(E) < +\infty$, $1 \leqslant s < r \leqslant +\infty$, $f_n, f \in L^s$,
    $f_n \xrightarrow[L^r]{} f$, тогда $f_n \xrightarrow[L^s]{} f$.
\end{corollary}

\begin{theorem}(О сходимости в $L^p$ и по мере)
    
    Пусть $1 \leqslant r < +\infty$, $f_n, f \in L^p$, тогда
    \begin{itemize}
        \item $f_n \xrightarrow[L^p]{} f \Lra f_n \underset{\mu}{\Lra} f$.
        \item $f_n \underset{\mu}{\Lra} f$, либо $f_n \to f$ почти везде, тогда
            если $\exists g \in L^p\colon~ |f_n| \leqslant g$, то
            $f_n \xrightarrow[L^p]{} f$.
    \end{itemize} 
\end{theorem}
\begin{proof}
    \enewline
    \begin{itemize}
        \item Обозначим
            \[
                X_n(\e) = X(|f_n - f| \geqslant \e)
            .\]
            Тогда на $X_n(\e)$ выполнено:
            \[
                1 \leqslant \frac{|f_n - f|^p}{\e^p}
            .\]
            Проверим сходимость по мере:
            \[
                \mu{X_n(\e)} = \intl_{X_n(\e)}{1 \dd \mu} \leqslant
                \frac{1}{\e^p} \intl_{X_n(\e)}{|f_n - f|^p \dd \mu} =
                \frac{1}{\e^p} \norm{f_n - f}_p \xrightarrow[n \to +\infty]{} 0
            .\]
        \item Из сходимости по мере имеем сходимость почти везде вдоль
            подпоследовательности:
            \[
                \exists n_k \colon~ f_{n_k} \xrightarrow[n \to +\infty]{} f
            \]
            при почти всех $x$. Поскольку $f_{n_k} \to f$, $|f_{n_k}| \leqslant 
            g$, верно:
            \[
                |f| \leqslant g
            .\]
            Поэтому:
            \[
                |f_n - f|^p \leqslant (2 g)^p
            .\]
            Интеграл последней функции конечен, так как $g \in L^p$.
            В таком случае по теореме Лебега заключаем
            \[
                \norm{f_n - f}^p_p = \intl_X{|f_n - f|^p \dd \mu}
                \xrightarrow[n \to +\infty]{} 0
            .\]
    \end{itemize}
\end{proof}

\begin{remark}
    $L^{\infty}$ -- полное метрическое пространство.
\end{remark}

\begin{theorem}(Полнота пространств $L^p$)
    
    $\forall 1 \leqslant p \leqslant \infty$ $L^p$ полно.
\end{theorem}
\begin{proof}
    Докажем утверждение для конечных $p$. 
    \begin{itemize}
        \item Пусть $f_n$ -- фундаментальная в $L^p$ последовательность, то есть
            \[
                \forall \e > 0 \exists N_\e \colon \forall n, k > N_\e~
                \norm{f_n - f_k}_p < \e
            .\]
            Пользуясь этим фактом, построим последовательности индексов
            $k_i, n_i$ следующим образом:
            \[
                \e = \frac{1}{2^i}~ \exists N_i \colon k_i > n_i > N_i \Lra
                \norm{f_{n_i} - f_{k_i}}_p < \e
            .\]
            Выделим из $n_i$ возрастающую последовательность:
            \[
                n_k = \max_{1 \leqslant i \leqslant k}{n_i}
            .\]
            Тогда верно неравенство:
            \[
                \sum_{k = 1}^{+\infty}{\norm{f_{n_{k + 1}} - f_{n_k}}_p} \leqslant 1
            .\]
        \item Зафиксируем функции $f_i$ -- представители соответствующих
            классов эквивалентности в $\cL^p$. Положим
            \[
                S(x) = \sum_{k = 1}^{+\infty}{|f_{n_{k + 1}}(x) - f_{n_k}(x)|}
                \in [0, +\infty]
            .\]
            Оценим частичные суммы ряда:
            \[
                \norm{S_N}_p \underset{\triangle}{\leqslant} \sum_{i = 1}^N
                {\norm{f_{n_{k + 1}} - f_{n_k}}_p} < 1
            .\]
            Из этой оценки следует:
            \[
                \intl_X{S_N^p} \leqslant 1
            .\]
            Из чего по теореме Фату получаем:
            \[
                \intl_X{S^p} \leqslant 1
            .\]
            Поэтому $S^p$ суммируема, а значит, $S(x)$ конечна почти везде.
        \item Пусть
            \[
                f(x) = f_{n_1} + \sum_{k \geqslant 1}{(f_{n_{k + 1}} - f_{n_k})}
            .\]
            При этом
            \[
                \sum_{1 \leqslant k \leqslant m - 1}{(f_{n_{k + 1}} - f_{n_k})}
                = f_{n_m} 
            .\]
            Помня про сходимость соответствующего ряда, получаем
            \[
                f_{n_m} \to f
            \]
            при почти всех $x$.
        \item Проверим наконец сходимость последовательности. Из равномерной 
            сходимости имеем:
            \[
                \forall \e > 0 \exists N \colon \forall n, m = n_k > N~
                \norm{f_n - f_m}_p < \e
            .\]
            Поэтому справедливо:
            \[
                \e^p > \norm{f_n - f_{n_k}}^p_p = \intl_X{|f_{n_k} - f_n|^p \dd \mu}
            .\]
            Отсюда по теореме Фату получаем:
            \[
                \intl_X{|f_n - f|^p \dd \mu} < \e^p
            .\]
    \end{itemize}
\end{proof}

\begin{definition}
    Пусть $X$ -- топологическое пространство, тогда множество
    $A \subset X$ называется всюду плотным, если $\Cl(A) = X$.
    Иначе говоря, $\Int(X \setminus A) = \varnothing$, 
    или \\ $\forall x \in X~ \forall U(x)~ U(x) \cap A \neq \varnothing$.
\end{definition}

\begin{definition}
    Множество всех ступенчатых функций $g \colon X \to \Rbar$
    обозначим $St(X)$.
\end{definition}

\begin{lemma}
    Пусть $1 \leqslant p \leqslant +\infty$, тогда множество
    $St(X)$ плотно в $L^p$.
\end{lemma}
\begin{proof}
    \enewline
    \begin{itemize}
        \item Пусть $p = +\infty$. Подменим $f \in L^\infty$ на
            множестве меры ноль так, чтобы
            \[
                \esssup_X{f} = \sup_X{f}
            .\]
            Тогда по теореме об аппроксимации существуют последовательности 
            ступенчатых функций $\f_n, \psi_n$ такие, что
            \[
                \f_n \rcon f_+,~~ \psi_n \rcon f_-
            .\]
            Из равномерной сходимости получаем
            \[
                \norm{\f_n - f_+}_\infty \to 0,~~ \norm{\psi_n - f_-}_\infty \to 0
            .\]
        \item Пусть теперь $p$ конечно. Рассмотрим неотрицательную $f \in L^p$.
            По теореме об аппроксимации
            \[
                \exists \f_n \to f
            ,\]
            причем $\f_n$ возрастают с номером $n$ и неотрицательны. Проверим,
            что эта последовательность ступенчатых функций аппроксимирует
            $f$ в смысле $L^p$. Для этого заметим, что поскольку $\f_n \leqslant f$,
            верно
            \[
                |\f_n - f|^p \leqslant |f|^p
            .\]
            При этом, $|f|^p$ суммируема. Значит, по теореме Лебега о
            мажорированной сходимости справедливо:
            \[
                \norm{\f_n - f}^p_p = \intl_X{|\f_n - f| \dd \mu} \to 0
            .\]
    \end{itemize}
\end{proof}

\begin{definition}(Четвертая аксиома отделимости)
    
    Топологическое пространство называется \textit{нормальным},
    если в нем любые два замкнутые непересекающиеся множества
    отделимы, причем любое одноточечное множество замкнуто.
\end{definition}

\begin{lemma}(Урысон)
    
    Пусть $X$ -- нормальное топологическое пространство, 
    $F_0, F_1$ -- замкнутые непересекающиеся множества. Тогда
    существует непрерывная функция $f \colon X \to \R$, такая, что
    \begin{itemize}
        \item $0 \leqslant f \leqslant 1$.
        \item $f\big|_{F_0} = 0$.
        \item $f\big|_{F_1} = 1$.
    \end{itemize} 
\end{lemma}
\begin{proof}
    \enewline
    \begin{itemize}
        \item Переформулируем определение нормальности с точностью до дополнения:
            для любого замкнутого $F$ и открытого $G$ таких, что $F \subset G$,
            верно 
            \[
                \exists U(F) \colon~ U(F) \subset \overline{U(F)} \subset G
            .\]
        \item Из нормальности и дизъюнктности $F_0, F_1$ очевидно включение
            \[
                F_0 \subset \underbrace{U(F_0)}_{G_0} \subset \overline{U(F_0)}
                \subset \underbrace{F_1^C}_{G_1}
            .\]
            Повторим, заменив $F_0$ на $\overline{G_0}$:
            \[
                \overline{G_0} \subset \underbrace{U(\overline{G_0})}_{G_{1/2}}
                \subset \overline{U(\overline{G_0})} \subset G_1
            .\]
            Аналогичным образом между $\overline{G_0}$ и $G_{1/2}$,
            $\overline{G_{1/2}}$ и $G_1$ получим $G_{1/4}$ и $G_{3/4}$
            соответственно. Продолжая этот процесс, определим $G_p$ для
            всех двоично-рациональных $p$.
        \item Пусть наконец
            \[
                f(x) = \inf{\{\, q \mid x \in G_q \,\}}
            .\]
            Из определения $f$ сразу следует
            \[
                f\big|_{F_0} = 0,~~ f\big|_{F_1} = 1
            .\]
            Осталось проверить непрерывность. Для этого проверим, что открыты
            все множества вида
            \[
                f^{-1}(-\infty, s)
            \]
            И замкнуты
            \[
                f^{-1}(-\infty, s]
            .\]
            Тогда будет выполнено топологическое определение непрерывности для
            базы топологии, состоящей из интервалов:
            \[
                f^{-1}(a, b) = f^{-1}(-\infty, b) \setminus f^{-1}(-\infty, a]
                \text{ -- открыто}
            .\]
            \begin{itemize}
                \item Проверим открытость $f^{-1}(-\infty, s)$. Для этого
                    достаточно показать, что
                    \[
                        f^{-1}(-\infty, s) = \bigcup_{\substack{q < s \\ 
                        q \text{ дв. рац.}}}{G_q}
                    .\]
                    Действительно, тогда $f^{-1}(-\infty, s)$ открыто как
                    объединение открытых множеств. Проверим равенство:
                    \begin{itemize}
                        \item[$\supset$] $x \in G_q \Lra f(x) \leqslant q < s$.
                        \item[$\subset$] $f(x) < s \Lra f(x) < q_1 < s$, $q$ -- 
                            двоично-рациональное. Поскольку $G_p$ монотонны по
                            $p$ и $f(x) < q_1$, получаем, что $x \in G_{q_1}$.
                    \end{itemize}
                \item Проверим замкнутость $f^{-1}(-\infty, s]$. Для этого
                    покажем
                    \[
                        f^{-1}(-\infty, s] = \bigcap_{\substack{q > s \\ 
                        q \text{ дв. рац.}}}{G_q} = \bigcap_{\substack{q > s \\
                        q \text{ дв. рац.}}}{\overline{G_q}}
                    .\]
                    Последнее выражение -- пересечение замкнутых множеств, замкнуто.
                    Проверим первое равенство:
                    \begin{itemize}
                        \item[$\subset$] $f(x) \leqslant s \Lra \inf{\{\,
                            q \mid x \in G_q \,\}} \leqslant s$. Поэтому
                            \[
                                \forall q > s~ x \in G_q \Lra x \in \bigcap_{q > s}{G_q}
                            .\]
                        \item[$\supset$] Раз $x \in G_q$ для любого двоично-рационального 
                            $q > s$, получаем
                            \[
                                f(x) = \inf{\{\, q \mid x \in G_q \,\}} \leqslant s
                            .\]
                    \end{itemize}
                    Проверим второе равенство:
                    \begin{itemize}
                        \item[$\subset$] Очевидно.
                        \item[$\supset$] Зафиксируем двоично-рациональное $r$
                            такое, что
                            \[
                                s < r < s
                            .\]
                            Тогда по построению
                            \[
                                \overline{G_r} \subset G_q
                            .\]
                            Отсюда получаем
                            \[
                                \bigcap_{q > s}{G_q} \supset \bigcap_{q > r > s}
                                {\overline{G_r}} \supset \bigcap_{r > s}{\overline{G_r}}
                            .\]
                    \end{itemize}
            \end{itemize}
    \end{itemize}
\end{proof}

\begin{definition}
    \textit{Финитной функцией в $\Rm$} называется функция $f$ такая, что
\[
    \exists B(a, r)\colon~ f\big|_{\overline{B}} = 0
.\] 
    По умолчанию, $f$ непрерывна. Множество непрерывных финитных функций
    обозначается $C_0(\Rm)$.
\end{definition}

\begin{theorem}
    $C_0(\Rm)$ плотно в $L^p(E, \lambda_m)$ при $1 \leqslant p < +\infty$,
    $E \in \mathfrak{M}^m$.
\end{theorem}
\begin{proof}
    \enewline
    \begin{itemize}
        \item Пусть $g \in L^p(E, \lambda_m)$. Нам нужно показать, что
            \[
                \forall \e > 0~ \exists f \in C_0(\Rm) \colon~ \norm{g - f\big|_E}_
                {L^p(E, \lambda_m)} < \e
            .\]
            Заменим $g$ на $0$ на дополнении $E$. Это никак не скажется на норме в
            $L^p(E, \lambda_m)$. Тогда
            \[
                \norm{g - f\big|_E}_{L^p(E, \lambda_m)} \leqslant \norm{g - f}_
                {L^p(\Rm, \lambda_m)}
            .\]
            Сократим запись: $\norm{}_{L^p(\Rm, \lambda_m)}$ обозначим как 
            $\norm{}_p$. Осталось построить функцию $f$ такую, чтобы выполнялось
            \[
                \norm{g - f}_p < \e
            .\]
        \item Пользуясь теоремой об аппроксимации, приблизим срезки $g$
            ступенчатыми функциями:
            \[
                \exists h_n \colon~ h_n \to g_+
            .\]
            Имеем
            \[
                h_n \leqslant g,~ g \in L^p
            .\]
            Отсюда по теореме о сходимости в $L^p$ и по мере
            \[
                \exists h \colon~ \norm{h - g_+}_p < \e    
            .\]
            Пусть
            \[
                h = \sum_k{c_k \chi_{A_k}}
            .\]
            Приблизим характеристические функции ступенчатыми. $A_k$ измеримы,
            поэтому
            \[
                \forall \e > 0~ \exists F_k \subset A_k \subset G_k \colon~
                \lambda_m(G_k \setminus A_k) < \frac{\e^p}{|c_k|^p k^p}
            .\]
            Здесь $F_k$ замкнуто, $G_k$ открыто. По лемме Урысона $\exists f_k$
            непрерывная такая, что
            \[
                0 \leqslant f_k \leqslant 1,~~ f\big|_{G_k} = 1,~~
                f\big|_{G_k^C} = 0
            .\]
            Положим
            \[
                f(x) = \sum_k{c_k f_k(x)}
            .\]
        \item Убедимся, что $f$ действительно хорошее приближение $g_+$, то есть
            \[
                \norm{g_+ - f} < \e
            .\]
            Проделав аналогичное действие для $g_-$, получим требуемое приближение
            $g$ по метрике в $L^p(\Rm, \lambda_m)$. Для начала оценим 
            $\norm{f_k - \chi_{A_k}}_p$:
            \[
                \norm{f_k - \chi_{A_k}}_p = \intl_{\Rm}{|f_k - \chi_{A_k}|^p}
                \leqslant \intl_{G_k \setminus F_k}{1^p} < \frac{\e^p}{|c_k|^p k^p}
            .\]
            Теперь покажем требуемое:
            \begin{align*}
                \norm{g_+ - f}_p 
                &\leqslant \norm{g^+ - \sum_k{c_k \chi_{A_k}}}_p
                + \norm{\sum_k{c_k \cdot (\chi_{A_k} - f_k)}}_p \\
                &\leqslant \e + \sum_k{|c_k| \cdot \norm{f_k - \chi_{A_k}}_p} \\
                &\leqslant \e + \sum_k{|c_k| \cdot \frac{\e}{|c_k| k}} = 2\e
            .\end{align*}
    \end{itemize}
\end{proof}

\begin{remark}
    Условие $p \neq +\infty$ существенно.
\end{remark}

\begin{definition}
    Множество непрерывных $T$-периодических функций будем обозначать $\widetilde{C}([0, T])$.
\end{definition}

\begin{theorem}(О непрерывности сдвига)
    
    Пусть $f_h(x) = f(x + h)$. Тогда
    \begin{enumerate}
        \item $f$ равномерно непрерывна в $\Rm \Lra \norm{f_h - f}_{\infty}
            \xrightarrow[h \to 0]{} 0$.
        \item $1 \leqslant p < +\infty$, $f \in L^p \Lra \norm{f_h - f}_p 
            \xrightarrow[h \to 0]{} 0$.
        \item $f \in \widetilde{C}([0, T]) \Lra \norm{f_h - f}_{\infty} 
            \xrightarrow[h \to 0]{} 0$.
        \item $1 \leqslant p < +\infty$, $f \in L^p([0, T]) \Lra \norm{f_h - f}_
            {p} \xrightarrow[h \to 0]{} 0$.
    \end{enumerate} 
\end{theorem}
\begin{proof}
    \enewline
    \begin{enumerate}
        \item Выпишем опредение равномерной непрерывности:
            \[
                \forall \e > 0~\exists \delta\colon \forall x, |h| < \delta~
                |f(x + h) - f(x)| = |f_h(x) - f(x)| < \e
            .\]
            То есть:
            \[
                \forall \e > 0~\exists \delta\colon \forall |h| < \delta~
                \sup_x{|f_h - f|} \leqslant \e \Lra \norm{f_h - f}_\infty 
                \leqslant \e
            .\]
        \item Пользуясь теоремой о полноте $C_0(\Rm)$ в $L^p$ подберем
            финитную непрерывную $g$ такую, что:
            \[
                \norm{f - g}_p < \frac{\e}{3} 
            .\]
            Тогда:
            \[
                \norm{f_h - f}_p \leqslant \norm{f - g}_p + \norm{g_h - f_h}_p
                + \norm{g_h - g}_p
            .\]
            $\norm{f_h - g_h} = \norm{f - g}$, поэтому:
            \[
                \norm{f_h - f} \leqslant \frac{2}{3} \e + \norm{g_h - g}_p
            .\]
            Зафиксируем шар $B(0, R + 1)$, содержащий носитель $g$. Пусть
            $|h| < 1$. Тогда
            \[
                \norm{g_h - g}_p = \norm{g_h - g}_{L^p(B(0, R + 1))} \leqslant
                \norm{g_h - g}_\infty \cdot \parens*{\intl_B{1}}^{\frac{1}{p}} =
                \underbrace{\norm{g_h - g}_\infty}_{\to 0} \cdot 
                {\lambda_m(B)}^{\frac{1}{p}} \xrightarrow[|h| \to 0]{} 0
            .\]
        \item Функция непрерывна на компакте, а поэтому равномерно непрерывна на
            нём. Раз так, работает первый пункт теоремы.
        \item Проведем доказательство, аналогичное доказательству второго пункта
            с точностью до оценки
            \[
                \norm{g_h - g}_p \leqslant \norm{g_h - g}_\infty \cdot 
                \lambda_m([0, T])^{\frac{1}{p}} \to 0
            .\]
    \end{enumerate}
\end{proof}

