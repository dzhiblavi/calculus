\chapter{Ряды Фурье}
\section{Пространство $L^p$}

\begin{definition}
    Комплексное отображение $f \colon X \to \bC$ назовем \textit{измеримым}, если
    $f(x) = g(x) + i h(x)$, $g, h \colon X \to \R$, причем $g$, $h$ измеримы.
\end{definition}

\begin{definition}
    Аналогично определим \textit{суммируемые} комплексные отображения.
\end{definition}
 
\begin{definition}
   Пусть $f \colon X \to \bC$, $f(x) = g(x) + i h(x)$, $g, h \colon X \to \R$. Тогда
   определим интеграл:
\[
    \intl_{E}{f \dd\mu} \defeq \intl_{E}{g \dd\mu} + i \intl_{E}{h \dd\mu}
.\] 
\end{definition}

\begin{remark}
\[
    \left|\intl_{E}{f \dd\mu}\right| \leqslant \intl_{E}{|f| \dd\mu}
.\] 
\end{remark}

\begin{theorem}(Интегральное неравенство Гёльдера)
    
    Пусть $p, q > 1$, $\frac{1}{p} + \frac{1}{q} = 1$, $f, g \colon X \to \bC$
    -- измеримые почти везде заданные функции.
    Тогда
\[
    \intl_{X}{|f g| \dd\mu} \leqslant \left(\intl_{X}{|f|^p \dd\mu}\right)^{\frac{1}{p}}
    \cdot \left(\intl_{X}{|g|^q \dd\mu}\right)^{\frac{1}{q}}
.\] 
\end{theorem}

\begin{theorem}(Интегральное неравенство Минковского)
    
    Пусть $f, g \colon X \to \bC$, $p \geqslant 1$, тогда
\[
    \left(\intl_{X}{|f + g|^p \dd\mu}\right)^{\frac{1}{p}} \leqslant
    \left(\intl_{X}{|f|^p \dd\mu}\right)^{\frac{1}{p}} 
    + \left(\intl_{X}{|g|^p \dd\mu}\right)^{\frac{1}{p}}
.\]  
\end{theorem}

\begin{definition}
    Пусть $\langle X, \cA, \mu \rangle$ -- пространство с мерой. Тогда для
    $1 \leqslant p < +\infty$ положим
\[
    \cL^p(X, \mu) \defeq \left\{f \colon X \to \bC / \R \mid 
    f \text{ измерима}, \intl_{X}{|f|^p \dd\mu} < +\infty\right\}
.\] 
\end{definition}

\begin{remark}
    $\cL^p(X, \mu)$ -- линейное пространство.
\end{remark}

\begin{definition}
    Зададим на $\cL^p$ отношение эквивалентности: $f \sim g$ тогда и
    только тогда, когда $f = g$ почти везде. Положим
\[
    L^p(X, \mu) \defeq \cL^p(X, \mu) / \sim
.\] 
\end{definition}

\begin{definition}
    В $L^p$ заведем норму: $\displaystyle \norm{[f]} \defeq 
    \left(\int_{X}{|f|^p \dd\mu}\right)^{\frac{1}{p}}$.
\end{definition}

\begin{definition}
    Пусть $f \colon X \to \Rbar$ задана почти везде. Тогда \textit{существенным супремумом
    $f$} называется
\[
    \esssup_{X}{f} \defeq \inf{\{A \in \Rbar \mid f(x) \leqslant A \text{ п.в.}\}}
.\] 
\end{definition}

\begin{theorem}(Свойства существенного супремума)
    \begin{itemize}
        \item $\esssup_{X}{f} \leqslant \sup_{X}{f}$.
        \item $f(x) \leqslant \esssup_{X}{f}$ при почти всех $x$.
        \item $\displaystyle \left|\int_{X}{fg \dd\mu}\right| 
               \leqslant \esssup_{X}{|f|} \cdot \int_{X}{|g|}$.
    \end{itemize} 
\end{theorem}
\begin{proof}
    \enewline
    \begin{itemize}
        \item Супремум есть в множестве, по которому берется инфимум 
            в определении существенного супремума.
        \item Пусть $M = \esssup_{X}{f}$. Тогда по определению инфимума:
            \[
                \forall n \in \bN~ f(x) \leqslant M + \frac{1}{n}
            \]
            при почти всех $x$. Объединяем все эти неравенства (их
            счетное число) и получаем требуемое.
        \item По предыдущему пункту $f$ может быть больше $\esssup_X{f}$ только
            на множестве $e$ меры $0$. Поэтому
            \[
                \abs*{\intl_X{f g \dd \mu}} \leqslant
                \intl_X{\abs*{f g} \dd \mu} \leqslant
                \underbrace{\intl_e{\abs*{f g} \dd \mu}}_{= 0} 
                + \intl_{X \setminus e}{\abs*{f g} \dd \mu} \leqslant
                \esssup_X{f} \cdot \intl_{X}{\abs*{g} \dd \mu}
            .\]
    \end{itemize}
\end{proof}

\begin{definition}
    Для $p = \infty$:
\[
    \cL^{\infty}(X, \mu) \defeq \left\{f \colon X \to \bC / \R \mid 
    f \text{ измерима}, \esssup_{X}{f} < +\infty\right\}
.\] 
\end{definition}

\begin{remark}
    $\cL^{\infty}(X, \mu)$ -- линейное пространство.
\end{remark}

\begin{definition}
    Пространство $L^{\infty}$ зададим аналогично конечному случаю.
    Нормой на этом пространстве положим $\esssup$.
\end{definition}

\begin{theorem}(О вложении пространств $L^p$)
    
    Пусть $\mu(X) < +\infty$, $1 \leqslant r < s \leqslant +\infty$.
    Тогда
    \begin{itemize}
        \item $L^r(X, \mu) \subset L^s(X, \mu)$.
        \item $\norm{f}^s \leqslant \mu(X)^{\frac{1}{s} - \frac{1}{r}}
               \cdot \norm{f}_r$.
    \end{itemize} 
\end{theorem}
\begin{proof}
    \enewline
    \begin{itemize}
        \item Первый пункт очевидным образом следует из второго.
        \item Пусть $r = +\infty$. Тогда
            \[
                \norm{f}_s = \parens*{\intl_X{\abs{f}^s}}^{\frac{1}{s}} \leqslant
                \parens*{\intl_X{\norm{f}^s_\infty}}^{\frac{1}{s}} = 
                \norm{f}_\infty \cdot {\mu(X)}^{\frac{1}{s}}
            .\]
            Неравенство выполнено потому, что функция почти везде не превосходит
            свой существенный супремум. Пусть теперь $r < +\infty$:
            \[
                \norm{f}^s_s = \intl_X{|f|^s \cdot 1 \dd \mu}
                \underset{\text{Гельдер}}{\leqslant} \parens*{\intl_X
                {\parens*{|f|^s}^{\frac{r}{s}}}}^{\frac{s}{r}} \cdot 
                \parens*{\intl_X{1}}^{\frac{r - s}{s}} =
                \parens*{\intl_X{|f|^r}}^{\frac{s}{r}} \cdot 
                {\mu(X)}^{1 - \frac{s}{r}} =
                \norm{f}_r^s \cdot {\mu(X)}^{1 - \frac{s}{r}}
            .\]
    \end{itemize}
\end{proof}

\begin{corollary}
    Пусть $\mu(E) < +\infty$, $1 \leqslant s < r \leqslant +\infty$, $f_n, f \in L^s$,
    $f_n \xrightarrow[L^r]{} f$, тогда $f_n \xrightarrow[L^s]{} f$.
\end{corollary}

\begin{theorem}(О сходимости в $L^p$ и по мере)
    
    Пусть $1 \leqslant r < +\infty$, $f_n, f \in L^p$, тогда
    \begin{itemize}
        \item $f_n \xrightarrow[L^p]{} f \Lra f_n \underset{\mu}{\Lra} f$.
        \item $f_n \underset{\mu}{\Lra} f$, либо $f_n \to f$ почти везде, тогда
            если $\exists g \in L^p\colon~ |f_n| \leqslant g$, то
            $f_n \xrightarrow[L^p]{} f$.
    \end{itemize} 
\end{theorem}
\begin{proof}
    \enewline
    \begin{itemize}
        \item Обозначим
            \[
                X_n(\e) = X(|f_n - f| \geqslant \e)
            .\]
            Тогда на $X_n(\e)$ выполнено:
            \[
                1 \leqslant \frac{|f_n - f|^p}{\e^p}
            .\]
            Проверим сходимость по мере:
            \[
                \mu{X_n(\e)} = \intl_{X_n(\e)}{1 \dd \mu} \leqslant
                \frac{1}{\e^p} \intl_{X_n(\e)}{|f_n - f|^p \dd \mu} =
                \frac{1}{\e^p} \norm{f_n - f}_p \xrightarrow[n \to +\infty]{} 0
            .\]
        \item Из сходимости по мере имеем сходимость почти везде вдоль
            подпоследовательности:
            \[
                \exists n_k \colon~ f_{n_k} \xrightarrow[n \to +\infty]{} f
            \]
            при почти всех $x$. Поскольку $f_{n_k} \to f$, $|f_{n_k}| \leqslant 
            g$, верно:
            \[
                |f| \leqslant g
            .\]
            Поэтому:
            \[
                |f_n - f|^p \leqslant (2 g)^p
            .\]
            Интеграл последней функции конечен, так как $g \in L^p$.
            В таком случае по теореме Лебега заключаем
            \[
                \norm{f_n - f}^p_p = \intl_X{|f_n - f|^p \dd \mu}
                \xrightarrow[n \to +\infty]{} 0
            .\]
    \end{itemize}
\end{proof}

\begin{remark}
    $L^{\infty}$ -- полное метрическое пространство.
\end{remark}

\begin{theorem}(Полнота пространств $L^p$)
    
    $\forall 1 \leqslant p \leqslant \infty$ $L^p$ полно.
\end{theorem}
\begin{proof}
    Докажем утверждение для конечных $p$. 
    \begin{itemize}
        \item Пусть $f_n$ -- фундаментальная в $L^p$ последовательность, то есть
            \[
                \forall \e > 0 \exists N_\e \colon \forall n, k > N_\e~
                \norm{f_n - f_k}_p < \e
            .\]
            Пользуясь этим фактом, построим последовательности индексов
            $k_i, n_i$ следующим образом:
            \[
                \e = \frac{1}{2^i}~ \exists N_i \colon k_i > n_i > N_i \Lra
                \norm{f_{n_i} - f_{k_i}}_p < \e
            .\]
            Выделим из $n_i$ возрастающую последовательность:
            \[
                n_k = \max_{1 \leqslant i \leqslant k}{n_i}
            .\]
            Тогда верно неравенство:
            \[
                \sum_{k = 1}^{+\infty}{\norm{f_{n_{k + 1}} - f_{n_k}}_p} \leqslant 1
            .\]
        \item Зафиксируем функции $f_i$ -- представители соответствующих
            классов эквивалентности в $\cL^p$. Положим
            \[
                S(x) = \sum_{k = 1}^{+\infty}{|f_{n_{k + 1}}(x) - f_{n_k}(x)|}
                \in [0, +\infty]
            .\]
            Оценим частичные суммы ряда:
            \[
                \norm{S_N}_p \underset{\triangle}{\leqslant} \sum_{i = 1}^N
                {\norm{f_{n_{k + 1}} - f_{n_k}}_p} < 1
            .\]
            Из этой оценки следует:
            \[
                \intl_X{S_N^p} \leqslant 1
            .\]
            Из чего по теореме Фату получаем:
            \[
                \intl_X{S^p} \leqslant 1
            .\]
            Поэтому $S^p$ суммируема, а значит, $S(x)$ конечна почти везде.
        \item Пусть
            \[
                f(x) = f_{n_1} + \sum_{k \geqslant 1}{(f_{n_{k + 1}} - f_{n_k})}
            .\]
            При этом
            \[
                \sum_{1 \leqslant k \leqslant m - 1}{(f_{n_{k + 1}} - f_{n_k})}
                = f_{n_m} 
            .\]
            Помня про сходимость соответствующего ряда, получаем
            \[
                f_{n_m} \to f
            \]
            при почти всех $x$.
        \item Проверим наконец сходимость последовательности. Из равномерной 
            сходимости имеем:
            \[
                \forall \e > 0 \exists N \colon \forall n, m = n_k > N~
                \norm{f_n - f_m}_p < \e
            .\]
            Поэтому справедливо:
            \[
                \e^p > \norm{f_n - f_{n_k}}^p_p = \intl_X{|f_{n_k} - f_n|^p \dd \mu}
            .\]
            Отсюда по теореме Фату получаем:
            \[
                \intl_X{|f_n - f|^p \dd \mu} < \e^p
            .\]
    \end{itemize}
\end{proof}

\begin{definition}
    Пусть $X$ -- топологическое пространство, тогда множество
    $A \subset X$ называется всюду плотным, если $\Cl(A) = X$.
    Иначе говоря, $\Int(X \setminus A) = \varnothing$, 
    или \\ $\forall x \in X~ \forall U(x)~ U(x) \cap A \neq \varnothing$.
\end{definition}

\begin{definition}
    Множество всех ступенчатых функций $g \colon X \to \Rbar$
    обозначим $St(X)$.
\end{definition}

\begin{lemma}
    Пусть $1 \leqslant p \leqslant +\infty$, тогда множество
    $St(X)$ плотно в $L^p$.
\end{lemma}
\begin{proof}
    \enewline
    \begin{itemize}
        \item Пусть $p = +\infty$. Подменим $f \in L^\infty$ на
            множестве меры ноль так, чтобы
            \[
                \esssup_X{f} = \sup_X{f}
            .\]
            Тогда по теореме об аппроксимации существуют последовательности 
            ступенчатых функций $\f_n, \psi_n$ такие, что
            \[
                \f_n \rcon f_+,~~ \psi_n \rcon f_-
            .\]
            Из равномерной сходимости получаем
            \[
                \norm{\f_n - f_+}_\infty \to 0,~~ \norm{\psi_n - f_-}_\infty \to 0
            .\]
        \item Пусть теперь $p$ конечно. Рассмотрим неотрицательную $f \in L^p$.
            По теореме об аппроксимации
            \[
                \exists \f_n \to f
            ,\]
            причем $\f_n$ возрастают с номером $n$ и неотрицательны. Проверим,
            что эта последовательность ступенчатых функций аппроксимирует
            $f$ в смысле $L^p$. Для этого заметим, что поскольку $\f_n \leqslant f$,
            верно
            \[
                |\f_n - f|^p \leqslant |f|^p
            .\]
            При этом, $|f|^p$ суммируема. Значит, по теореме Лебега о
            мажорированной сходимости справедливо:
            \[
                \norm{\f_n - f}^p_p = \intl_X{|\f_n - f| \dd \mu} \to 0
            .\]
    \end{itemize}
\end{proof}

\begin{definition}(Четвертая аксиома отделимости)
    
    Топологическое пространство называется \textit{нормальным},
    если в нем любые два замкнутые непересекающиеся множества
    отделимы, причем любое одноточечное множество замкнуто.
\end{definition}

\begin{lemma}(Урысон)
    
    Пусть $X$ -- нормальное топологическое пространство, 
    $F_0, F_1$ -- замкнутые непересекающиеся множества. Тогда
    существует непрерывная функция $f \colon X \to \R$, такая, что
    \begin{itemize}
        \item $0 \leqslant f \leqslant 1$.
        \item $f\big|_{F_0} = 0$.
        \item $f\big|_{F_1} = 1$.
    \end{itemize} 
\end{lemma}
%:: NOTE all proof

\begin{definition}
    \textit{Финитной функцией в $\Rm$} называется функция $f$ такая, что
\[
    \exists B(a, r)\colon~ f\big|_{\overline{B}} = 0
.\] 
    По умолчанию, $f$ непрерывна.
\end{definition}

\begin{theorem}
    Множество финитных функций плотно в $L^p$ при $1 \leqslant p < +\infty$.
\end{theorem}
%:: NOTE all proof

\begin{remark}
    Условие $p \neq +\infty$ существенно.
%:: NOTE all example
\end{remark}

\begin{definition}
    Множество непрерывных $T$-периодических функций будем обозначать $\widetilde{C}([0, T])$.
\end{definition}

\begin{theorem}(О непрерывности сдвига)
    
    Пусть $f_h(x) = f(x + h)$. Тогда
    \begin{itemize}
        \item $f$ равномерно непрерывна в $\Rm \Lra \norm{f_h - f}_{\infty} \xrightarrow[h \to 0]{} 0$.
        \item $1 \leqslant p < +\infty$, $f \in L^p \Lra \norm{f_h - f}_p \xrightarrow[h \to 0]{} 0$.
        \item $f \in \widetilde{C}([0, T]) \Lra \norm{f_h - f}_{\infty} \xrightarrow[h \to 0]{} 0$.
        \item $1 \leqslant p < +\infty$, $f \in L^p([0, T]) \Lra \norm{f_h - f}_{\infty} \xrightarrow[h \to 0]{} 0$.
    \end{itemize} 
\end{theorem}
%:: NOTE all proof


