\chapter{Ряды Фурье}
\section{Пространство $L^p$}

\begin{definition}
    Комплексное отображение $f \colon X \to \bC$ назовем \textit{измеримым}, если
    $f(x) = g(x) + i h(x)$, $g, h \colon X \to \R$, причем $g$, $h$ измеримы.
\end{definition}

\begin{definition}
    Аналогично определим \textit{суммируемые} комплексные отображения.
\end{definition}
 
\begin{definition}
   Пусть $f \colon X \to \bC$, $f(x) = g(x) + i h(x)$, $g, h \colon X \to \R$. Тогда
   определим интеграл:
\[
    \intl_{E}{f \dd\mu} \defeq \intl_{E}{g \dd\mu} + i \intl_{E}{h \dd\mu}
.\] 
\end{definition}

\begin{remark}
\[
    \left|\intl_{E}{f \dd\mu}\right| \leqslant \intl_{E}{|f| \dd\mu}
.\] 
\end{remark}

\begin{theorem}(Интегральное неравенство Гёльдера)
    
    Пусть $p, q > 1$, $\frac{1}{p} + \frac{1}{q} = 1$, $f, g \colon X \to \bC$
    -- измеримые почти везде заданные функции.
    Тогда
\[
    \intl_{X}{|f g| \dd\mu} \leqslant \left(\intl_{X}{|f|^p \dd\mu}\right)^{\frac{1}{p}}
    \cdot \left(\intl_{X}{|g|^q \dd\mu}\right)^{\frac{1}{q}}
.\] 
\end{theorem}
%:: NOTE all proof

\begin{theorem}(Интегральное неравенство Минковского)
    
    Пусть $f, g \colon X \to \bC$, $p \geqslant 1$, тогда
\[
    \left(\intl_{X}{|f + g|^p \dd\mu}\right)^{\frac{1}{p}} \leqslant
    \left(\intl_{X}{|f|^p \dd\mu}\right)^{\frac{1}{p}} 
    + \left(\intl_{X}{|g|^p \dd\mu}\right)^{\frac{1}{p}}
.\]  
\end{theorem}
%:: NOTE all proof

\begin{definition}
    Пусть $\langle X, \cA, \mu \rangle$ -- пространство с мерой. Тогда для
    $1 \leqslant p < +\infty$ положим
\[
    \cL^p(X, \mu) \defeq \left\{f \colon \text{п.в. } X \to \bC(\R) \mid 
    f \text{ измерима}, \intl_{X}{|f|^p \dd\mu} < +\infty\right\}
.\] 
\end{definition}

\begin{remark}
    $\cL^p(X, \mu)$ -- линейное пространство.
\end{remark}

\begin{definition}
    Зададим на $\cL^p$ отношение эквивалентности: $f \sim g$ тогда и
    только тогда, когда $f = g$ почти везде. Положим
\[
    L^p(X, \mu) \defeq \cL^p(X, \mu) / \sim
.\] 
\end{definition}

\begin{definition}
    В $L^p$ заведем норму: $\displaystyle \norm{[f]} \defeq 
    \left(\int_{X}{|f|^p \dd\mu}\right)^{\frac{1}{p}}$.
\end{definition}

\begin{definition}
    Пусть $f \colon X \to \Rbar$ задана почти везде. Тогда \textit{существенным супремумом
    $f$} называется
\[
    \esssup_{X}{f} \defeq \inf{\{A \in \Rbar \mid f(x) \leqslant A \text{ п.в.}\}}
.\] 
\end{definition}

\begin{theorem}(Свойства существенного супремума)
    \begin{itemize}
        \item $\esssup_{X}{f} \leqslant \sup_{X}{f}$.
        \item $f(x) \leqslant \esssup_{X}{f}$ при почти всех $x$.
        \item $\displaystyle \left|\int_{X}{fg \dd\mu}\right| 
               \leqslant \esssup_{X}{|f|} \cdot \int_{X}{|g|}$.
    \end{itemize} 
\end{theorem}
%:: NOTE all proof

\begin{definition}
    Для $p = \infty$:
\[
    \cL^{\infty}(X, \mu) \defeq \left\{f \colon \text{п.в. } X \to \bC(\R) \mid 
    f \text{ измерима}, \esssup_{X}{f} < +\infty\right\}
.\] 
\end{definition}

\begin{remark}
    $\cL^{\infty}(X, \mu)$ -- линейное пространство.
\end{remark}

\begin{definition}
    Пространство $L^{\infty}$ зададим аналогично конечному случаю.
    Нормой на этом пространстве положим $\esssup$.
\end{definition}

\begin{theorem}(О вложении пространств $L^p$)
    
    Пусть $\mu(X) < +\infty$, $1 \leqslant r < s \leqslant +\infty$.
    Тогда
    \begin{itemize}
        \item $L^r(X, \mu) \subset L^s(X, \mu)$.
        \item $\norm{f}^s \leqslant \mu(X)^{\frac{1}{s} - \frac{1}{r}}
               \cdot \norm{f}_r$.
    \end{itemize} 
\end{theorem}
%:: NOTE all proof

\begin{corollary}
    Пусть $\mu(E) < +\infty$, $1 \leqslant s < r \leqslant +\infty$, $f_n, f \in L^s$,
    $f_n \xrightarrow[L^r]{} f$, тогда $f_n \xrightarrow[L^s]{} f$.
\end{corollary}
%:: NOTE all proof

\begin{theorem}(О сходимости в $L^p$ и по мере)
    
    Пусть $1 \leqslant r < +\infty$, $f_n, f \in L^p$, тогда
    \begin{itemize}
        \item $f_n \xrightarrow[L^p]{} f \Lra f_n \underset{\mu}{\Lra} f$.
        \item $f_n \underset{\mu}{\Lra} f$, либо $f_n \to f$ почти везде, тогда
            если $\exists g \in L^p\colon~ |f_n| \leqslant g$, то
            $f_n \xrightarrow[L^p]{} f$.
    \end{itemize} 
\end{theorem}
%:: NOTE all proof

\begin{remark}
    $L^{\infty}$ -- полное метрическое пространство.
\end{remark}

\begin{theorem}(Полнота пространств $L^p$)
    
    $\forall 1 \leqslant p \leqslant \infty$ $L^p$ полно.
\end{theorem}
%:: NOTE all proof

\begin{definition}
    Пусть $X$ -- топологическое пространство, тогда множество
    $A \subset X$ называется всюду плотным, если $\Cl(A) = X$.
    Иначе говоря, $\Int(X \setminus A) = \varnothing$, 
    или \\ $\forall x \in X~ \forall U(x)~ U(x) \cap A \neq \varnothing$.
\end{definition}

\begin{definition}
    Множество всех ступенчатых функций $g \colon X \to \Rbar$
    обозначим $St(X)$.
\end{definition}

\begin{lemma}
    Пусть $1 \leqslant p \leqslant +\infty$, тогда множество
    $St(X) \cap L^p$ плотно в $L^p$.
\end{lemma}
%:: NOTE all proof

\begin{definition}(Четвертая аксиома отделимости)
    
    Топологическое пространство называется \textit{нормальным},
    если в нем любые два замкнутые непересекающиеся множества
    отделимы, причем любое одноточечное множество замкнуто.
\end{definition}

\begin{lemma}(Урысон)
    
    Пусть $X$ -- нормальное топологическое пространство, 
    $F_0, F_1$ -- замкнутые непересекающиеся множества. Тогда
    существует непрерывная функция $f \colon X \to \R$, такая, что
    \begin{itemize}
        \item $0 \leqslant f \leqslant 1$.
        \item $f\big|_{F_0} = 0$.
        \item $f\big|_{F_1} = 1$.
    \end{itemize} 
\end{lemma}
%:: NOTE all proof

\begin{definition}
    \textit{Финитной функцией в $\Rm$} называется функция $f$ такая, что
\[
    \exists B(a, r)\colon~ f\big|_B = 0
.\] 
    По умолчанию, $f$ непрерывна.
\end{definition}

\begin{theorem}
    Множество финитных функций плотно в $L^p$ при $1 \leqslant p < +\infty$.
\end{theorem}
%:: NOTE all proof

\begin{remark}
    Условие $p \neq +\infty$ существенно.
%:: NOTE all example
\end{remark}

\begin{definition}
    Множество непрерывных $T$-периодических функций будем обозначать $\widetilde{C}([0, T])$.
\end{definition}

\begin{theorem}(О непрерывности сдвига)
    
    Пусть $f_h(x) = f(x + h)$. Тогда
    \begin{itemize}
        \item $f$ равномерно непрерывна в $\Rm \Lra \norm{f_h - f}_{\infty} \xrightarrow[h \to 0]{} 0$.
        \item $1 \leqslant p < +\infty$, $f \in L^p \Lra \norm{f_h - f}_p \xrightarrow[h \to 0]{} 0$.
        \item $f \in \widetilde{C}([0, T]) \Lra \norm{f_h - f}_{\infty} \xrightarrow[h \to 0]{} 0$.
        \item $1 \leqslant p < +\infty$, $f \in L^p([0, T]) \Lra \norm{f_h - f}_{\infty} \xrightarrow[h \to 0]{} 0$.
    \end{itemize} 
\end{theorem}
%:: NOTE all proof


