\chapter{Интеграл}

\section{Определение интеграла}

\textit{Общий контекст: $\langle X, \cA, \mu \rangle$ --- пространство с мерой}

\begin{definition}
	Введем обозначение
\[
	\cL^0(X) = \{\, f \colon X \to \Rbar \mid f \text{ измерима и п.в. конечна} \,\}
.\]
\end{definition}

\begin{definition}
	Пусть $0 \leqslant f \colon X \to \Rbar$ --- ступенчатая функция, то есть
\[
	f = \sum_{fin} {\l_k \chi_{E_k}}
.\]
	Причем все $E_k$ измеримы. Интеграл такой функции определим следующим образом:
\[
	\int_X{f \dd\mu} \defeq \sum_{k}{\l_k \mu{E_k}}
.\]
\end{definition}

\begin{definition}
	Аналогично определим интеграл по измеримому множеству:
\[
	\int_E{f \dd\mu} \defeq \sum_k{\l_k \mu(E \cap E_k)}
.\]
\end{definition}

\begin{theorem}(Свойства интеграла ступенчатой функции)
	\enewline

	\begin{enumerate}
		\item Интеграл не зависит от допустимого разбиения.
		\item $f \leqslant g \Lra \int{f \dd\mu} \leqslant \int{g \dd\mu}$.
	\end{enumerate}
\end{theorem}
\begin{proof}
	\enewline
	\begin{enumerate}
		\item Пусть $f = \sum_k{\l_k \chi_{E_k}} = \sum_j{\a_j \chi_{F_j}}$. Тогда
			$f = \sum_{k, j}{\l_k \chi_{E_k \cap F_j}} = \sum_{k, j}{\a_k \chi_{E_k \cap F_j}}$. 
			Пользуясь этим, перепишем интеграл:
\[	
	\int_1{f} = \sum_k{\l_k \mu{E_k}} = \sum_k{\l_k \sum_j{\mu(E_k \cap F_j)}}
    = \sum_j{\a_j \sum_k{\mu(E_k \cap F_j)}} = \sum_j{\a_j \mu{F_j}} = \int_2{f}
.\]
		\item Воспользуемся общим допустимым разбиением:
\[
	\int{f} = \sum_k{\l_k \mu{E_k}} = \sum_{k, j}{\l_k \mu(E_k \cap F_j)} \leqslant
	\sum_{k, j}{\a_j \mu(E_k \cap F_j)} = \sum_j{\a_j \mu{F_j}} = \int{g}
.\]
	\end{enumerate}
\end{proof}

\begin{definition}
	Пусть $0 \leqslant f \colon X \to \Rbar$ измерима. Интеграл такой функции определим так:
\[
	\int_X{f \dd\mu} \defeq \sup_{\substack{0 \leqslant g \leqslant f \\ g \text{ ступенч.}}}
									{\int_X{g \dd\mu}}
.\]
\end{definition}

\begin{definition}
	Аналогично определим интеграл по измеримому множеству:
\[
	\int_E{f \dd\mu} \defeq \sup_{\substack{0 \leqslant g \leqslant f \\ g \text{ ступенч.}}}
									{\int_E{g \dd\mu}}
.\]
\end{definition}

\begin{theorem}(Свойства интеграла измеримой функции)
	\enewline

	\begin{itemize}
		\item Если функция ступенчатая, то интеграл совпадает с интегралом, определенным
			для ступенчатых функций.
		\item $0 \leqslant \int{f \dd\mu} \leqslant +\infty$.
		\item $0 \leqslant g \leqslant f$, $g$ ступенчатая, $f$ измеримая, тогда
			$\int{g \dd\mu} \leqslant \int{f \dd\mu}$.
		\item $0 \leqslant g \leqslant f$, $f$, $g$ измеримы, тогда $\int{g \dd\mu} \leqslant 
            \int{f \dd\mu}$.
	\end{itemize}
\end{theorem}
\begin{proof}
	\enewline
	\begin{enumerate}
		\item Очевидно, так как супремум реализуется на самой интегрируемой функции.
		\item[3.] Поскольку $g$ -- ступенчатая и $0 \leqslant g \leqslant f$, $g$ входит
			в супремум из определения интеграла $f$, поэтому автоматически
			$\int{g} \leqslant \int{f}$.
		\item Все ступенчатые функции, супремум по которым берется в определении интеграла 
			функции $g$, входят так же и в супремум для интеграла $f$, так как
			$0 \leqslant h \leqslant g \leqslant f$.
	\end{enumerate}
\end{proof}

\begin{definition}
	Пусть $f$ --- измеримая функция $X$, причем хотя бы один из интегралов срезок конечен.
	Для такой функции определим интеграл:
\[
	\int_X{f \dd\mu} \defeq \int_X{f_+ \dd\mu} - \int_X{f_- \dd\mu}
.\]
\end{definition}

\begin{definition}
	Определим интеграл по измеримому множеству:
\[
	\int_E{f \dd\mu} \defeq \int_X{f \cdot \chi_{E} \dd\mu}
.\]
\end{definition}

\begin{definition}
	Назовем функцию \textit{суммируемой}, если интегралы её срезок конечны.
\end{definition}

\begin{theorem}(Свойства интеграла)
	\enewline
	\begin{enumerate}
		\item Измеримая $f \geqslant 0 \Lra$ интеграл совпадает с предыдущим определением.
		\item $f$ суммируема $\Llra \int{|f| \dd\mu} < +\infty$. 
        \item Интеграл монотонен по функции, то есть для измеримых $f$, $g$ верно:
\[
    f \leqslant g \Lra \int_{E}{f \dd\mu} \leqslant \int_{E}{g \dd\mu}
.\] 
        \item $\displaystyle \int_E{1 \dd\mu} = \mu(E)$, $\displaystyle \int_E{0 \dd\mu} = 0$.
        \item Пусть $\mu(E) = 0$, $f$ измерима. Тогда
\[
    \int_E{f \dd\mu} = 0
.\] 
        \item $\displaystyle \int{-f \dd\mu} = -\int{f \dd\mu}$, $\displaystyle 
               \forall c > 0~ \int{c \cdot f \dd\mu} = c \cdot \int{f \dd\mu}$.
        \item Пусть $\displaystyle \exists \int_E{f \dd\mu}$, Тогда
\[
    \left|{\int_E{f \dd\mu}}\right| \leqslant {\int_E{|f| \dd\mu}}
.\] 
        \item Пусть $f$ измерима на $E$, $\mu(E) < +\infty$, 
            $\forall x \in E~ A \leqslant f(x) \leqslant B$,
            тогда
\[
    A \cdot \mu(E) \leqslant \int_E{f \dd\mu} \leqslant B \cdot \mu(E)
.\] 
	\end{enumerate}
\end{theorem}
\begin{proof}
    \enewline
	\begin{enumerate}
        \item[2.] Следует из аддитивности интеграла по функции, что будет доказано позже.
        \item Для неотрицательных $f, g$ это уже было доказано. Для произвольных воспользуемся
            определением и тем соображением, что $f^+ \leqslant g^+$ и $f^- \geqslant g^-$:
\[
    \int_E{f} = \int_E{f^+} - \int_E{f^-} \leqslant \int_E{g^+} - \int_E{g^-} = \int_E{g}
.\]
        \item[5.] Если $f$ ступенчата, то утверждение очевидно. Если $f \geqslant 0$ и измерима,
            то супремум из определения равен нулю. Если $f$ -- произвольная измеримая функция,
            то  $\int{f} = \int{f^+} - \int{f^-} = 0$.
        \item Очевидным образом следует из определений и того, что $\sup{c A} = c \sup{A}$.
        \item $-|f| \leqslant f \leqslant |f| \Lra -\int{|f|} \leqslant \int{f} \leqslant \int{|f|}$.
	\end{enumerate}
\end{proof}


\begin{lemma}
    Пусть $A = \bigsqcup_i{A_i}$, $A, A_i \in \cA$, $g \colon X \to \Rbar$, $g \geqslant 0$, 
    ступенчата. Тогда
\[
    \intl_A{g \dd\mu} = \sum_i{\intl_{A_i}{g \dd\mu}}
.\] 
\end{lemma}
\begin{proof}
    Пусть $g = \sum_k{\l_k \chi_{E_k}}$, тогда
\[
    \intl_A{g \dd\mu} = \sum_k{\l_k \mu(E_k \cap A)}
.\]
    Воспользуемся счетной аддитивностью меры:
\[
    \sum_k{\l_k \mu(E_k \cap A)} = \sum_k{\l_k \sum_i{\mu(E_k \cap A_i)}}
.\]
    Последний ряд сходится абсолютно, поэтому можно переставить порядок суммирования:
\[
    \sum_k{\l_k \sum_i{\mu(E_k \cap A_i)}} = \sum_i{\sum_k{\l_k \mu(E_k \cap A_i)}}
    = \sum_i{\intl_{A_i}{g \dd\mu}}
.\]
\end{proof}

\begin{theorem}
    Пусть $A = \bigsqcup_i{A_i}$, $A, A_i \in \cA$, $f \colon X \to \Rbar$, $f \geqslant 0$,
    измерима на $A$. Тогда
\[
    \intl_A{f \dd\mu} = \sum_i{\intl_{A_i}{f \dd\mu}}
.\] 
\end{theorem}
\begin{proof}
    \enewline
    \begin{itemize}
        \item[($\leqslant$)] Левая часть равенства аппроксимируется ступенчатыми функциями \\
            $0 \leqslant g \leqslant f$. Для них имеем:
\[
    \intl_A{g \dd\mu} = \sum_i{\intl_{A_i}{g \dd\mu}} \leqslant \sum_i{\intl_{A_i}}{f \dd\mu}
.\]
            Теперь имеем:
\[
    \intl_{A}{f \dd\mu} = \sup{\intl_{A}{g \dd\mu}} \leqslant \sum_i{\intl_{A_i}}{f \dd\mu}
.\]
        \item[($\geqslant$)] Для начала рассмотрим случай, когда $A = A_1 \sqcup A_2$. Рассмотрим
            ступенчатую функцию $0 \leqslant g \leqslant f$ и функции $g_1, g_2$ такие, 
            что $g_i\big|_{A_i} = g$, $g_i\big|_{\overline{A_i}} = 0$.
            Очевидно, что $g_1 + g_2 = g$ на $A$. Тогда по построению:
\[
    \intl_{A_1}{g_1 \dd\mu} + \intl_{A_2}{g_2 \dd\mu} = \intl_{A}{(g_1 + g_2)\dd\mu}
    = \intl_{A}{g \dd\mu} \leqslant \intl_{A}{f \dd\mu}
.\]
            Возьмём супремум от обеих частей сначала по $g_1$, потом по $g_2$:
\[
    \intl_{A_1}{f \dd\mu} + \intl_{A_2}{f \dd\mu} \leqslant \intl_{A}{f \dd\mu}
.\]
            Теперь разберемся с бесконечным случаем. Пусть $A = A_1 \sqcup A_2 \sqcup \ldots \sqcup A_n \sqcup B_n$,
            где $B_n = \bigsqcup_{i > n}{A_i}$. Тогда, пользуясь уже доказанным фактом для конечных разбиений,
            имеем:
\[
    \intl_{A}{f \dd\mu} \geqslant \sum_{i = 1}^{n}{\intl_{A_i}{f \dd\mu}} + \intl_{B_n}{f \dd\mu} \geqslant 
    \sum_{i = 1}^{n}{\intl_{A_i}{f \dd\mu}}
.\]
            Совершая предельный переход при $n \to +\infty$, имеем:
\[
    \intl_{A}{f \dd\mu} \geqslant \sum_{i = 1}^{+\infty}{\intl_{A_i}{f \dd\mu}}
.\]
    \end{itemize}
\end{proof}

\begin{corollary}
    Пусть $f \colon X \to \Rbar$, $f \geqslant 0$, измерима. Зададим отображение:
    \begin{align*}
        \nu \colon &\cA \to \Rbar_{\geqslant 0} \\ 
                   &E \mapsto \int_E{f \dd\mu}
    .\end{align*}
    Тогда $\nu$ -- мера.
\end{corollary}
\begin{proof}
    Единственное, что нужно проверить, это счетную аддитивность. Она как раз и проверена в теореме.
\end{proof}

\begin{lemma}
    Пусть $f$ суммируема, $g$ измерима, причем $f = g$ при почти всех $x$. Тогда
    $\displaystyle \intl_E{f \dd\mu} = \intl_E{g \dd\mu}$.
\end{lemma}
\begin{proof}
    Пусть $e \in \cA\colon~ \mu{e} = 0$, $f = g$ на $E \setminus e$. Тогда
\[
    \intl_{E}{f \dd\mu} = \intl_{E \setminus e}{f \dd\mu} + \intl_{e}{f \dd\mu}
    = \intl_{E \setminus e}{f \dd\mu} = \intl_{E \setminus e}{g \dd\mu}
    = \intl_{E \setminus e}{g \dd\mu} + \intl_e{g \dd\mu} = \intl_{E}{g \dd\mu}
.\]
\end{proof}

