\chapter{Интеграл}

\section{Собственно, интеграл}

\textit{Общий контекст: $\langle X, \cA, \mu \rangle$ --- пространство с мерой}

\begin{definition}
	Введем обозначение
\[
	\cL^0(X) = \{\, f \colon X \to \Rbar \mid f \text{ измерима и п.в. конечна} \,\}
\]
\end{definition}

\begin{definition}
	Пусть $0 \leqslant f \colon X \to \Rbar$ --- ступенчатая функция, то есть
\[
	f = \sum_{fin} {\l_k \chi_{E_k}}
\]
	Причем все $E_k$ измеримы. Интеграл такой функции определим следующим образом:
\[
	\int_X{f \d\mu} \defeq \sum_{k}{\l_k \mu{E_k}}
\]
\end{definition}

\begin{definition}
	Аналогично определим интеграл по измеримому множеству:
\[
	\int_E{f} \defeq \sum_k{\l_k \mu{E \cap E_k}}
\]
\end{definition}

\begin{theorem}(Свойства интеграла ступенчатой функции)
	\enewline

	\begin{itemize}
		\item Интеграл не зависит от допустимого разбиения.
		\item $f \leqslant g \Lra \int{f} \leqslant \int{g}$.
	\end{itemize}
\end{theorem}

\begin{definition}
	Пусть $0 \leqslant f \colon X \to \Rbar$ измерима. Интеграл такой функции определим так:
\[
	\int_X{f \d\mu} \defeq \sup_{\substack{0 \leqslant g \leqslant f \\ g \text{ ступенч.}}}
									{\int_X{g \d\mu}}
\]
\end{definition}

\begin{definition}
	Аналогично определим интеграл по измеримому множеству:
\[
	\int_E{f \d\mu} \defeq \sup_{\substack{0 \leqslant g \leqslant f \\ g \text{ ступенч.}}}
									{\int_E{g \d\mu}}
\]
\end{definition}

\begin{theorem}(Свойства интеграла измеримой функции)
	\enewline

	\begin{itemize}
		\item Если функция ступенчатая, то интеграл совпадает с интегралом, определенным
			для ступенчатых функций.
		\item $0 \leqslant \int{f} \leqslant +\infty$.
		\item $0 \leqslant g \leqslant f$, $g$ ступенчатая, $f$ измеримая, тогда
			$\int{g} \leqslant \int{f}$.
		\item $0 \leqslant g \leqslant f$, $f$, $g$ измеримы, тогда $\int{g} \leqslant \int{f}$.
	\end{itemize}
\end{theorem}

\begin{definition}
	Пусть $f$ --- измеримая функция $X$, причем хотя бы один из интегралов срезок конечен.
	Для такой функции определим интеграл:
\[
	\int_X{f \d\mu} \defeq \int_X{f_+ \d\mu} - \int_X{f_- \d\mu}
\]
\end{definition}

\begin{definition}
	Определим интеграл по измеримому множеству:
\[
	\int_E{f \d\mu} \defeq \int_X{f \cdot \chi_{E} \d\mu}
\]
\end{definition}

\begin{definition}
	Назовем функцию \textit{суммируемой}, если интегралы её срезок конечны.
\end{definition}

\begin{theorem}(Свойства интеграла)
	\enewline

	\begin{itemize}
		\item Измеримая $f \geqslant 0 \Lra$ интеграл совпадает с предыдущим определением.
		\item $f$ суммируема $\Llra \int{|f|} < +\infty$. 
	\end{itemize}
\end{theorem}
