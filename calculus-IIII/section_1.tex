\chapter{Интеграл}

\section{Определение интеграла}

\textit{Общий контекст: $\langle X, \cA, \mu \rangle$ --- пространство с мерой}

\begin{definition}
	Введем обозначение
\[
	\cL^0(X) = \{\, f \colon X \to \Rbar \mid f \text{ измерима и п.в. конечна} \,\}
.\]
\end{definition}

\begin{definition}
	Пусть $0 \leqslant f \colon X \to \Rbar$ --- ступенчатая функция, то есть
\[
	f = \sum_{fin} {\l_k \chi_{E_k}}
.\]
	Причем все $E_k$ измеримы. Интеграл такой функции определим следующим образом:
\[
	\int_X{f \d\mu} \defeq \sum_{k}{\l_k \mu{E_k}}
.\]
\end{definition}

\begin{definition}
	Аналогично определим интеграл по измеримому множеству:
\[
	\int_E{f} \defeq \sum_k{\l_k \mu{E \cap E_k}}
.\]
\end{definition}

\begin{theorem}(Свойства интеграла ступенчатой функции)
	\enewline

	\begin{itemize}
		\item Интеграл не зависит от допустимого разбиения.
		\item $f \leqslant g \Lra \int{f} \leqslant \int{g}$.
	\end{itemize}
\end{theorem}

\begin{definition}
	Пусть $0 \leqslant f \colon X \to \Rbar$ измерима. Интеграл такой функции определим так:
\[
	\int_X{f \d\mu} \defeq \sup_{\substack{0 \leqslant g \leqslant f \\ g \text{ ступенч.}}}
									{\int_X{g \d\mu}}
.\]
\end{definition}

\begin{definition}
	Аналогично определим интеграл по измеримому множеству:
\[
	\int_E{f \d\mu} \defeq \sup_{\substack{0 \leqslant g \leqslant f \\ g \text{ ступенч.}}}
									{\int_E{g \d\mu}}
.\]
\end{definition}

\begin{theorem}(Свойства интеграла измеримой функции)
	\enewline

	\begin{itemize}
		\item Если функция ступенчатая, то интеграл совпадает с интегралом, определенным
			для ступенчатых функций.
		\item $0 \leqslant \int{f} \leqslant +\infty$.
		\item $0 \leqslant g \leqslant f$, $g$ ступенчатая, $f$ измеримая, тогда
			$\int{g} \leqslant \int{f}$.
		\item $0 \leqslant g \leqslant f$, $f$, $g$ измеримы, тогда $\int{g} \leqslant \int{f}$.
	\end{itemize}
\end{theorem}

\begin{definition}
	Пусть $f$ --- измеримая функция $X$, причем хотя бы один из интегралов срезок конечен.
	Для такой функции определим интеграл:
\[
	\int_X{f \d\mu} \defeq \int_X{f_+ \d\mu} - \int_X{f_- \d\mu}
.\]
\end{definition}

\begin{definition}
	Определим интеграл по измеримому множеству:
\[
	\int_E{f \d\mu} \defeq \int_X{f \cdot \chi_{E} \d\mu}
.\]
\end{definition}

\begin{definition}
	Назовем функцию \textit{суммируемой}, если интегралы её срезок конечны.
\end{definition}

\begin{theorem}(Свойства интеграла)
	\enewline

	\begin{itemize}
		\item Измеримая $f \geqslant 0 \Lra$ интеграл совпадает с предыдущим определением.
		\item $f$ суммируема $\Llra \int{|f|} < +\infty$. 
        \item Интеграл монотонен по функции, то есть для измеримых $f$, $g$ верно:
\[
    f \leqslant g \Lra \int_{E}{f \d\mu} \leqslant \int_{E}{g \d\mu}
.\] 
        \item $\displaystyle \int_E{1 ~\d\mu} = \mu(E)$, $\displaystyle \int_E{0 \d\mu} = 0$.
        \item Пусть $\mu(E) = 0$, $f$ измерима. Тогда
\[
    \int_E{f} = 0
.\] 
        \item $\displaystyle \int{-f} = -\int{f}$, $\displaystyle 
               \forall c > 0~ \int{c \cdot f} = c \cdot \int{f}$.
        \item Пусть $\displaystyle \exists \int_E{f}$, Тогда
\[
    \left|{\int_E{f}}\right| \leqslant {\int_E{|f|}}
.\] 
        \item Пусть $f$ измерима на $E$, $\mu(E) < +\infty$, 
            $\forall x \in E~ A \leqslant f(x) \leqslant B$,
            тогда
\[
    A \cdot \mu(E) \leqslant \int_E{f \d\mu} \leqslant B \cdot \mu(E)
.\] 
	\end{itemize}
\end{theorem}

\begin{lemma}
    Пусть $A = \bigsqcup_i{A_i}$, $A, A_i \in \cA$, $g \colon X \to \Rbar$, $g \geqslant 0$, 
    ступенчата. Тогда
\[
    \int_A{g \d\mu} = \sum_i{\int_{A_i}{g \d\mu}}
.\] 
\end{lemma}

\begin{theorem}
    Пусть $A = \bigsqcup_i{A_i}$, $A, A_i \in \cA$, $f \colon X \to \Rbar$, $f \geqslant 0$,
    измерима на $A$. Тогда
\[
    \int_A{f \d\mu} = \sum_i{\int_{A_i}{f \d\mu}}
.\] 
\end{theorem}

\begin{corollary}
    Пусть $f \colon X \to \Rbar$, $f \geqslant 0$, измерима. Зададим отображение:
    \begin{align*}
        \nu \colon &\cA \to \Rbar_{\geqslant 0} \\ 
                   &E \mapsto \int_E{f \d\mu}
    \end{align*}
    Тогда $\nu$ -- мера.
\end{corollary}

\begin{lemma}
    Пусть $f$ суммируема, $g$ измерима, причем $f = g$ при почти всех $x$. Тогда
    $\displaystyle \int{f} = \int{g}$.
\end{lemma}
