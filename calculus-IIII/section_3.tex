\section{Произведение мер}

\textit{В этом разделе мы начинаем с того, что по двум пространствам
$\langle X, \cA, \mu \rangle$, $\langle Y, \cB, \nu \rangle$ строим пространство
$\langle X \times Y, \cA \times \cB, \mu \times \nu \rangle$.}

\begin{lemma}
    $\cA$, $\cB$ -- полукольца, тогда $\cA \times \cB$ -- полукольцо.
\end{lemma}

\begin{definition}
    $\cA$, $\cB$ -- полукольца, назовем тогда $\cA \times \cB$ \textit{полукольцом
    измеримых прямоугольников}. Заведем отображение:
\begin{align*}
    m_0 \colon &\cA \times \cB \to \Rbar \\
               &A \times B \mapsto \mu(A) \cdot \nu(B)
\end{align*}
\end{definition}

\begin{theorem}
    \enewline
    \begin{itemize}
        \item $m_0$ -- мера на полукольце $\cA \times \cB$.
        \item Если $\mu$, $\nu$ $\sigma$-конечны, тогда $m_0$ тоже $\sigma$-конечна.
    \end{itemize} 
\end{theorem}
\begin{proof}
    \enewline
    \begin{itemize}
        \item Достаточно доказать счетную аддитивность. Пусть $P = \bigsqcup{P_i}$, $P, P_i \in \cA \times \cB$
            $P = A \times B$, $P_i = A_i \times B_i$. Зметим, что верны утверждения:
\[
    \chi_P(x, y) = \sum_{i}{\chi_{P_i}(x, y)},~~
    \chi_A(x) \cdot \chi_B(y) = \sum_{i}{\chi_{A_i}(x) \cdot \chi_{B_i}(y)}
.\]            
            Проинтегрируем последнее равенство по мере $\nu$ в $Y$:
        \begin{align*}
            &\chi_A(x) \cdot \intl_B{\chi_B(y) \dd\nu} = \sum_{i}{\chi_A(x) \cdot \intl_B{\chi_B(y) \dd\nu}} \\
            &\chi_A(x) \cdot \nu{B} = \sum_{i}{\chi_A(x) \cdot \nu{B}}
        \end{align*}
            Интегрируя второй раз по переменной $x$ по мере $\mu$, получаем:
\[
    \mu{A} \nu{B} = \sum_{i}{\mu{A_i} \nu{B_i}}
.\]
        \item Пусть $X = \bigcup{X_i}$, $Y = \bigcup{Y_i}$, $\mu{X_k} < +\infty$, $\nu{Y_k} < +\infty$, 
            тогда 
        \begin{align*}
            X = \bigcup_{k, j}{X_k \times Y_j},~~ m_0(X_k \times Y_j) = \mu{X_k} \nu{Y_j} < +\infty
        \end{align*}
    \end{itemize}
\end{proof}

\begin{definition}

    Мы получили $\langle X \times Y, \cA \times \cB, m_0 \rangle$ -- пространство с
    мерой на полукольце. Продолжим её, пользуясь теоремой о продолжении,
    до $\sigma$-алгебры, которую будем обозначать $\cA \otimes \cB$. Результирующее пространство
    назовем \textit{произведением пространств с мерой}, а полученную меру -- \textit{произведением мер}.
\end{definition}

\begin{theorem}
    Произведение мер ассоциативно.
\end{theorem}

\begin{theorem}
    $\lambda_{m + n} = \lambda_{m} \times \lambda_{n}$.
\end{theorem}

\begin{definition}
    Пусть $C \subseteq X \times Y$. Тогда \textit{сечением} для произвольного $x \in X$
    назовем множество
\[
    C_x \defeq \{\,y \in Y \mid (x, y) \in C\,\}
.\] 
\end{definition}

\begin{remark}
    Для сечений верны формулы, связанные с операциями над множествами,
    подобные этой:
\[
    \left(\bigcup_{\a}{C_{\a}}\right)_x = \bigcup_{\a}{(C_{\a})_x}
.\] 
\end{remark}

\begin{theorem}(Принцип Кавальери)
    
    Пусть $\mu$, $\nu$ -- $\sigma$-конечные полные меры, $m = \mu \times \nu$, $C \in \cA \otimes \cB$,
    тогда
    \begin{enumerate}
        \item При почти всех $x$ $C_x \in \cB$.
        \item Отображение $x \mapsto \nu{C_x}$ измеримо на $X$.
        \item $\displaystyle m(C) = \intl_{X}{\nu{C_x} \dd\mu}$.
    \end{enumerate} 
\end{theorem}
\begin{proof}
    Пусть множество $D \subseteq \cA \times \cB$ -- элементы $X \times Y$, для которых 
    принцип Кавальери верен.
    \begin{itemize}
        \item Запасем какие-нибудь простые множества в $D$. А именно, $\forall A \in \cA, B \in \cB$
            верно, что $C = A \times B \in D$. Проверим это:
            \begin{enumerate}
                \item $C_x = B \in \cB$ или $\varnothing$, в обоих случаях измеримо.
                \item $x \mapsto \nu{C_x} = \nu{B} \chi_A(x)$ -- очевидно измерима.
                \item Вычислим интеграл:
\[
    \intl_X{\nu{C_x} \dd\mu} = \intl_X{\nu{B} \chi_A(x) \dd\mu} = 
    \intl_A{\nu{B} \dd\mu} = \mu{A} \nu{B} = m{C}
.\]
            \end{enumerate}
        \item Пусть теперь $E = \bigsqcup{E_i}$, $E_i \in D$. Тогда $E \in D$.
            \begin{enumerate}
                \item $E_x = \bigsqcup_i{(E_i)_x}$ -- измеримо почти везде, потому что
                    $(E_i)_x$ измеримы почти везде.
%:: NOTE all 
% Проверить, что это доказано ранее
% (измеримость произвольной суммы измеримых множеств
% или доказать здесь
                \item $x \mapsto \nu{E_x} = \sum_i{\nu{(E_i)_x}}$ -- измерима как сумма
                    измеримых функций.
                \item Считаем:
\[
    \intl_X{\nu{E_x} \dd\mu} = \intl_X{\sum_i{\nu{(E_i)_x}} \dd\mu} =
    \sum_i{\intl_X{\nu{(E_i)_x} \dd\mu}} = \sum_i{m(E_i)} = m{E}
.\]
            \end{enumerate}
        \item Пусть $E_i \in D$, $E_1 \supseteq \ldots $, $\bigcap{E_i} = E$, $m{E_i} < +\infty$.
            Тогда $E \in D$. Для начала заметим, что
\[
    \intl_X{\nu{(E_i)_x} \dd\mu} = m{E_i} < +\infty
.\]
            Поэтому почти везде $\nu{(E_i)_x} < +\infty$.
            \begin{enumerate}
                \item При почти всех $x$ одновременно измеримы все $(E_i)_x$, поэтому измеримо
                    множество $\bigcap{(E_i)_x} = E_x$.
                \item Пользуясь непрерывностью меры сверху получаем, что функция
                    $x \mapsto \nu{E_x} = \lim{\nu{(E_i)_x}}$ измерима как предел измеримых функций.
                \item Считаем:
\[
    \intl_X{\nu{E_x} \dd\mu} = \intl_X{\lim{\nu{(E_i)_x}} \dd\mu}
.\]
                По теореме Лебега, которую мы пока не знаем, можно вынести предел из под знака интеграла
                в случае, когда подынтегральное выражение можно мажорировать суммируемой функцией, 
                не зависящей от $i$:
\[
    \nu{(E_i)_x} \leqslant \nu{(E_1)_x}
.\]
                Последняя функция суммируема, поэтому
\[
    \intl_X{\nu{E_x} \dd\mu} = \intl_X{\lim{\nu{(E_i)_x}} \dd\mu} = \lim{\intl_X{\nu{(E_i)_x} \dd\mu}} =
    \lim{m(E_i)} = m{E}
.\]
                Последнее равенство верно в силу непрерывности меры $m$ сверху.
            \end{enumerate}
        \item Если $A_{i,j} \in \cA \times \cB$, то $\bigcap_j{\bigcup_i{A_{i,j}}} \in D$.
                Сделаем множества $A_{i, j}$ дизъюнктными (пользуясь стандартной техникой, мы останемся в полукольце),
                а затем сделаем множества $\bigsqcup_i{\hat{A}_{i, j}}$ убывающими, взяв
                в $\widetilde{A}_0 = \hat{A}_{i, j}, \ldots, \widetilde{A}_n = \bigsqcup_{i = 1}^n{\hat{A}_{i, j}}$.
        \item Покажем, что если $m{E} = 0$, то $E \in D$. Аппроксимируем $E$ сериями прямоугольников
                $P_{i, j}$ (из теоремы о стандартном продолжении меры): пусть $H = \bigcap_i{\bigcup_j{P_{i, j}}}$, 
                очевидно, что $E \subset H \in D$, $m{H} = 0$. Обладая этими знаниями, проверим, что $E \in D$:
            \begin{enumerate}
                \item Поскольку $H \in D$:
\[
    0 = m{H} = \intl_{X}{\nu{H_x} \dd\mu} \Lra \nu{H_x} = 0 \text{ при п.в. }x
.\]
                Пользуясь полнотой меры $\nu$ и тем фактом, что $E_x \subset H_x$, получаем, что $\nu{E_x} = 0$ при
                почти всех $x$.
                \item Отображение $x \mapsto \nu{E_x}$ измеримо как отображение, почти всюду равное нулю.
                \item Поскольку $\nu{E_x} = 0$ почти везде, очевидно, что $\intl_X{\nu{E_x} \dd\mu} = 0 = m{E}$.
            \end{enumerate}
        \item Покажем, что если $C \in \cA \otimes \cB$, $m{C} < +\infty$, то $C \in D$.
                $\exists e \colon~ m{e} = 0$, $H = \bigcap_j{\bigcup_i{P_{i, j}}}$, $H = C \setminus e$,
                $m{C} = m{H}$ (как в предыдущем пункте, из теоремы о стандартном продолжении меры).
            \begin{enumerate}
                \item $C_x = H_x \ e_x$ -- измеримо как разность измеримых множеств.
                \item $\nu{C_x} = \nu{H_x} - \nu{e_x}$ -- измерима как разность измеримых функций. 
                \item Считаем:
\[
    \intl_X{\nu{C_x} \dd\mu} = \intl_X{\nu{H_x} \dd\mu} = \intl_X{\nu{e_x} \dd\mu} =
    \intl_X{\nu{H_x} \dd\mu} = m{H} = m{C}
.\]
            \end{enumerate}
        \item Пусть, наконец, $C \in \cA \otimes \cB$. Пусть $X = \bigsqcup{X_i}, Y = \bigsqcup{Y_i}$  
            (пользуемся $\sigma$-конечностью мер), тогда:
\[
    C = \bigsqcup_{i, j}{C \cap (X_i \times Y_j)} \in D
.\]
    \end{itemize}
\end{proof}

\begin{corollary}
    Пусть $C \in \cA \otimes \cB$, $p_1(C) \in \cA$, тогда
\[
    m(C) = \intl_{p_1(C)}{\nu(C_x) \dd\mu}
.\] 
\end{corollary}
%:: NOTE all proof

\begin{corollary}
    Пусть $f \colon [a, b] \to \R$, $f \in C$, тогда
\[
    \intl_a^b{f(x) \dd x} = \intl_{[a, b]}{f \dd\lambda_1}
.\] 
\end{corollary}
%:: NOTE all proof

\begin{remark}
    Пусть $f \geqslant 0$, измерима, тогда
\[
    \lambda_2 \text{ПГ}(f, [a, b]) = \intl_{[a, b]}{f \dd\lambda_1}
.\] 
\end{remark}

\begin{definition}
    Пусть $f \colon X \times Y \to \Rbar$, $C \in X \times Y$. Зафиксируем $x \in X$
    и определим отображение:
\begin{align*}
    f_x \colon &C_x \to \Rbar \\
               &y \mapsto f(x, y)
\end{align*}
    Аналогично определим $f_y \colon C_y \to \Rbar$ для всех $y \in Y$.
\end{definition}

\begin{theorem}(Тонелли)
    
    Пусть $\mu$, $\nu$ -- $\sigma$-конечные полные меры, $m = \mu \times \nu$,
    $f \colon X \times Y \to \Rbar$, $f \geqslant 0$, \textbf{измерима} по мере $m$.
    Тогда
    \begin{itemize}
        \item При почти всех $x$ $f_x$ \textbf{измерима} на $Y$.
        \item Отображение $x \mapsto \f(x) = \int_{Y}{f(x, y) \dd\nu} 
        = \int_{Y}{f_x \dd\nu}$ \textbf{измеримо} на $X$.
        \item $\displaystyle \intl_{X \times Y}{f(x, y) \dd m} 
            = \intl_{X}{\left(\intl_{Y}{f(x, y) \dd\nu}\right) \dd\mu}$.
    \end{itemize} 
\end{theorem}
%:: NOTE all proof

\begin{theorem}(Фубини)
    
    Пусть $\mu$, $\nu$ -- $\sigma$-конечные полные меры, $m = \mu \times \nu$,
    $f \colon X \times Y \to \Rbar$, $f \geqslant 0$, \textbf{суммируема}.
    Тогда
    \begin{itemize}
        \item При почти всех $x$ $f_x$ \textbf{суммируема} на $Y$.
        \item Отображение $x \mapsto \f(x) = \int_{Y}{f(x, y) \dd\nu} 
            = \int_{Y}{f_x \dd\nu}$ \textbf{суммируемо} на $X$.
        \item $\displaystyle \intl_{X \times Y}{f(x, y) \dd m} 
            = \intl_{X}{\left(\intl_{Y}{f(x, y) \dd\nu}\right) \dd\mu}$.
    \end{itemize} 
\end{theorem}
%:: NOTE all proof

\begin{corollary}
    Если $p_1(C)$ измеримо, то
\[
    \intl_{C}{f \dd m} = \intl_{X \times Y}{f \chi_C \dd m} 
    = \intl_{X}{\left(\intl_{Y}{f \chi_C \dd\nu}\right) \dd\mu}
    = \intl_{p_1(C)}{\left(\intl_{C_x}{f \dd\nu}\right) \dd\mu}
.\]
\end{corollary}
%:: NOTE all proof

\begin{remark}
    Посмотрим на два вида сходимости: по мере и в смысле интеграла:
    \begin{itemize}
        \item[1.] $f_n \underset{\mu}{\Lra} f \Llra \mu X(|f_n - f| < \e) \to 0$.
        \item[2.] $\intl_{X}{|f_n - f| \dd\mu} \to 0$.
    \end{itemize} 
    Оказывается, верно $2 \Lra 1$, но без дополнительных требований неверно $1 \Lra 2$.
\end{remark}

\begin{theorem}(Лебега о мажорированной сходимости)
    
    Пусть $f_n$, $f$ измеримы и почти везде конечны, $f_n \underset{\mu}{\Lra} f$, 
    $\exists g\colon~$
    \begin{itemize}
        \item $\forall n~ |f_n| \leqslant g$ при почти всех $x$.
        \item $g$ суммируема на $X$.
    \end{itemize} 
    В такой ситуации $g$ называется \textit{Мажорантой}. Тогда
    \begin{itemize}
        \item $f_n$, $f$ суммируемы.
        \item $\displaystyle \intl_{X}{|f_n - f| \dd\mu} \to 0$.
    \end{itemize} 
\end{theorem}
%:: NOTE all proof

\begin{corollary}
    
    В условиях предыдущей теоремы верно
\[
    \intl_{X}{f_n \dd\mu} \xrightarrow[n \to +\infty]{} \intl_{X}{f \dd\mu}
.\] 
\end{corollary}
%:: NOTE all proof

\begin{theorem}
    Пусть $f_n$, $f$ измеримы и почти везде конечны, $f_n \to f$ почти везде,
    $\exists g\colon~$
    \begin{itemize}
        \item $\forall n~ |f_n| \leqslant g$.
        \item $g$ суммируема на $X$.
    \end{itemize}
    Тогда
    \begin{itemize}
        \item $f_n$, $f$ суммируемы.
        \item $\displaystyle \intl_{X}{|f_n - f| \dd\mu} \to 0$.
    \end{itemize} 
\end{theorem}
%:: NOTE all proof

\begin{corollary}
    
    В условиях предыдущей теоремы верно
\[
    \intl_{X}{f_n \dd\mu} \xrightarrow[n \to +\infty]{} \intl_{X}{f \dd\mu}
.\] 
\end{corollary}
%:: NOTE all proof

\begin{theorem}(Фату)

    Пусть $f_n \geqslant 0$, $f_n$ измеримы, $f_n \to f$ почти везде. Если
\[
    \exists c > 0\colon~ \forall n~ \intl_{X}{f_n \dd\mu} \leqslant c
.\] 
    то
\[
    \intl_{X}{f \dd\mu} \leqslant c
.\] 
\end{theorem}
%:: NOTE all proof

\begin{corollary}
    
    Теорема Фату верна и в случае $f_n \underset{\mu}{\Lra} f$.
\end{corollary}
%:: NOTE all proof

\begin{corollary}
    
    Пусть $f_n \geqslant 0$, $f_n$ измеримы, тогда
\[
    \intl_{X}{\varliminf{f_n} \dd\mu} \leqslant \varliminf{\intl_{X}{f_n \dd\mu}}
.\] 
\end{corollary}
%:: NOTE all proof
