\section{Произведение мер}

\textit{В этом разделе мы начинаем с того, что по двум пространствам
$\langle X, \cA, \mu \rangle$, $\langle Y, \cB, \nu \rangle$ строим пространство
$\langle X \times Y, \cA \times \cB, \mu \times \nu \rangle$.}

\begin{lemma}
    $\cA$, $\cB$ -- полукольца, тогда $\cA \times \cB$ -- полукольцо.
\end{lemma}

\begin{definition}
    $\cA$, $\cB$ -- полукольца, назовем тогда $\cA \times \cB$ \textit{полукольцом
    измеримых прямоугольников}. Заведем отображение:
\begin{align*}
    m_0 \colon &\cA \times \cB \to \Rbar \\
               &A \times B \mapsto \mu(A) \cdot \nu(B)
\end{align*}
\end{definition}

\begin{theorem}
    \enewline
    \begin{itemize}
        \item $m_0$ -- мера на полукольце $\cA \times \cB$.
        \item Если $\mu$, $\nu$ $\sigma$-конечны, тогда $m_0$ тоже $\sigma$-конечна.
    \end{itemize} 
\end{theorem}

\begin{definition}

    Мы получили $\langle X \times Y, \cA \times \cB, m_0 \rangle$ -- пространство с
    мерой на полукольце. Продолжим её, пользуясь теоремой о продолжении,
    до $\sigma$-алгебры, которую будем обозначать $ $. Результирующее пространство
    назовем \textit{произведением пространств с мерой}, а полученную меру -- \textit{произведением мер}.
\end{definition}

\begin{theorem}
    Произведение мер ассоциативно.
\end{theorem}

\begin{theorem}
    $\lambda_{m + n} = \lambda_{m} \times \lambda_{n}$.
\end{theorem}

\begin{definition}
    Пусть $C \subseteq X \times Y$. Тогда \textit{сечением} для произвольного $x \in X$
    назовем множество
\[
    C_x \defeq \{\,y \in Y \mid (x, y) \in C\,\}
.\] 
\end{definition}

\begin{remark}
    Для сечений верны формулы, связанные с операциями над множествами,
    подобные этой:
\[
    \left(\bigcup_{\a}{C_{\a}}\right)_x = \bigcup_{\a}{(C_{\a})_x}
.\] 
\end{remark}

