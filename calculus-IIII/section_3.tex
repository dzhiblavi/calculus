\section{Произведение мер}

\textit{В этом разделе мы начинаем с того, что по двум пространствам
$\langle X, \cA, \mu \rangle$, $\langle Y, \cB, \nu \rangle$ строим пространство
$\langle X \times Y, \cA \times \cB, \mu \times \nu \rangle$.}

\begin{lemma}
    $\cA$, $\cB$ -- полукольца, тогда $\cA \times \cB$ -- полукольцо.
\end{lemma}
%:: NOTE all proof

\begin{definition}
    $\cA$, $\cB$ -- полукольца, назовем тогда $\cA \times \cB$ \textit{полукольцом
    измеримых прямоугольников}. Заведем отображение:
\begin{align*}
    m_0 \colon &\cA \times \cB \to \Rbar \\
               &A \times B \mapsto \mu(A) \cdot \nu(B)
\end{align*}
\end{definition}

\begin{theorem}
    \enewline
    \begin{itemize}
        \item $m_0$ -- мера на полукольце $\cA \times \cB$.
        \item Если $\mu$, $\nu$ $\sigma$-конечны, тогда $m_0$ тоже $\sigma$-конечна.
    \end{itemize} 
\end{theorem}
%:: NOTE all proof

\begin{definition}

    Мы получили $\langle X \times Y, \cA \times \cB, m_0 \rangle$ -- пространство с
    мерой на полукольце. Продолжим её, пользуясь теоремой о продолжении,
    до $\sigma$-алгебры, которую будем обозначать $\cA \otimes \cB$. Результирующее пространство
    назовем \textit{произведением пространств с мерой}, а полученную меру -- \textit{произведением мер}.
\end{definition}

\begin{theorem}
    Произведение мер ассоциативно.
\end{theorem}
%:: NOTE all proof

\begin{theorem}
    $\lambda_{m + n} = \lambda_{m} \times \lambda_{n}$.
\end{theorem}
%:: NOTE all proof

\begin{definition}
    Пусть $C \subseteq X \times Y$. Тогда \textit{сечением} для произвольного $x \in X$
    назовем множество
\[
    C_x \defeq \{\,y \in Y \mid (x, y) \in C\,\}
.\] 
\end{definition}

\begin{remark}
    Для сечений верны формулы, связанные с операциями над множествами,
    подобные этой:
\[
    \left(\bigcup_{\a}{C_{\a}}\right)_x = \bigcup_{\a}{(C_{\a})_x}
.\] 
\end{remark}

\begin{theorem}(Принцип Кавальери)
    
    Пусть $\mu$, $\nu$ -- $\sigma$-конечные полные меры, $m = \mu \times \nu$, $C \in \cA \otimes \cB$,
    тогда
    \begin{itemize}
        \item При почти всех $x$ $C_x \in \cB$.
        \item Отображение $x \mapsto \nu(C_x)$ измеримо на $X$.
        \item $\displaystyle m(C) = \int_{X}{\nu(C_x) \dd\mu}$.
    \end{itemize} 
\end{theorem}
%:: NOTE all proof

\begin{corollary}
    Пусть $C \in \cA \otimes \cB$, $p_1(C) \in \cA$, тогда
\[
    m(C) = \intl_{p_1(C)}{\nu(C_x) \dd\mu}
.\] 
\end{corollary}
%:: NOTE all proof

\begin{corollary}
    Пусть $f \colon [a, b] \to \R$, $f \in C$, тогда
\[
    \intl_a^b{f(x) \dd x} = \intl_{[a, b]}{f \dd\lambda_1}
.\] 
\end{corollary}
%:: NOTE all proof

\begin{remark}
    Пусть $f \geqslant 0$, измерима, тогда
\[
    \lambda_2 \text{ПГ}(f, [a, b]) = \intl_{[a, b]}{f \dd\lambda_1}
.\] 
\end{remark}

\begin{definition}
    Пусть $f \colon X \times Y \to \Rbar$, $C \in X \times Y$. Зафиксируем $x \in X$
    и определим отображение:
\begin{align*}
    f_x \colon &C_x \to \Rbar \\
               &y \mapsto f(x, y)
\end{align*}
    Аналогично определим $f_y \colon C_y \to \Rbar$ для всех $y \in Y$.
\end{definition}

\begin{theorem}(Тонелли)
    
    Пусть $\mu$, $\nu$ -- $\sigma$-конечные полные меры, $m = \mu \times \nu$,
    $f \colon X \times Y \to \Rbar$, $f \geqslant 0$, \textbf{измерима} по мере $m$.
    Тогда
    \begin{itemize}
        \item При почти всех $x$ $f_x$ \textbf{измерима} на $Y$.
        \item Отображение $x \mapsto \f(x) = \int_{Y}{f(x, y) \dd\nu} 
        = \int_{Y}{f_x \dd\nu}$ \textbf{измеримо} на $X$.
        \item $\displaystyle \intl_{X \times Y}{f(x, y) \dd m} 
            = \intl_{X}{\left(\intl_{Y}{f(x, y) \dd\nu}\right) \dd\mu}$.
    \end{itemize} 
\end{theorem}
%:: NOTE all proof

\begin{theorem}(Фубини)
    
    Пусть $\mu$, $\nu$ -- $\sigma$-конечные полные меры, $m = \mu \times \nu$,
    $f \colon X \times Y \to \Rbar$, $f \geqslant 0$, \textbf{суммируема}.
    Тогда
    \begin{itemize}
        \item При почти всех $x$ $f_x$ \textbf{суммируема} на $Y$.
        \item Отображение $x \mapsto \f(x) = \int_{Y}{f(x, y) \dd\nu} 
            = \int_{Y}{f_x \dd\nu}$ \textbf{суммируемо} на $X$.
        \item $\displaystyle \intl_{X \times Y}{f(x, y) \dd m} 
            = \intl_{X}{\left(\intl_{Y}{f(x, y) \dd\nu}\right) \dd\mu}$.
    \end{itemize} 
\end{theorem}
%:: NOTE all proof

\begin{corollary}
    Если $p_1(C)$ измеримо, то
\[
    \intl_{C}{f \dd m} = \intl_{X \times Y}{f \chi_C \dd m} 
    = \intl_{X}{\left(\intl_{Y}{f \chi_C \dd\nu}\right) \dd\mu}
    = \intl_{p_1(C)}{\left(\intl_{C_x}{f \dd\nu}\right) \dd\mu}
.\]
\end{corollary}
%:: NOTE all proof

\begin{remark}
    Посмотрим на два вида сходимости: по мере и в смысле интеграла:
    \begin{itemize}
        \item[1.] $f_n \underset{\mu}{\Lra} f \Llra \mu X(|f_n - f| < \e) \to 0$.
        \item[2.] $\intl_{X}{|f_n - f| \dd\mu} \to 0$.
    \end{itemize} 
    Оказывается, верно $2 \Lra 1$, но без дополнительных требований неверно $1 \Lra 2$.
\end{remark}

\begin{theorem}(Лебега о мажорированной сходимости)
    
    Пусть $f_n$, $f$ измеримы и почти везде конечны, $f_n \underset{\mu}{\Lra} f$, 
    $\exists g\colon~$
    \begin{itemize}
        \item $\forall n~ |f_n| \leqslant g$ при почти всех $x$.
        \item $g$ суммируема на $X$.
    \end{itemize} 
    В такой ситуации $g$ называется \textit{Мажорантой}. Тогда
    \begin{itemize}
        \item $f_n$, $f$ суммируемы.
        \item $\displaystyle \intl_{X}{|f_n - f| \dd\mu} \to 0$.
    \end{itemize} 
\end{theorem}
%:: NOTE all proof

\begin{corollary}
    
    В условиях предыдущей теоремы верно
\[
    \intl_{X}{f_n \dd\mu} \xrightarrow[n \to +\infty]{} \intl_{X}{f \dd\mu}
.\] 
\end{corollary}
%:: NOTE all proof

\begin{theorem}
    Пусть $f_n$, $f$ измеримы и почти везде конечны, $f_n \to f$ почти везде,
    $\exists g\colon~$
    \begin{itemize}
        \item $\forall n~ |f_n| \leqslant g$.
        \item $g$ суммируема на $X$.
    \end{itemize}
    Тогда
    \begin{itemize}
        \item $f_n$, $f$ суммируемы.
        \item $\displaystyle \intl_{X}{|f_n - f| \dd\mu} \to 0$.
    \end{itemize} 
\end{theorem}
%:: NOTE all proof

\begin{corollary}
    
    В условиях предыдущей теоремы верно
\[
    \intl_{X}{f_n \dd\mu} \xrightarrow[n \to +\infty]{} \intl_{X}{f \dd\mu}
.\] 
\end{corollary}
%:: NOTE all proof

\begin{theorem}(Фату)

    Пусть $f_n \geqslant 0$, $f_n$ измеримы, $f_n \to f$ почти везде. Если
\[
    \exists c > 0\colon~ \forall n~ \intl_{X}{f_n \dd\mu} \leqslant c
.\] 
    то
\[
    \intl_{X}{f \dd\mu} \leqslant c
.\] 
\end{theorem}
%:: NOTE all proof

\begin{corollary}
    
    Теорема Фату верна и в случае $f_n \underset{\mu}{\Lra} f$.
\end{corollary}
%:: NOTE all proof

\begin{corollary}
    
    Пусть $f_n \geqslant 0$, $f_n$ измеримы, тогда
\[
    \intl_{X}{\varliminf{f_n} \dd\mu} \leqslant \varliminf{\intl_{X}{f_n \dd\mu}}
.\] 
\end{corollary}
%:: NOTE all proof
