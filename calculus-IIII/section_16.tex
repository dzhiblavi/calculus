\section{Преобразование Фурье}

\begin{definition}
    Пусть $f \in L^1(\Rm, \lambda_m)$. Тогда \textit{преобразованием
    Фурье $f$} называется функция
    \[
        \hat{f}(y) = \intl_{\Rm}{f(x) e^{-2i\pi\scp{x}{y}} \dd x}
    .\]
\end{definition}

\begin{theorem}(Свойства преобразования Фурье)

    Пусть $f \in L^1(\Rm, \lambda_m)$, $f_h(x) = f(x - h)$. Тогда
    \begin{enumerate}
        \item $\hat{f} \in C(\Rm)$.
        \item $|\hat{f}(y)| \leqslant \norm{f}_1$.
        \item $\hat{f}_h(y) = e^{-2i\pi \scp{y}{h}} \hat{f}(y)$.
        \item Пусть $g(x) = f(ax), a \neq 0$, тогда $\hat{g}(y) = 
            \frac{1}{|a|^m}\hat{f}(\frac{y}{a})$.
        \item $\hat{f}(y) \xrightarrow[|y| \to +\infty]{} 0$.
    \end{enumerate}
\end{theorem}
\begin{proof}
    \enewline
    \begin{enumerate}
        \item Пусть
            \[
                g(x) = f(x) e^{-2i\pi \scp{x}{y}}
            .\]
            Очевидно, что $g(x) \in C(\Rm)$. Если мы покажем, что
            $\forall y_0~ g \in L_{loc}(y_0)$, то по теореме Лебега
            мы докажем требуемое:
            \[
                |f(x) e^{-2i\pi\scp{x}{y}}| \leqslant |f(x)|
            .\]
            $f(x) \in L^1$, поэтому суммируема.
        \item
            \[
                |\hat{f}(y)| = \abs*{\intl_{\Rm}{f(x) e^{-2i\pi \scp{y}{x}} \dd x}}
                \leqslant \intl_{\Rm}{\abs*{f(x)} \dd x} = \norm{f}_1
            .\]
        \item Очевидно из замены переменной.
        \item \begin{align*}
                \hat{g}(y) = \intl_{\Rm}{f(ax) e^{-2i\pi\scp{y}{x}} \dd x} =
                \frac{1}{|a|^m} \intl_{\Rm}{f(x) e^{-2i\pi\scp{y}{\frac{x}{a}}} \dd x} =
                \frac{1}{|a|^m} \hat{f}\parens*{\frac{y}{a}}
            .\end{align*} 
        \item Теорема Римана-Лебега. \ref{rhi_leb} 
    \end{enumerate}
\end{proof}

\begin{definition}
    Пусть $f, g \in L^1(\Rm, \lambda_m)$. \textit{Свёрткой функций $f$ и $g$}
    называется
    \[
        (f \ast g)(x) \defeq \intl_{\Rm}{f(x - u) g(u) \dd u}
    .\]
\end{definition}

\begin{remark}
    Корректность определения и тот факт, что
    \[
        \norm{f \ast g}_1 \leqslant \norm{f}_1 \norm{g}_1
    \]
    доказываются аналогично свертке функций из $L_p$.
\end{remark}

\begin{theorem}
    Пусть $f, g \in L^1(\Rm)$. Тогда
    \begin{itemize}
        \item $(f \ast g)^{\wedge} = \hat{f} \cdot \hat{g}$.
        \item $\intl_{\Rm}{\hat{f}(y) g(y) \dd y} = \intl_{\Rm}{f(y) \hat{g}(y) \dd y}$.
    \end{itemize}
\end{theorem}
\begin{proof}
    \enewline
    \begin{itemize}
        \item \begin{align*}
                &(f \ast g)^\wedge(y) = \intl_{\Rm}{\parens*{
                \intl_{\Rm}{f(x - u) g(u) \dd u}} e^{-2i\pi\scp{y}{x}} \dd x} \\
                &\underset{\text{Фубини}}{=}
                \intl_{\Rm}{f(x - u) e^{-2i\pi\scp{y}{x - u}} \dd x}
                \intl_{\Rm}{g(u) e^{-2i\pi\scp{y}{u}} \dd x} =
                \hat{f}(y) \hat{g}(y)
            .\end{align*}
        \item \begin{align*}
                \intl_{\Rm}{\parens*{\intl_{\Rm}{f(x) e^{-2i\pi\scp{y}{x}} \dd x}}
                g(y) \dd y} \underset{\text{Фубини}}{=}
                \intl_{\Rm}{f(x) \parens*{\intl_{\Rm}{g(y) e^{-2i\pi\scp{y}{x}}
                \dd y}} \dd x} = \intl_{\Rm}{f(x) \hat{g}(x) \dd x}
            .\end{align*}
            Переход, связанный с теоремой Фубини выполнен потому, что
            функция
            \[
                f(x)g(y) e^{-2i\pi\scp{y}{x}}
            \]
            суммируема на $\Rm \times \Rm$:
            \[
                \intl_{\Rm}{\intl_{\Rm}{|f(x) g(y)|}} \leqslant \intl_{\Rm}{f}
                \cdot \intl_{\Rm}{g} \leqslant \norm{f}_1 \cdot \norm{g}_1 < +\infty
            .\]
    \end{itemize}
\end{proof}

\begin{lemma}
    Пусть $f \in L^1(\Rm)$, дифференцируема, причем $\pderv{f}{x_m}$
    непрерывны и суммруемы на $\Rm$. Тогда для почти всех
    $u = (x_1, \ldots x_{m - 1}) \in \R^{m - 1}$
    \[
        \lim_{t \to \pm \infty}{f(u, t)} = 0
    .\]
\end{lemma}
\begin{proof}
    По теореме Фубини производные суммируемы по последнему аргументу
    при почти всех $u$. Тогда
    \[
        f(u, t) - f(u, 0) = \intl_0^t{\pderv{f}{x_m}(u, \tau) \dd \tau}
    .\]
    Из этого сразу следует существование конечного предела
    \[
        f(u, t) \xrightarrow[t \to +\infty]{} f_0
    .\]
    Поскольку $f \in L^1(\Rm)$, она суммируема по последнему аргументу (опять же
    по теореме Фубини). При этом она имеет предел на бесконечности. Поэтому этот
    предел не может отличаться от нуля:
    \[
        f(u, t) \xrightarrow[t \to +\infty]{} 0
    .\]
\end{proof}

\begin{theorem}(Преобразование Фурье и дифференцирование)
   
    Пусть $f \in L^1(\Rm)$, $g = \pderv{f}{x_k}$.
    \begin{itemize}
        \item Если $g$ существует, непрерывна и суммируема на $\Rm$,
            то $\hat{g}(y) = 2i\pi y_k \hat{f}(y)$.
        \item Если $|x|f(x)$ суммируема, то $\hat{f} \in C^1(\Rm)$, причем
            $\pderv{\hat{f}}{y_k}(y) = -2i\pi (x_k f(x))^\wedge$.
    \end{itemize}
\end{theorem}
\begin{proof}
    \enewline
    \begin{itemize}
        \item Пусть $k = m, u = (x_1, \ldots x_{m - 1})$. Тогда
            \begin{align*}
                \intl_{-\infty}^\infty{g(u, t) e^{-2i\pi y_m t} \dd t} =
                \underbrace{f(u, t) e^{-2i\pi y_m t}\big|_{-\infty}^{+\infty}}_{= 0}
                +\intl_{-\infty}^\infty{2i\pi y_m f(u, t) e^{-2i\pi y_m t} \dd t}
            .\end{align*}
            Интегрируя равенство по остальному $\R^{m - 1}$, получаем требуемое:
            \[
                \hat{g}(y) = \intl_{\Rm}{g(x) e^{-2i\pi \scp{y}{x}} \dd x} =
                \intl_{\R^{m - 1}}{\parens*{\intl_{-\infty}^{\infty}{g(u, t) 
                        e^{-2i\pi y_m t}} \dd t} e^{-2i\pi \sum_{i = 1}^{m - 1}
                {y_i u_i}} \dd u} = 2i\pi y_k \hat{f}(y)
            .\]
        \item Для начала вслепую продифференцируем:
            \[
                \hat{f}(y) = \intl_{\Rm}{f(x) e^{-2i\pi \scp{y}{x}} \dd x} \Lra
                \pderv{\hat{f}}{y_k}(y) = -2i\pi \intl_{\Rm}{x_k f(x) e^{-2i\pi
                \scp{y}{x}} \dd x} = -2i\pi (x_k f(x))^\wedge
            .\]
            Теперь чтобы обосновать корректность осталось показать $L_{loc}$:
            \[
                |2i\pi x_k f(x) e^{-2i\pi \scp{y}{x}}| \leqslant \norm{x} f(x)
            .\]
    \end{itemize}
\end{proof}

\begin{lemma}
    При $k > m$ сходится интеграл
    \[
        \intl_{\Rm \setminus B(0, R)}{\frac{1}{|x|^k} \dd x}
    .\]
\end{lemma}
\begin{proof}
    \begin{align*}
        \intl_{\Rm \setminus B(0, R)}{\frac{1}{|x|^k} \dd x} =
        \intl_{R}^\infty{\intl_0^\pi{\ldots \intl_0^{2\pi}{r^{-k} r^{m - 1}
        \cdot \sin{\f_1}^{m - 2} \ldots 1 \dd \f^{m - 1}}}} =
        C \cdot \intl_{R}^\infty{r^{-k + m - 1} \dd r}
    .\end{align*}
\end{proof}

\begin{corollary}
    \enewline
    \begin{itemize}
        \item $f \in L^1(\Rm)$, финитная, тогда $\hat{f} \in C^\infty(\Rm)$.
        \item $f \in C_0^\infty$, тогда $\forall p > 0~ |y|^p \hat{f}(y)$
            суммируема в $\Rm$.
    \end{itemize}
\end{corollary}
\begin{proof}
    \enewline
    \begin{itemize}
        \item Для финитной функции верно, что $|x|^k f(x)$ суммируема для любого $k$.
        \item По первому пункту теоремы верно, что для любого $k$:
            \[
                \parens*{\pderv{f}{x_k}}^\wedge = 2i\pi y_k \hat{f}(y)
            .\]
            Более того, нетрудно понять, что в мультииндексной записи верно:
            \[
                \parens*{\hderv{\a}{f}{x^\a}}^\wedge = (2i\pi)^{|\a|} y^\a
                \hat{f}(y)
            .\]
            Поскольку преобразование Фурье ограничено:
            \[
                \begin{cases}
                    |y_1|^n |\hat{f}(y)| < C \\
                    \vdots \\
                    |y_m|^n |\hat{f}(y)| < C \\
                    |\hat{f}(y)| < C
                \end{cases} \Lra \hat{f}(y) < \frac{(m + 1) C}{1 + |y_1|^n +
                \ldots + |y_m|^n}
            .\]
            Заметим, что
            \[
                |y_1| + \ldots + |y_n| \geqslant |y| \Lra \max_{i}{|y_i|^n}
                \geqslant \abs*{\frac{|y|}{m}}^n
            .\]
            Тогда при получаем
            \[
                |y|^p |\hat{f}(y)| < \frac{(m + 1) C |y|^p}{1 + |y_1|^n +
                \ldots + |y_m|^n} \leqslant \frac{(m + 1) C |y|^p}{1 + \parens*{
                \frac{|y|}{m}}^n}
            .\]
            Что при достаточно большом $n$, а именно, при $n - p > m$, по лемме
            дает требуемое.
    \end{itemize}
\end{proof}

