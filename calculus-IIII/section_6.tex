\section{Поверхностные интегралы}

\begin{definition}
    Пусть $M$ -- простое гладкое двумерное многообразие в $\R^3$,
    $\f \colon \cO \subseteq \R^2 \to \R^3$ -- параметризация $M$,
    тогда $E \subseteq M$ называется \textit{измеримым}, если
    $\f^{-1}{E} \in \mathfrak{M}^2$.
\end{definition}

\begin{definition}
    Введем обозначение:
\[
    \cA_M \defeq \{ E \subseteq M \mid E \text{ измеримо} \}
.\] 
\end{definition}

\begin{remark}
    $\cA_M$ -- $\sigma$-алгебра.
\end{remark}

\begin{definition}
    На $\cA_M$ заведем меру:
    \begin{align*}
        S \colon &\cA_M \to \Rbar \\
                 &E \mapsto \iintl_{\f^{-1}(E)}{\norm{\f'_u \times \f'_v} \dd u \d v}
                 .
    \end{align*}
\end{definition}

\begin{remark}
    Замкнутые, открытые, компактные $E \subset M$ измеримы.
\end{remark}

\begin{lemma}
    $S$ не зависит от выбора параметризации.
\end{lemma}

\begin{definition}
    $f \colon M \to \Rbar$ измерима по мере $S$, если $f \circ \f$ измерима на $\cO$ по мере
    $\lambda$.
\end{definition}

\begin{definition}
    Пусть $M$ -- простое гладкое двумерное многообразие, $\f$ -- его параметризация,
    $0 \leqslant f \colon M \to \Rbar$ измерима по $S$, тогда
    \textit{поверхностным интегралом I рода} назовем интеграл
\[
    \iintl_{M}{f \dd S}
.\] 
    Или развернуто, пользуясь теоремой об интегрировании по взвешенному образу меры:
\[
    \iintl_{M}{f \dd S} = \iintl_{\f^{-1}(M)}{f(x(u, v), y(u, v), z(u, v)) \cdot
    \norm{\f'_u \times \f'_v} \dd u \d v}
.\] 
\end{definition}

\begin{definition}
    $M \subseteq \R^3$ назовем \textit{кусочно-гладким многообразием в $\R^3$},
    если $M$ представляется в виде конечного дизъюнктного объединения объектов вида
    \begin{itemize}
        \item простое гладкое двумерное многообразие.
        \item простое гладкое одномерное многообразие (носитель гладкого пути).
        \item точка.
    \end{itemize} 
\end{definition}

\begin{definition}
    Мера $S$ на кусочно-гладком многобразии $E = \bigsqcup_{i}{M_i}$ вычисляется следующим образом:
\[
    S(E) = \sum_{i}{S(E \cap M_i)}
.\] 
\end{definition}



