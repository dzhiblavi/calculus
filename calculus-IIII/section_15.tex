\section{Интегрирование рядов Фурье}

\begin{lemma}
    \enewline
    \begin{itemize}
        \item $D_n(t) = \frac{\sin{nt}}{\pi t} + \frac{1}{2\pi}
            \parens*{\cos{nt} + \sin{nt} \cdot h(t)}$, $|h(t)| \leqslant 1$
            на $[-\pi, \pi]$.
    \item $\forall |x| < 2\pi~ \abs*{\intl_0^x{D_n(t) \dd t}} < 2$.
    \end{itemize}
\end{lemma}
\begin{proof}
    \enewline
    \begin{itemize}
        \item \begin{align*}
                D_n(t) = \frac{1}{2\pi}\frac{\sin{(n + \frac{1}{2}) t}}{
                \sin{\frac{t}{2}}} = \frac{\sin{nt}}{2\pi \tan{\frac{t}{2}}}
                + \frac{1}{2\pi}\cos{nt} =
                \frac{\sin{nt}}{\pi t} + \frac{1}{2\pi} \parens*{\underbrace{\cos{nt} +
                        \sin{nt} \parens*{\frac{1}{\tan{\frac{t}{2}}} -
                \frac{1}{\frac{t}{2}}}}_{h(t)}}
            .\end{align*}
            Очевидно, что $h(t)$ убывает. Поэтому
            \[
                h(t) \leqslant h(\pi) = \frac{2}{\pi} < 1
            .\]
        \item \begin{align*}
                \abs*{\intl_0^x{D_n(t) \dd t} - \intl_0^x{\frac{\sin{nt}}{\pi t}
                        \dd t}} = \abs*{\frac{1}{2\pi} \intl_0^x{(\cos{nt} + \sin{nt}
                ~h(t)) \dd t}} \leqslant \frac{1}{2\pi} \intl_0^x{2 \dd t} \leqslant 1
            .\end{align*}
            Для $x \in (0, \pi)$ имеем оценку
            \begin{align*}
                \intl_0^x{\frac{\sin{nt}}{\pi t}} = \intl_0^{nx}{\frac
                {\sin{v}}{\pi v} \dd v} < \intl_0^{\pi}
                {\frac{\sin{v}}{\pi v}} = 1
            .\end{align*}
            Для $x \in [\pi, 2\pi]$:
            \begin{align*}
                \intl_0^x{D_n(t) \dd t} = \intl_0^{2\pi} - \intl_x^{2\pi} =
                1 - \intl_0^{2\pi - x}{D_n(t) \dd t} \in [-1, 2)
            .\end{align*}
    \end{itemize}
\end{proof}

\begin{theorem}(Об интегрировании ряда Фурье)

    Пусть $f \in L_1$. Тогда $\forall a, b \in \R$
    \[
        \intl_a^b{f \dd x} = \sum_{k \in \bZ}{c_k(f) \intl_a^b{e^{ikx} \dd x}}
    .\]
\end{theorem}
\begin{proof}
    Достаточно проверить утверждение для $-\pi < a < b < \pi$. В остальных случаях
    отрезок можно разбить на периоды и всё получится. Пусть $\chi = \chi_{[a, b]}$.
    \begin{align*}
        \sum_{k = -n}^n{c_k(f) \underbrace{\intl_a^b{e^{ikx} \dd x}}_{2 \pi 
            c_{-k}(\chi)}} 
        &= \sum_{k = -n}^n{\frac{1}{2\pi} 
            \intl_{-\pi}^\pi{f(x) e^{-ikx}
                2 \pi c_{-k}(\chi) \dd x}} = \intl_{-\pi}^\pi{\parens*{f(x) 
        \sum_{k = -n}^n{c_{-k}(\chi) e^{-ikx}}} \dd x} \\
        &= \intl_{-\pi}^\pi{f(x) \underbrace{S_n(\chi, x)}_{\to \chi} \dd x} 
        \xrightarrow[n \to +\infty]{} \intl_a^b{f(x) \dd x}
    .\end{align*}
    Чтобы предельный переход сработал, надо его обосновать. Воспользуемся
    теоремой Лебега о мажорированной сходимости. Достаточно показать,
    что $S_n$ мажорируется суммируемой функцией:
    \begin{align*}
        |S_n(\chi, x)| 
        &= \abs*{\intl_{-\pi}^\pi{\chi(t) D_n(x - t) \dd t}} =
        \abs*{\intl_a^b{D_n(x - t) \dd t}} = \abs*{-\intl_{x - a}^{x - b}{
        D_n(\tau) \dd \tau}} \\
        &= \abs*{\intl_0^{x - a}{D_n(t) \dd t} -
        \intl_0^{x - b}{D_n(t) \dd t}} \leqslant 4
    .\end{align*}
\end{proof}

\begin{remark}
    Не предполагается, что ряд Фурье функции $f$ сходится.
\end{remark}

\begin{remark}
    Мы в том числе проверили тот факт, что $S_n(\chi)$ равномерно ограничены.
\end{remark}

\begin{remark}
    Если $f \in \widetilde{C}^1[-\pi, \pi]$, то $S_n(f)$ равномерно ограничены.
\end{remark}
\begin{proof}
    Пусть
    \[
        H_n(x) = \intl_0^x{D_n(t) \dd t} \leqslant 2
    \]
    Тогда
    \begin{align*}
        |S_n(f, x)| = \abs*{\intl_{-\pi}^\pi{f(x - t) D_n(t) \dd t}} =
        \abs*{f(x - t) H_n(t) + \intl_{-\pi}^\pi{f'(x - t) H_n(t) \dd t}}
        \leqslant C
    .\end{align*} 
\end{proof}

\begin{remark}
    $i(c_k(f) - c_{-k}(f)) = b_k(f)$.
\end{remark}
\begin{proof}
    \[
        -i(c_k(f) - c_{-k}(f)) = \frac{1}{\pi} \intl_{-\pi}^\pi{f(x)
        \frac{e^{-ikx} - e^{ikx}}{2i} \dd x} = -b_k(f) 
    .\]
\end{proof}

\begin{remark}
    \[
        \intl_{-\pi}^\pi{\intl_0^u{e^{ikx} \dd x} \dd u} = -\frac{2\pi}{ik}
    .\]
\end{remark}
\begin{proof}
    \[
        \intl_0^u{e^{ikx} \dd x} = \frac{1}{ik}e^{ikx}\big|_{x = 0}^{x = u}
        = \frac{1}{ik}(e^{iku} - 1)
    .\]
    \[
        \frac{1}{ik}\intl_{-\pi}^\pi{(e^{iku} - 1) \dd u} = -\frac{2\pi}{ik}
    .\]
\end{proof}

\begin{corollary}
    $f \in L_1 \Lra \sum{\frac{b_n(f)}{n}}$ сходится.
\end{corollary}
\begin{proof}
    Пусть $u \in (-\pi, \pi)$, $\chi = \chi_{[0, u]}$. Заметим, что
    \[
        \sum_{k = -N}^N{c_k(f) \intl_0^u{e^{ikx} \dd x}} \xrightarrow[N \to +\infty]{}
        \intl_0^u{f(x) \dd x}
    .\]
    Кроме того, имеет место суммируемая мажоранта:
    \[
        \abs*{\sum_{k = -N}^N{c_k(f) \intl_0^u{e^{ikx} \dd x}}} \leqslant
        \intl_{-\pi}^\pi{|f(x)| \underbrace{|S_n(\chi, x)|}_{\leqslant 4} \dd x}
        \leqslant 4 \norm{f}_1
    .\]
    Теперь выполнены условия теоремы Лебега о предельном переходе под знаком
    интеграла, то есть верно
    \begin{align*}
        \intl_{-\pi}^\pi{\intl_0^u{f(x) \dd x} \dd u} 
        &= \intl_{-\pi}^\pi{
            \parens*{\lim_{N \to +\infty}{\sum_{k = -N}^N{c_k(f) \intl_0^u{e^{ikx}
        \dd x}}}} \dd u} 
        = \lim_{N \to +\infty}{\intl_{-\pi}^\pi{\sum_{k = -N}^N{
        c_k(f) \intl_0^u{e^{ikx} \dd x}} \dd u}} \\
        &= \lim_{N \to +\infty}{\sum_{k = -N}^N{c_k(f) \intl_{-\pi}^\pi{
                \intl_0^u{e^{ikx}\dd x}}}} = \sum_{k \in \bZ}{c_k(f) \intl_{-\pi}^\pi{
        \intl_0^u{e^{ikx} \dd x}}}
    .\end{align*}
    Мы получили сходимость суммы
    \[
        \sum_{k \in \bZ}{c_k(f) \intl_{-\pi}^\pi{\intl_0^u{e^{ikx} \dd x}}}
    .\]
    Далее
    \begin{align*}
        \sum_{k \in \bZ}{c_k(f) \intl_{-\pi}^\pi{\intl_0^u{e^{ikx} \dd x}}} =
        -\frac{2\pi}{i} \sum_{k \in \bZ}{\frac{c_k(f)}{k}} =
        -2\pi \sum_{k \in \bN}{\frac{1}{ik} (c_k(f) - c_{-k}(f))}
        = C \sum_{k \in \bN}{\frac{b_k(f)}{k}}
    .\end{align*}
\end{proof}

\begin{definition}
    \textit{$f_n$ сходится к $f$ в смысле обобщённых функций}, если
    \[
        \forall h \in C^\infty~ \intl_{-\pi}^\pi{f_n(x) h(x) \dd x}
        \xrightarrow[n \to +\infty]{} \intl_{-\pi}^\pi{f(x) h(x) \dd x}
    .\]
\end{definition}

\begin{lemma}
    Пусть $f \in L_1$, тогда $S_n(f, x)$ сходится к $f(x)$ в смысле обобщённых
    функций.
\end{lemma}
\begin{proof}
    Пусть $f \in L_1$ , $h \in C^\infty \subset L_\infty$. Тогда свёртка
    $f \ast h$ непрерывна. Кроме того, она гладкая: теорема Лейбница, позволяющая
    дифференцировать под знаком интеграла, поставляет непрерывность производной.
    Проверим:
    \[
        (f \ast h)'_x(x) = \intl_{-\pi}^\pi{f(t) h'_x(x - t) \dd t}
    .\]
    Чтобы переход был корректным, надо проверить $L_{loc}$:
    \[
        |f(t) h'(x - t)| \leqslant C |f(t)|
    .\]
    Обозначим $\underline{h}(x) = h(-x)$. Тогда
    \begin{align*}
        \intl_{-\pi}^\pi{S_n(f, x) h(x) \dd x} 
        &= \sum_{k = -n}^n{c_k(f) \intl_{-\pi}^\pi{
        e^{ikx} h(x) \dd x}} = \sum_{k = -n}^n{2\pi c_k(f) c_k(\underline{h})} =
        \sum_{k = -n}^n{c_k(f \ast \underline{h})} \\
        &= \sum_{k = -n}^n{c_k(f \ast \underline{h}) e^{ikx}\big|_{x = 0}}
        \xrightarrow[n \to +\infty]{\text{Дини}} (f \ast \underline{h})(0) =
        \intl_{-\pi}^\pi{f(t) h(t) \dd t}
    .\end{align*}
\end{proof}

\begin{remark}
    \[
        c_n(f) = \frac{1}{2\pi} \hat{f}\parens*{\frac{n}{2\pi}}
    .\]
\end{remark}
\begin{proof}
    Считаем, что $f = 0$ на дополнении $[-\pi, \pi]$.
    \[
        c_n(f) = \frac{1}{2\pi} \intl_{-\pi}^\pi{f(x) e^{-inx} \dd x} =
        \frac{1}{2\pi} \intl_{\R}{f(x) e^{-inx} \dd x} = \frac{1}{2\pi}
        {\hat{f}\parens*{\frac{n}{2\pi}}}
    .\]
\end{proof}

\begin{lemma}(Обобщённое равенство Парсиваля)
    
    Пусть $g$ -- измеримая, ограниченная, периодическая функция, $f \in L_1$,
    $\exists C\colon~ \forall n, x~ |S_n(g, x)| \leqslant C$. Тогда
    \[
        \intl_{-\pi}^\pi{f(x) \overline{g(x)} \dd x} =
        2\pi \sum{c_n(f) \overline{c_n(g)}} =
        \frac{1}{2\pi} \sum{\hat{f}\parens*{\frac{n}{2\pi}}
        \hat{g}\parens*{\frac{n}{2\pi}}}
    .\]
\end{lemma}
\begin{proof}
    Второе равенство следует из первого и предыдущего замечания.
    Поскольку $g$ ограничено, $g \in L_2$.
    \begin{align*}
        \intl_{-\pi}^\pi{f(x) \overline{S_n(g, x)} \dd x} = \intl_{-\pi}^\pi{
        f(x) \sum_{k = -n}^n{\overline{c_k(g)} e^{-ikx}}} =
        \sum_{k = -n}^n{\overline{c_k(g)} \cdot 2\pi c_k(f)}
        \xrightarrow[n \to +\infty]{} 2\pi \sum_{k \in \bZ}{c_k(f) \overline{c_k(g)}}
    .\end{align*}
    Осталось проверить, что
    \[
        \intl_{-\pi}^\pi{f(x) \overline{S_n(g, x)} \dd x} \xrightarrow[n \to +\infty]{}
        \intl_{-\pi}^\pi{f(x) \overline{g(x)} \dd x}
    .\]
    Из теоремы Фейера известно, что
    \[
        S_n(g) \xrightarrow[n \to +\infty]{L_2} g
    .\]
    Отсюда следует сходимость по мере:
    \[
        S_n(g) \Lra g 
    .\]
    Поэтому
    \[
        \overline{S_n(g)} \Lra \overline{g}
    .\]
    Но тогда
    \[
        f(x) \overline{S_n(g)} \Lra f(x) \overline{g(x)}
    .\]
    Теперь если мы промажорируем суммируемой функцией подынтегральное выражение,
    то по теореме Лебега получим требуемое:
    \[
        |f(x) \overline{S_n(g)}| \leqslant C|f(x)|
    .\]
\end{proof}

\begin{remark}
    Пусть $g(x) = e^{itx}$. Тогда
    \[
        c_n(g) = \frac{\sin{(t - n) \pi}}{(t - n) \pi}
    .\]
\end{remark}
\begin{proof}
    \[
        c_n(g) = \frac{1}{2\pi} \intl_{-\pi}^\pi{e^{ix (t - n)}} =
        \frac{1}{i(t - n)} e^{ix (t - n)}\big|_{x = -\pi}^{x = \pi} =
        \frac{\sin{(t - n) \pi}}{(t - n) \pi}
    .\]
\end{proof}

\begin{theorem}(Котельников, формула отсчёта)
    
    Пусть $f \in L^1(\R)$, $f = 0$ на дополнении $[-\pi, \pi]$, $F(t) =
    \hat{f}\parens*{\frac{t}{2\pi}}$. Тогда
    \[
        F(t) = \frac{\sin{\pi t}}{\pi} \sum_{n \in \bZ}{\frac{(-1)^n F(n)}{t - n}}
    .\]
\end{theorem}
\begin{proof}
    Обозначим $f_0 = f\big|_{[-\pi, \pi]}$. Тогда
    \begin{align*}
        \hat{f}\parens*{\frac{t}{2\pi}} = \intl_{-\pi}^\pi{f_0(x) e^{-inx} \dd x}
    .\end{align*}
    Для функций $f_0$, $g(x) = e^{itx}$ выполнены условия обобщенного равенства
    Парсиваля. Тогда
    \[
        \hat{f}\parens*{\frac{t}{2\pi}} = \sum_{k \in \bZ}{2\pi c_k(f) \overline{c_k(g)}}
        = \sum_{n \in \bZ}{\hat{f}\parens*{\frac{n}{2\pi}} \cdot \frac{\sin{(t - n) \pi}}
        {\pi (t - n)}} = \frac{\sin{\pi t}}{n} \sum_{n \in \bZ}{\frac{(-1)^n F(n)}{t - n}}
    .\]
\end{proof}

\begin{theorem}(О равносходимости)
   
    Пусть $f \in L^1(\R)$, $f_0 \in \widetilde{L}_1$, $x \in \R$,
    $\exists U(x)\colon~ f = f_0$ на $U(x)$. Тогда сходимость ряда
    Фурье в $x$ функции $f_0$ эквивалентна сходимости интеграла Фурье
    $f$ в $x$, причем в случае сходимости выполняется
    \[
        \intl_{\R}{\hat{f}(y) e^{2i\pi yx} \dd y} = \sum_{n \in \bZ}{c_n(f) e^{inx}}
    .\]
\end{theorem}
\begin{proof}
    Пусть $U(x) = (x - \delta, x + \delta),~ \pi > \delta > 0$. Проверим, что
    \[
        \abs*{I_A(f, x) - S_{[2\pi A]}(f, x)} \xrightarrow[A \to +\infty]{} 0
    .\]
    Для этого вспомним, что
    \begin{align*}
        &I_A(f, x) = \intl_{-\delta}^\delta{f(x - t) \frac{\sin{2 \pi tA}}{\pi t} \dd t}
        + o(1),~ A \to +\infty, \\ 
        &S_n(f, x) = \intl_{-\delta}^\delta{f(x - t) \frac{\sin{nt}}{\pi t} \dd t}
        + o(1),~ A \to +\infty
    .\end{align*}
    В случае $2\pi A \in \bN$ доказывать нечего. Пусть $n = [2\pi A]$. Тогда
    \[
        |I_A - S_n| \to |I_A - I_{\frac{n}{2\pi}}| \leqslant 
        \parens*{\intl_{A - \frac{1}{2\pi}}^{A}
        + \intl_{-A}^{-A + \frac{1}{2\pi}}} |\hat{f}| \dd y \leqslant
        2 \frac{1}{2\pi} \cdot \max_{|y| > A - \frac{1}{2\pi}}{|\hat{f}(y)|}
        \xrightarrow[A \to +\infty]{} 0
    .\]
\end{proof}

\begin{theorem}(Признак Абеля-Дирихле равномерной сходимости интеграла)
    
    Пусть $I = \intl_A^\infty{f(x, t) g(x, t) \dd x}$, $f \in C((a, +\infty) \times
    [c, d])$, $g$ дифференцируема по $x$, $g'_x$ непрерывна. Тогда $I$ сходится
    равномерно при выполнении одного из условий:
    \begin{itemize}
        \item $\exists C\colon~ \forall B > a~ \forall t \in [c, d]$
            \[
                \abs*{\intl_a^B{f(x, t) \dd x}} \leqslant C
            .\]
            Кроме этого, $g_t$ монотонна при всех $t \in [c, d]$, причем
            \[
                g(x, t) \rcon_{x \to +\infty} 0
            .\]
        \item Равномерно сходится интеграл
            \[
                \intl_a^\infty{f(x, t) \dd x}
            .\]
            Кроме этого, $g_t$ монотонна при всех $t \in [c, d]$ и равномерно
            ограничена.
    \end{itemize}
\end{theorem}

\begin{definition}
    \textit{Вариацией функции $f$ на $[a, b]$} называется
    \[
        \var_a^b{f} = \sup_{\tau}{\sum_{i \geqslant 1}{|f(t_i) - f(t_{i - 1})|}}
    .\]
\end{definition}

\begin{remark}
    В случае непрерывных функций
    \[
        \var_a^b{f} = l(f) = \intl_a^b{|f'(x)| \dd x}
    .\]
\end{remark}

\begin{definition}
    \textit{Функцией ограниченной вариации} называется функция, для которой
    \[
        \var_a^b{f} < +\infty
    .\]
\end{definition}

\begin{remark}
    Если $f$ -- функция ограниченной вариации, то найдутся монотонные $g_1, g_2$
    такие, что $f = g_1 - g_2$.
\end{remark}

\begin{theorem}(Признак Дирихле-Жордана)

    Пусть $f \in L^1(\R)$ либо $f \in \widetilde{L}_1$, $f$ -- функция ограниченной
    вариации в $U(x)$, $x \in \R$. Тогда ряд Фурье и интеграл Фурье соответствующих
    функций в точке $x$ сходится к
    \[
        \frac{1}{2}(f(x + 0) + f(x - 0))
    .\]
\end{theorem}
\begin{proof}
    Проверим утверждение для случая рядов Фурье. Пусть $U(x) = (x - \delta, x + \delta)$.
    Тогда
    \[
        S_n(f, x) = \intl_{-\delta}^\delta{f(x - t) \frac{\sin{nt}}{\pi t} \dd t}
        + o(1) = \underbrace{\intl_0^\delta{\f(t) \frac{\sin{nt}}{\pi t} \dd t}}_{
        I_n} + o(1),~ n \to +\infty
    ,\]
    где $\f(u) = f(x - u) + f(x + u)$ -- тоже функция ограниченной вариации.
    Не умаляя общности, учитывая предыдущее замечание, считаем, что
    $\f$ -- монотонно убывающая неотрицательная функция. Обозначим
    \[
        \Phi(u) = \f(u) \chi_{[0, \delta]},~ u \in \R
    .\]
    Тогда
    \begin{align*}
        I_n 
        &= \intl_0^\infty{\Phi(u) \frac{\sin{nu}}{\pi u} \dd u} =
        \underbrace{\intl_0^\infty{\Phi\parens*{\frac{t}{n}} \frac{\sin{nt}}{\pi t} \dd t}}_
        {I^\ast}
        \xrightarrow[n \to +\infty]{} \intl_0^\infty{\Phi(0 + 0) \frac{\sin{nt}}
        {\pi t} \dd t} = \frac{1}{2} \Phi(0 + 0) \\
        &= \frac{1}{2}(f(x + 0) + f(x - 0))
    .\end{align*}
    Осталось обосновать предельный переход. На допредельном уровне верно
    \[
        \intl_0^A{\Phi\parens*{\frac{t}{n}} \frac{\sin{nt}}{\pi t} \dd t}
        \to \intl_0^A{\Phi(0 + 0) \frac{\sin{nt}}{\pi t}}
    \]
    потому, что есть суммируемая мажоранта. Кроме того, в $U(0)$ по признаку
    Дирихле равномерно сходится интеграл $I^\ast$, что завершает обоснование
    предельного перехода.
\end{proof}

