\section{Преобразование Фурье}

\begin{definition}
    Пусть $f \in L^1(\Rm, \lambda_m)$. Тогда \textit{преобразованием
    Фурье $f$} называется функция
    \[
        \hat{f}(y) = \intl_{\Rm}{f(x) e^{-2i\pi\scp{x}{y}} \dd x}
    .\]
\end{definition}

\begin{theorem}(Свойства преобразования Фурье)

    Пусть $f \in L^1(\Rm, \lambda_m)$, $f_h(x) = f(x - h)$. Тогда
    \begin{enumerate}
        \item $\hat{f} \in C(\Rm)$.
        \item $\hat{f}_h(y) = e^{-2i\pi \scp{y}{h}} \hat{f}(y)$.
        \item Пусть $g(x) = f(ax), a \neq 0$, тогда $\hat{g}(y) = 
            \frac{1}{|a|^m}\hat{f}(\frac{y}{a})$.
        \item $\hat{f}(y) \xrightarrow[|y| \to +\infty]{} 0$.
    \end{enumerate}
\end{theorem}
\begin{proof}
    \enewline
    \begin{enumerate}
        \item Пусть
            \[
                g(x) = f(x) e^{-2i\pi \scp{x}{y}}
            .\]
            Очевидно, что $g(x) \in C(\Rm)$. Если мы покажем, что
            $\forall y_0~ g \in L_{loc}(y_0)$, то по теореме Лебега
            мы докажем требуемое:
            \[
                |f(x) e^{-2i\pi\scp{x}{y}}| \leqslant |f(x)|
            .\]
            $f(x) \in L^1$, поэтому суммируема.
        \item Очевидно из замены переменной.
        \item \begin{align*}
                \hat{g}(y) = \intl_{\Rm}{f(ax) e^{-2i\pi\scp{y}{x}} \dd x} =
                \frac{1}{|a|^m} \intl_{\Rm}{f(x) e^{-2i\pi\scp{y}{\frac{x}{a}}} \dd x} =
                \frac{1}{|a|^m} \hat{f}\parens*{\frac{y}{a}}
            .\end{align*} 
        \item Теорема Римана-Лебега. \ref{rhi_leb} 
    \end{enumerate}
\end{proof}

\begin{definition}
    Пусть $f, g \in L^1(\Rm, \lambda_m)$. \textit{Свёрткой функций $f$ и $g$}
    называется
    \[
        (f \ast g)(x) \defeq \intl_{\Rm}{f(x - u) g(u) \dd u}
    .\]
\end{definition}

\begin{remark}
    Корректность определения и тот факт, что
    \[
        \norm{f \ast g}_1 \leqslant \norm{f}_1 \norm{g}_1
    \]
    доказываются аналогично свертке функций из $L_p$.
\end{remark}

\begin{theorem}
    Пусть $f, g \in L^1(\Rm)$. Тогда
    \begin{itemize}
        \item $(f \ast g)^{\wedge} = \hat{f} \cdot \hat{g}$.
        \item $\intl_{\Rm}{\hat{f}(y) g(y) \dd y} = \intl_{\Rm}{f(y) \hat{g}(y) \dd y}$.
    \end{itemize}
\end{theorem}
\begin{proof}
    \enewline
    \begin{itemize}
        \item \begin{align*}
                &(f \ast g)^\wedge(y) = \intl_{\Rm}{\parens*{
                \intl_{\Rm}{f(x - u) g(u) \dd u}} e^{-2i\pi\scp{y}{x}} \dd x} \\
                &\underset{\text{Фубини}}{=}
                \intl_{\Rm}{f(x - u) e^{-2i\pi\scp{y}{x - u}} \dd x}
                \intl_{\Rm}{g(u) e^{-2i\pi\scp{y}{u}} \dd x} =
                \hat{f}(y) \hat{g}(y)
            .\end{align*}
        \item \begin{align*}
                \intl_{\Rm}{\parens*{\intl_{\Rm}{f(x) e^{-2i\pi\scp{y}{x}} \dd x}}
                g(y) \dd y} \underset{\text{Фубини}}{=}
                \intl_{\Rm}{f(x) \parens*{\intl_{\Rm}{g(y) e^{-2i\pi\scp{y}{x}}
                \dd y}} \dd x} = \intl_{\Rm}{f(x) \hat{g}(x) \dd x}
            .\end{align*}
            Переход, связанный с теоремой Фубини выполнен потому, что
            функция
            \[
                f(x)g(y) e^{-2i\pi\scp{y}{x}}
            \]
            суммируема на $\Rm \times \Rm$:
            \[
                \intl_{\Rm}{\intl_{\Rm}{|f(x) g(y)|}} \leqslant \intl_{\Rm}{f}
                \cdot \intl_{\Rm}{g} \leqslant \norm{f}_1 \cdot \norm{g}_1 < +\infty
            .\]
    \end{itemize}
\end{proof}

