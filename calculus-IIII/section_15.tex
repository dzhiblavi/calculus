\section{Интегрирование рядов Фурье}

\begin{lemma}
    \enewline
    \begin{itemize}
        \item $D_n(t) = \frac{\sin{nt}}{\pi t} + \frac{1}{2\pi}
            \parens*{\cos{nt} + \sin{nt} \cdot h(t)}$, $|h(t)| \leqslant 1$
            на $[-\pi, \pi]$.
    \item $\forall |x| < 2\pi~ \abs*{\intl_0^x{D_n(t) \dd t}} < 2$.
    \end{itemize}
\end{lemma}
\begin{proof}
    \enewline
    \begin{itemize}
        \item \begin{align*}
                D_n(t) = \frac{1}{2\pi}\frac{\sin{(n + \frac{1}{2}) t}}{
                \sin{\frac{t}{2}}} = \frac{\sin{nt}}{2\pi \tan{\frac{t}{2}}}
                + \frac{1}{2\pi}\cos{nt} =
                \frac{\sin{nt}}{\pi t} + \frac{1}{2\pi} \parens*{\underbrace{\cos{nt} +
                        \sin{nt} \parens*{\frac{1}{\tan{\frac{t}{2}}} -
                \frac{1}{\frac{t}{2}}}}_{h(t)}}
            .\end{align*}
            Очевидно, что $h(t)$ убывает. Поэтому
            \[
                h(t) \leqslant h(\pi) = \frac{2}{\pi} < 1
            .\]
        \item \begin{align*}
                \abs*{\intl_0^x{D_n(t) \dd t} - \intl_0^x{\frac{\sin{nt}}{\pi t}
                        \dd t}} = \abs*{\frac{1}{2\pi} \intl_0^x{(\cos{nt} + \sin{nt}
                ~h(t)) \dd t}} \leqslant \frac{1}{2\pi} \intl_0^x{2 \dd t} \leqslant 1
            .\end{align*}
            Для $x \in (0, \pi)$ имеем оценку
            \begin{align*}
                \intl_0^x{\frac{\sin{nt}}{\pi t}} = \intl_0^{nx}{\frac
                {\sin{v}}{\pi v} \dd v} < \intl_0^{\pi}
                {\frac{\sin{v}}{\pi v}} = 1
            .\end{align*}
            Для $x \in [\pi, 2\pi]$:
            \begin{align*}
                \intl_0^x{D_n(t) \dd t} = \intl_0^{2\pi} - \intl_x^{2\pi} =
                1 - \intl_0^{2\pi - x}{D_n(t) \dd t} \in [-1, 2)
            .\end{align*}
    \end{itemize}
\end{proof}

\begin{theorem}(Об интегрировании ряда Фурье)

    Пусть $f \in L_1$. Тогда $\forall a, b \in \R$
    \[
        \intl_a^b{f \dd x} = \sum_{k \in \bZ}{c_k(f) \intl_a^b{e^{ikx} \dd x}}
    .\]
\end{theorem}
\begin{proof}
    Достаточно проверить утверждение для $-\pi < a < b < \pi$. В остальных случаях
    отрезок можно разбить на периоды и всё получится. Пусть $\chi = \chi_{[a, b]}$.
    \begin{align*}
        \sum_{k = -n}^n{c_k(f) \underbrace{\intl_a^b{e^{ikx} \dd x}}_{2 \pi 
            c_{-k}(\chi)}} 
        &= \sum_{k = -n}^n{\frac{1}{2\pi} 
            \intl_{-\pi}^\pi{f(x) e^{-ikx}
                2 \pi c_{-k}(\chi) \dd x}} = \intl_{-\pi}^\pi{\parens*{f(x) 
        \sum_{k = -n}^n{c_{-k}(\chi) e^{-ikx}}} \dd x} \\
        &= \intl_{-\pi}^\pi{f(x) \underbrace{S_n(\chi, x)}_{\to \chi} \dd x} 
        \xrightarrow[n \to +\infty]{} \intl_a^b{f(x) \dd x}
    .\end{align*}
    Чтобы предельный переход сработал, надо его обосновать. Воспользуемся
    теоремой Лебега о мажорированной сходимости. Достаточно показать,
    что $S_n$ мажорируется суммируемой функцией:
    \begin{align*}
        |S_n(\chi, x)| 
        &= \abs*{\intl_{-\pi}^\pi{\chi(t) D_n(x - t) \dd t}} =
        \abs*{\intl_a^b{D_n(x - t) \dd t}} = \abs*{-\intl_{x - a}^{x - b}{
        D_n(\tau) \dd \tau}} \\
        &= \abs*{\intl_0^{x - a}{D_n(t) \dd t} -
        \intl_0^{x - b}{D_n(t) \dd t}} \leqslant 4
    .\end{align*}
\end{proof}

\begin{remark}
    Не предполагается, что ряд Фурье функции $f$ сходится.
\end{remark}

\begin{remark}
    Мы в том числе проверили тот факт, что $S_n(\chi)$ равномерно ограничены.
\end{remark}

\begin{remark}
    Если $f \in \widetilde{C}^1[-\pi, \pi]$, то $S_n(f)$ равномерно ограничены.
\end{remark}
\begin{proof}
    Пусть
    \[
        H_n(x) = \intl_0^x{D_n(t) \dd t} \leqslant 2
    \]
    Тогда
    \begin{align*}
        |S_n(f, x)| = \abs*{\intl_{-\pi}^\pi{f(x - t) D_n(t) \dd t}} =
        \abs*{f(x - t) H_n(t) + \intl_{-\pi}^\pi{f'(x - t) H_n(t) \dd t}}
        \leqslant C
    .\end{align*} 
\end{proof}

