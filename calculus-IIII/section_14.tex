\section{Аппроксимативные единицы}

\begin{remark}
    Относительно $\ast$ не существует единичного элемента. Действительно,
    пусть $g$ -- этот единичный элемент. Тогда $c_k(f) = c_k(f \ast g) = 2\pi c_k(f) c_k(g)$,
    откуда $c_k(g) = \frac{1}{2 \pi}$, что не стремится к нулю, хотя из теоремы
    Римана-Лебега мы знаем обратное.
\end{remark}

\begin{definition}
    Будем обозначать
    \[
        E_\delta = [-\pi, \pi] \setminus (-\delta, \delta)
    .\]
\end{definition}

\begin{definition}
    Пусть $D \subseteq \R$, $h_0 \in \Rbar$ -- предельная точка $D$. \textit{
    Апроксимативной единицей (АЕ)} называется семейство функций, параметризованное
    $D$:
    \[
        \{\,K_h\,\}_{h \in D}
    ,\]
    если оно обладает свойствами:
    \begin{enumerate}
        \item $\forall h \in D~ K_h \in L_1$, $\intl_{-\pi}^\pi{K_h(t) \dd t} = 1$.
        \item $\exists M > 0\colon \forall h \in D~ \norm{K_h}_1 \leqslant M$.
        \item $\forall \delta \in (0, \pi)~ \intl_{E_\delta}{|K_h(t)| \dd t} 
            \xrightarrow[h \to h_0]{} 0$.
    \end{enumerate}
\end{definition}

\begin{remark}
    В случае $K_h \geqslant 0$ верно $1 \Lra 2$.
\end{remark}

\begin{definition}
    Рассмотрим условие $3'$:
    \[
        K_h \in L_\infty,~ \forall \delta \in (0, \pi)~ \esssup_{x \in E_\delta}
        {K_h(x)} \xrightarrow[h \to h_0]{} 0
    .\]
    Из этого условия очевидным образом следует условие $3$. Семейство,
    обладающее свойствами $1, 2, 3'$ называется \textit{усиленной
    аппроксимативной единицей (УАЕ)}.
\end{definition}

\begin{remark}
    Если $K_h$ -- (усиленная) аппроксимативная единица, то
    \[
        \frac{|K_h|}{\norm{K_h}_1}
    \]
    -- (усиленная) аппроксимативная единица.
\end{remark}
\begin{proof}
    Условия $1, 2$ очевидно выполняются. Проверим $3$, $3'$ доказывается аналогично.
    Нужно заметить, что
    \[
        1 = \abs*{\intl_{-\pi}^\pi{K_h(t) \dd t}} \leqslant \intl_{-\pi}^\pi{|K_h(t)| \dd t}
        = \norm{K_h}_1
    ,\]
    то есть
    \[
        \norm{K_h}_1 \geqslant 1 
    .\]
    Значит, $|K_h|$ делится на число, большее единицы, то есть уменьшается.
    Поэтому свойство всё ещё выполняется.
\end{proof}

\begin{theorem}(О свойствах аппроксимативной единицы)
    
    Пусть $K_h$ -- аппроксимативная единица. Тогда
    \begin{enumerate}
        \item $f \in \widetilde{C}[-\pi, \pi] \Lra f \ast K_h \rcon_{h -> h_0} f$.
        \item $f \in L_1 \Lra f \ast K_h \xrightarrow[h \to h_0]{L_1} f$.
        \item Пусть $K_h$ -- усиленная. Тогда если $f \in L_1$, $f$ непрерывна в
            $x$, то $f \ast K_h$ непрерывна в $x$, причем
            \[
                (f \ast K_h)(x) \xrightarrow[h \to h_0]{} f(x)
            .\]
    \end{enumerate}
\end{theorem}
\begin{proof}
    \enewline
    \begin{enumerate}
        \item[0.] Заметим, что поскольку $\intl_{-\pi}^\pi{K_h(t) \dd t} = 1$, то
            \[
                (f \ast K)(x) - f(x) = \intl_{-\pi}^\pi{(f(x - t) - f(x)) K_h(t) \dd t}
            .\]
        \item Раз $f$ непрерывна на компакте, она равномерна непрерывна на нём.
            Зафиксируем $\e > 0$. Из равномерной непрерывности имеем
            \[
                \exists \delta > 0~ \forall |t| < \delta~ \forall x~
                |f(x - t) - f(x)| < \frac{\e}{2M}
            .\]
            В таком случае,
            \[
                (f \ast K)(x) - f(x) = \intl_{-\delta}^\delta{} + \intl_{E_\delta} {}
                = I_1 + I_2
            .\]
            Оценим $I_1$:
            \[
                |I_1| \leqslant \intl_{-\delta}^\delta{|f(x - t) - f(x)| |K_h(t)| \dd t}
                \leqslant \frac{\e}{2M} \intl_{-\pi}^\pi{|K_h(t)| \dd t} = \frac{\e}{2}
            .\]
            Оценим $I_2$:
            \[
                |I_2| \leqslant \intl_{E_\delta}{|f(x - t) - f(x)| |K_h(t)| \dd t}
                \leqslant 2 \max_{[-\pi, \pi]}{f} \cdot \intl_{E_\delta}{|K_h(t)| \dd t}
                \xrightarrow[h \to h_0]{} 0
            .\]
            Проведенные оценки не зависят от $x$, поэтому сходимость равномерная
            сразу по всем $x \in [-\pi, \pi]$.
        \item Пусть
            \[
                g(t) = \intl_{-\pi}^\pi{|f(x + t) - f(x)| \dd x}
            .\]
            Эта функция непрерывна. Покажем это:
            \begin{align*}
                |g(t) - g(t_0)| 
                &\leqslant \intl_{-\pi}^\pi{|f(x + t) - f(x)|
                - |f(x + t_0) - f(x)| \dd t} \\
                &\leqslant
                \intl_{-\pi}^\pi{|f(x_0 + t) - f(x + t_0)| \dd t}
                \xrightarrow[t \to t_0]{} 0
            .\end{align*}
            Оценим норму:
            \begin{align*}
                \norm{f \ast K_h - f}_1 
                &= \intl_{-\pi}^\pi{\abs*{\intl_{-\pi}^\pi{
                (f(x - t) - f(x)) K_h(t) \dd t}} \dd x} \leqslant
                \intl_{-\pi}^\pi{\intl_{-\pi}^\pi{|f(x - t) - f(x)| |K_h(t)| \dd t \d x}} \\
                &= \intl_{-\pi}^\pi{|K_h(t)| g(-t) \dd t} = \norm{K_h}_1 \cdot
                \intl_{-\pi}^\pi{\frac{|K_h(t)|}{\norm{K_h}_1} g(-t) \dd t} \\
                &= \norm{K_h}_1 \cdot \intl_{-\pi}^\pi{g(0 - t) \frac{|K_h(t)|}{\norm{K_h}_1} \dd t}
                = \norm{K_h}_1 \parens*{g \ast \frac{|K_h|}{\norm{K_h}_1}}(0)
                \xrightarrow[h \to h_0]{(1)} \norm{K_h}_1 g(0) = 0
            .\end{align*}
        \item $f \ast K_h$ непрерывна по свойству свёртки. Из непрерывности $f$
            имеем:
            \[
                \forall \e > 0~ \exists \delta > 0\colon~ \forall |t| < \delta~
                |f(x - t) - f(x)| < \frac{\e}{2M}
            .\]
            Тогда как и в первом пунктуе оценим
            \[
                |(f \ast K_h)(x) - f(x)| \leqslant I_1 + I_2
            .\]
            $I_1$ оценим аналогично первому пункту.
            \begin{align*}
                |I_2| 
                &\leqslant \esssup_{E_\delta}{K_h} \cdot \intl_{E_\delta}{|f(x - t) - f(x)| \dd t}
                \leqslant \esssup_{E_\delta}{K_h} \intl_{E_\delta}{(|f(x - t)| + |f(x)|) \dd t} \\
                &\leqslant \esssup_{E_\delta}{K_h} \cdot 2 \pi |f(x)| \norm{f}_1 
                \xrightarrow[h \to h_0]{} 0
            .\end{align*}
    \end{enumerate}
\end{proof}

\begin{corollary}
    $f \in L_p \Lra f \ast K_h \xrightarrow[h \to h_0]{L_p} f$.
\end{corollary}
\begin{proof}
    \begin{align*}
        \norm{f \ast K_h - f}_p^p 
        &= \intl_{-\pi}^\pi{\abs*{\intl_{-\pi}^\pi{f(x - t)
                K_h(t) \dd t}}^p \dd x} \leqslant \intl_{-\pi}^\pi{\parens*{
                \intl_{-\pi}^\pi{|f(x - t) - f(x)| |K_h(t)|^{\frac{1}{p}} |K_h(t)|^
        {\frac{1}{q}} \dd t}}^p \dd x} \\
        &\underset{\text{Гёльдер}}{\leqslant}
        \intl_{-\pi}^\pi{\parens*{\intl_{-\pi}^\pi{|f(x - t) - f(x)|^p |K_h(t)|
        \dd t}} \norm{K_h}_1^{\frac{p}{q}} \dd x} \\
        &= \norm{K_h}_1^{\frac{p}{q}} \cdot \intl_{-\pi}^\pi{g(-t)
        \frac{|K_h(t)|}{\norm{K_h}_1} \dd t}
    .\end{align*}
    Далее как в предыдущей теореме. 
\end{proof}

\begin{remark}
    Последний пункт теоремы можно сформулировать по-другому:
    если $K_h$ -- четная функция, причем существуют и конечны пределы
    $f(x \pm 0)$, то
    \[
        (f \ast K_h)(x) \xrightarrow[h \to h_0]{} \frac{1}{2} (f(x - 0) + f(x + 0))
    .\]
\end{remark}

\begin{definition}
    \textit{Суммами Фейера} называются суммы вида
    \[
        \sigma_n(f, x) = \frac{1}{n + 1} \sum_{k = 0}^n{S_k(f, x)}
    .\]
\end{definition}

\begin{lemma}
    $\sigma_n(f, x) = \intl_{-\pi}^\pi{f(x + t) \Phi_n(t) \dd t} =
    \intl_{-\pi}^\pi{f(x - t) \Phi_n(t)} = f \ast \Phi_n$.
\end{lemma}
\begin{proof}
    Первое равенство очевидно. Второе следует из того факта, что $\Phi_n$ --
    четная функция.
\end{proof}

\begin{theorem}(Фейер)
    \enewline
    \begin{itemize}
        \item $f \in \widetilde{C}[-\pi, \pi] \Lra \sigma_n(f) \rcon_{n \to +\infty} f$.
        \item $1 \leqslant p < +\infty,~ f \in L_p \Lra \sigma_n(f) 
            \xrightarrow[n \to +\infty]{L_p} f$.
        \item $f \in L_1, f \in C(x_0) \Lra \sigma_n(f, x_0) \xrightarrow[n \to +\infty]{} f(x_0)$.
    \end{itemize}
\end{theorem}
\begin{proof}
    Утверждения теоремы -- в точности утверждения теоремы о свойствах аппроксимативной
    единицы (усиленной). Проверим, что $\Phi_n$ -- усиленная аппроксимативная единица.
    \begin{enumerate}
        \item $\Phi_n \in C \Lra \Phi_n \in L_\infty \subset L_1,~
            \intl_{-\pi}^\pi{\Phi_n(t) \dd t} = 1$.
        \item $\Phi_n \geqslant 0 \Lra $ из (1) следует (2).
        \item[3'.] $\Phi_n \in L_\infty \Lra \sup = \esssup$:
            \[
                \frac{1}{2 \pi (n + 1)} \cdot \sup_{E_\delta}{\frac{\sin^2
                {\frac{n + 1}{2} x}}{\sin^2{\frac{x}{2}}}} \leqslant
                \frac{1}{2 \pi (n + 1)} \frac{1}{\sin^2{\frac{\delta}{2}}}
                \xrightarrow[n \to +\infty]{} 0
            .\]
    \end{enumerate}
\end{proof}

\begin{corollary}
    Пусть $f \in L_1$, $f$ непрерывна в $x$. Тогда если ряд Фурье
    $f$ сходится в $x$, то он сходится к $f(x)$.
\end{corollary}
\begin{proof}
    $\sigma_n(f, x)$ в случае сходимости вычисляет сумму, совпадающую с обычной
    суммой:
    \[
        \xymatrix{
            \sigma_n(f, x) \ar[r] \ar[d]_{\text{Фейер}} & \text{Ряд Фурье} \\
            f(x) 
        }
    .\]
\end{proof}

\begin{corollary}
    $f \in L_1, \forall k~ a_k(f) = 0, b_k(f) = 0 \Lra f = 0$ при почти всех $x$.
\end{corollary} 
\begin{proof}
    $a_k(f), b_k(f) = 0$, поэтому $\sigma_n(f) = 0$, поэтому $f = 0$ из теоремы
    Фейера.
\end{proof}

\begin{corollary}
    Тригонометрическая система полна в $L_2$.
\end{corollary}
\begin{proof}
    Если какой-то элемент $h$ ортогонален всем элементам тригонометрической
    системы, то очевидно, что $a_k(h) = b_k(h) = 0$. Раз так, по предыдущему
    следствию $h = 0$ почти везде.
\end{proof}

\begin{corollary}
    Пусть $f \in L^1[0, \pi]$. Если ряд Фурье по синусам (косинусам) равен нулю,
    то $f = 0$ почти везде.
\end{corollary}

\begin{corollary}
    $f \in L_2 \Lra S_n(f, x) \xrightarrow[n \to +\infty]{L_2} f(x)$.
\end{corollary}
\begin{proof}
    Следует из полноты.
\end{proof}

\begin{corollary}
    Пусть $f, g \in L_2$. Тогда 
    \begin{itemize}
        \item $\intl_{-\pi}^\pi{f \overline{g}} = 2\pi \sum_{k \in \bZ}
            {c_k(f) \overline{c_k(g)}}$.
        \item $\intl_{-\pi}^\pi{|f|^2} = 2\pi \sum_{k \in \bZ}{|c_k(f)|^2}$.
        \item $\intl_{-\pi}^\pi{fg} = \pi\parens*{\frac{a_0(f)a_0(g)}{2}
            + \sum_{k \in \bN}{a_k(f) a_k(g) + b_k(f) b_k(g)}}$.
        \item $\intl_{-\pi}^\pi{|f|^2} = \pi\parens*{\frac{a_0(f)^2}{2}
            + \sum_{k \in \bN}{a_k(f)^2 + b_k(f)^2}}$.
    \end{itemize}
\end{corollary}
\begin{proof}
    Все равенства -- просто разложения по базису элемента Гильбертова
    пространства.
\end{proof}

\begin{corollary}
    Тригонометричекие полиномы плотны в $\widetilde{C}[-\pi, \pi]$, $L_p$ для
    $1 \leqslant p < +\infty$.
\end{corollary}
\begin{proof}
    Очевидное следствие теоремы Фейера.
\end{proof}

