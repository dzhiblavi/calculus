\section{Замена переменных в интеграле}

\begin{definition}
    
    Отображение $\Phi \colon X \to Y$ называется \textit{измеримым}, если
\[
    \forall B \in \cB~ \Phi^{-1}(B) \in \cA
.\] 
    Иначе говоря, прообраз измеримого множества измерим.
\end{definition}

\begin{lemma}
    $\Phi^{-1}(\cB)$ -- $\sigma$-алгебра.
\end{lemma}

\begin{definition}
    При фиксированном измеримом $\Phi \colon X \to Y$ отображение
    \begin{align*}
        \nu \colon &\cB \to \Rbar \\ 
                   &B \mapsto \mu(\Phi^{-1}(B))
    .\end{align*}
    назовем \textit{образом меры $\mu$ при отображении} $\Phi$.
\end{definition}

\begin{lemma}
    Образ меры при отображении является мерой.
\end{lemma}

\begin{remark}
    $\displaystyle \nu(B) = \intl_{\Phi^{-1}(B)}{1 \dd\mu}$
\end{remark}

\begin{lemma}
    Если функция $f \colon Y \to \Rbar$ измерима относительно $\cB$, 
    то $f \circ \Phi \colon X \to \Rbar$ измерима относительно $\cA$.
\end{lemma}
\begin{proof}
    \[
        X(f(\Phi(x)) < a) = \Phi^{-1}(Y(f < a)) \in \cA
    .\]
\end{proof}

\begin{definition}
    Пусть $\omega \colon X \to \Rbar$, $\omega \geqslant 0$, измерима.
    В этом контексте $\omega$ называется \textit{весовой функцией}. Тогда
    \textit{взвешенным образом меры $\mu$ с весом $\omega$} называется мера
    \begin{align*}
        \nu(B) = \intl_{\Phi^{-1}(B)}{\omega \dd\mu}    
    .\end{align*}
\end{definition}

\begin{theorem}(Об интегрировании по взвешенному образу меры)
    
    Пусть $\Phi \colon X \to Y$ -- измеримое отображение, $0 \leqslant 
    \omega \colon X \to \Rbar$ -- весовая функция, измерима на $X$, 
    $\nu$ -- взвешенный образ меры $\mu$ с весом $\omega$.
    Тогда для любой измеримой $f \colon Y \to \Rbar$ верно:
    \begin{itemize}
        \item $f \circ \Phi$ измерима на $X$.
        \item $\displaystyle \intl_{Y}{f \dd\nu} = \intl_{X}{(f \circ \Phi)~ \omega \dd\mu}$
    \end{itemize} 
\end{theorem}
\begin{proof}
    \enewline
    \begin{itemize}
        \item $(f \circ \Phi)$ измерима по предыдущей лемме. 
        \item Пусть $f = \chi_B$, $B \in \cB$. Тогда
            \[
                (f \circ \Phi)(x) = \begin{cases}
                    1, \Phi(x) \in B \\
                    0, \Phi(x) \notin B
                \end{cases} = \chi_{\Phi^{-1}(B)}
            .\]
            \[
                \intl_X{(f \circ \Phi)~\omega \dd\mu} = \intl_X{\chi_{\Phi^{-1}(B)}~\omega \dd\mu} =
                \intl_{\Phi^{-1}(B)}{\omega \dd\mu} = \nu{B} = \intl_Y{f \dd\nu}
            .\]
        \item Пусть теперь $f$ -- ступенчатая функция. Тогда:
            \[
                \intl_Y{f \dd\nu} = \sum_{k = 1}^n{\intl_Y{f_k \dd\nu}} = 
                \sum_{k = 1}^n{\intl_X{(f_k \circ \Phi)~\omega \dd\mu}} = \intl_X{(f \circ \Phi)~\omega \dd\mu}
            .\]
        \item Докажем утверждение для измеримой неотрицательной $f$. Аппроксимируем $f$ ступенчатыми
        $f_n$ так, чтобы $0 \leqslant f_n \leqslant f_{n+1}$. Тогда справедлива теорема Леви:
            \[
                \intl_X{(f \circ \Phi)~\omega \dd\mu} = \lim{\intl_X{(f_n \circ \Phi)~\omega \dd\mu}} =
                \lim{\intl_Y{f_n \dd\nu}} = \intl_Y{f \dd\nu}
            .\]
        \item Проверим утверждение для произвольной суммируемой функции $f$. Интеграл её модуля конечен,
            кроме того, срезки мажирируются модулем, поэтому их интегралы тоже конечны. Таким образом,
            мы имеем право писать все формулы, использующие срезки.
            \[
                \intl_X{(f \circ \Phi)~\omega \dd\mu} = \intl_X{(f_+ \circ \Phi)~\omega \dd\mu} -
                \intl_X{(f_- \circ \Phi)~\omega \dd\mu} = \intl_Y{f_+ \dd\nu} - \intl_Y{f_- \dd\nu} 
                = \intl_Y{f \dd\nu}
            .\]
    \end{itemize}
\end{proof}

\begin{corollary}
    Пусть $f$ суммируема на $Y$, $B \in \cB$, тогда в условиях теоремы:
\[
    \intl_{B}{f \dd\nu} = \intl_{\Phi^{-1}(B)}{(f \circ \Phi)~ \omega \dd\mu}
.\] 
\end{corollary}

\begin{definition}
    В ситуации $X = Y$, $\cA = \cB$, $\Phi = \id$, если $\omega \geqslant 0$ измерима,
    причем $\displaystyle \nu(B) = \intl_{B}{\omega \dd\mu}$, $\omega$ называется
    \textit{плотностью меры $\nu$ относительно меры $\mu$}.
    В таком случае
\[
    \intl_{X}{f \dd\nu} = \intl_{X}{f \omega \dd\mu}
.\] 
\end{definition}

\begin{theorem}(Критерий плотности)

    Пусть $\nu$ -- мера на $\cA$, $\omega \geqslant 0$ измерима, тогда
    верно, что $\omega$ -- плотность $\nu$ относительно $\mu$ тогда
    и только тогда, когда
\[
    \forall A \in \cA~ \inf_{A}{\omega} \cdot \mu(A) \leqslant \nu(A)
    \leqslant \sup_{A}{\omega} \cdot \mu(A)
.\] 
\end{theorem}
\begin{proof}
    \enewline
    \begin{itemize}
        \item[$(\Lra)$] $\forall A \in \cA$:
            \[
                \inf_A{\omega} \cdot \mu{A} \leqslant \intl_A{\omega \dd\mu} = \nu{A}
                \leqslant \sup_A{\omega} \cdot \mu{A}
            .\]
        \item[$(\Lla)$]
            \begin{itemize}
                \item Пусть $\omega > 0$, $q \in (0, 1)$, $A_j = A(q^j \leqslant \omega < q^{j - 1})$.
                    Тогда по посылке теоремы имеем:
                    \[
                        q^j \cdot \mu{A_j} \leqslant_1 \nu{A_j} \leqslant_2 q^{j - 1} \cdot \mu{A_j}
                    .\]
                    Кроме того, из простейших свойств интеграла слудует:
                    \[
                        q^j \cdot \mu{A_j} \leqslant_3 \intl_{A_j}{\omega \dd\mu} 
                        \leqslant_4 q^{j - 1} \cdot \mu{A_j}
                    .\]
                    Воспользуемся этим:
                    \[
                        q \intl_A{\omega \dd\mu} = q \sum_j{\intl_{A_j}{\omega \dd\mu}} \leqslant_4
                        \sum_j{q^j \cdot \mu{A_j}} \leqslant_1 \nu{A} \leqslant_2 \frac{1}{q}
                        \sum_j{q^j \cdot \mu{A_j}} \leqslant_3 \frac{1}{q} \intl_A{\omega \dd\mu}
                    .\]
                    Устремляя $q$ к единице, получаем требуемое.
                \item Пусть теперь $\omega \geqslant 0$, $e = X(\omega = 0)$. Тогда $\nu{e} = 0$ по условию.
                    Получается, что
                    \[
                        \nu{e} = 0 = \intl_e{w \dd\mu}
                    .\]
                    Проверим теперь утверждение для произвольного измеримого $A$:
                    \[
                        \nu{A} = \intl_{X \setminus e}{\omega \dd\mu} + 0 = \intl_{X \setminus e}{\omega \dd\mu}
                        + \intl_{e}{\omega \dd\mu} = \intl_X{\omega \dd\mu}
                    .\]
            \end{itemize}
    \end{itemize}
\end{proof}

\begin{lemma}
    Пусть $f$, $g$ -- суммируемые на $X$ функции, причем
    \[
        \forall A \in \cA~ \intl_{A}{f \dd\mu} = \intl_{A}{g \dd\mu}
    .\] 
    Тогда $f = g$ почти везде.
\end{lemma}
\begin{proof}
    Проверим, что $h = f - g = 0$ при почти всех $x$. По условию, $\forall A \in \cA~ 
    \intl_A{h \dd\mu} = \intl_A{(f - g) \dd\mu} = 0$. Положим $A_+ = X(h \geqslant 0)$, 
    $A_- = X(h < 0)$. Тогда $X = A_+ \sqcup A_-$ и 
    \[
        \intl_{A_+}{|h| \dd\mu} = \intl_{A_+}{h \dd\mu} = 0
    .\]
    \[
        \intl_{A_-}{|h| \dd\mu} = -\intl_{A_-}{h \dd\mu} = 0
    .\]
    Получается,
    \[
        \intl_{X}{|h| \dd\mu} = 0 + 0 = 0
    .\]
    Поэтому $h = 0$ за исключением, может быть, множества меры ноль.
\end{proof}

\begin{remark}
    Из последней леммы очевидно следует, что плотность одной меры относительно другой 
    определена однозначно с точностью до равенства почти везде. 
\end{remark}

\begin{lemma}(Об образе малых кубических ячеек)

    Пусть $\cO$ открыто, $\Phi \colon \cO \subseteq \Rm \to \Rm$, $\ea \in \cO$, $\Phi$
    дифференцируемо в $U(\ea)$, $\det{\Phi'(\ea)} \neq 0$, $c > |\det{\Phi'(\ea)}| > 0$.
    Тогда
    \[
        \exists \delta > 0~ \forall Q \text{ -- куб}, Q \subset B(\ea, \delta)~
        \lambda \Phi(Q) < c \cdot \lambda(Q)
    .\] 
\end{lemma}
\begin{proof}
    Обозначим $L = \Phi'(\ea)$ -- обратимый линейный оператор. Выпишем определение дифференцируемости:
    \[
        \Phi(\ex) - \Phi(\ea) = L(\ex - \ea) + o(\ex - \ea)
    .\]
    Умножим обе части на $L^{-1}$ и перенесем через знак равенства:
    \[
        \underbrace{\ea + L^{-1}(\Phi(\ex) - \Phi(\ea))}_{\Psi(\ex)} = \ex + o(\ex - \ea)
    .\]
    Получается, что $\Psi$ близок к тожественному отображению. Из последней
    формулы очевидно следует, что $\forall \e > 0~ \exists B_\e(\ea)\colon$
    \[
        |\Psi(\ex) - \ex| < \frac{\e}{\sqrt{m}} |\ex - \ea|
    .\]
    Пусть теперь $Q \subset B_\e(\ea)$ -- куб со стороной $h$. Тогда $\forall \ex \in Q$
    \[
        |\Psi(\ex) - \ex| < \e h,~ |\ex_i - \ea_i| < h
    .\]
    Попытаемся понять, что делает с $Q$ отображение $\Psi$. Пусть $x, y \in Q$. Тогда
    \begin{align*}
        |\Psi(\ex)_i - \Psi(\ey)_i| &\leqslant |\Psi(\ex)_i - \ex_i| + |\ex_i - \ey_i| + |\Psi(\ey)_i - \ey_i| \\ 
                                    &\leqslant |\Psi(\ex) - \ex| + h + |\Psi(\ey) - \ey| \\
                                    &\leqslant (1 + 2\e) h
    .\end{align*}
    Получается, что $\Psi(Q)$ содержится в кубе со стороной $(1 + 2\e) h$. Тогда:
    \[
        \lambda{\Psi(Q)} \leqslant (1 + 2\e)^m \lambda{Q}
    .\]
    Отображение $\Psi$ отличается от отображения $\Phi$ только однократным применением оператора $L$
    и парой сдвигов. Поэтому:
    \[
        \lambda{\Phi(Q)} = |\det{L}| \cdot \lambda{\Psi(Q)} \leqslant |\det{L}| \cdot (1 + 2\e)^m \lambda{Q}
    .\]
    Подберем такое $\e$ чтобы выполнялось неравенство
    \[
        |\det{L}| \cdot (1 + 2\e)^m < c
    .\]
    И выберем в качестве $\delta$ радиус шара $B_\e(\ea)$.
\end{proof}

\begin{lemma}
    Пусть $\cO$ открыто, $f \colon \cO \subseteq \Rm \to \R$, $f \in C(\cO)$, 
    $A \in \mathfrak{M}^m$, $A \subseteq Q$, $Q$ -- куб, причем $\Cl(Q) \subseteq \cO$. 
    Тогда
    \[
        \inf_{\substack{A \subset G \\ G \text{ открыто}}}
        {\left(\lambda(G) \cdot \sup_{G}{f}\right)} = \lambda(A) \cdot \sup_{A}{f}
    .\] 
\end{lemma}
%:: NOTE all proof

\begin{theorem}
    Пусть $\cO$ открыто, $\Phi \colon \cO \subseteq \Rm \to \Rm$ -- диффеоморфизм,
    $A \in \mathfrak{M}^m$, $A \subseteq \cO$, тогда
    \[
        \lambda \Phi(A) = \intl_{A}{|\det{\Phi'}| \dd\lambda_m}
    .\] 
\end{theorem}
\begin{proof}
    \enewline
    \begin{itemize}
        \item Пусть $\nu{A} = \lambda{\Phi(A)}$. Это мера, потому что $\Phi$ -- диффеоморфизм. Надо проверить,
            что $J_\Phi = |\det{\Phi'}|$ -- плотность меры $\nu$ относительно меры $\lambda$. Для этого
            будем проверять критерий плотности: $\forall A \in \cA$
            \[
                \inf_A{J_\Phi} \cdot \lambda{A} \leqslant \nu{A} \leqslant \sup_A{J_\Phi} \cdot \lambda{A}
            .\]
        \item Проверим второе неравенство. Пусть $Q$ -- кучибеская ячейка, причем $\overline{Q} \subseteq \cO$.
            Предположим противное:
            \[
                \lambda{Q} \cdot \sup_Q{J_\Phi} < \nu{Q}
            .\]
            Возьмем $c > \sup_Q{J_\Phi}$ такое, чтобы
            \[
                \lambda{Q} \cdot c < \nu{Q}
            .\]
            Запустим процесс половинного деления. На каждом шаге найдется часть $Q_i$, для которой
            верно
            \[
                \lambda{Q_i} \cdot c < \nu{Q_i}
            .\]
            Это верно потому, что иначе по аддитивности меры не было бы выполнено аналогичное неравенство
            на предыдущем шаге. По теореме Кантора
            \[
                \exists! \ea \in \bigcap_{i = 1}^{+\infty}{\overline{Q_n}}
            .\]
            То есть, мы имеем кубы с центром в точке $\ea$ со сколь угодно малой стороной. Это
            противоречит лемме об образе малых кубических ячеек, которая для достаточно малых кубов
            устанавливает неравенство
            \[
                \nu{Q} = \lambda{\Phi(Q)} > c \cdot \lambda{Q}
            .\]
        \item Пусть теперь $A$ -- открытое множетсво. Представим его в виде дизъюнктного объединения
            кубических ячеек $Q_i \colon~ \overline{Q_i} \subseteq \cO$. Тогда
            \[
                \nu{A} = \sum_i{\nu{Q_i}} \leqslant \sum_i{\sup_{Q_i}{J_\Phi} \cdot \lambda{Q_i}} \leqslant
                \sum_i{\sup_A{J_\Phi} \cdot \lambda{Q_i}} = \sup_A{J_\Phi} \cdot \sum_i{\lambda{Q_i}} =
                \sup_A{J_\Phi} \cdot \lambda{A}
            .\]
        \item Пусть наконец $A$ -- измеримое множество. Пользуясь предыдущей леммой имеем
            \[
                \lambda{A} \cdot \sup_A{J_\Phi} = \inf_{G \supset A}{\left(\lambda{G} \cdot \sup_G{J_\Phi}\right)}
                \geqslant \inf_{G \supset A}{\nu{G}} \geqslant \inf_{G \supset A}{\nu{A}} = \nu{A}
            .\]
        \item Таким образом, мы доказали второе неравенство из критерия плотности. Докажем левую часть,
            перейдя к $\hat{A} = \Phi(A)$, $\hat{\Phi} = \Phi^{-1}$:
            \[
                \lambda{\hat{\Phi}(\hat{A})} = \nu{\hat{A}} \leqslant 
                \sup_{\hat{A}}{J_{\hat{\Phi}}} \cdot \lambda{\hat{A}} \Lra
                \lambda{A} \leqslant \frac{1}{\inf_{A}{J_\Phi}} \cdot \lambda{\Phi(A)} \Lra
                \nu{A} = \lambda{\Phi(A)} \geqslant \inf_{A}{J_\Phi} \cdot \lambda{A}
            .\]
    \end{itemize}
\end{proof}

\begin{theorem}
    Пусть $\cO$ открыто, $\Phi \colon \cO \subseteq \Rm \to \Rm$ -- диффеоморфизм,
    $\cO^1 = \Phi(\cO)$, $f$ -- измеримая неотрицательная функция, тогда
    \[
        \intl_{\cO^1}{f(\ey) \dd \ey} = \intl_{\cO}{f(\Phi(\ex))~ |\det{\Phi'(\ex)}| \dd \ex}
    .\] 
\end{theorem}
\begin{proof}
    По предыдущей теореме, $J_\Phi = |\det{\Phi'(\ex)}|$ -- плотность меры $\nu = \lambda \circ \Phi$.
    Тогда теорема напрямую следует из теоремы об интегрировании по взвешенному образу меры.
\end{proof}

\begin{remark}
    То же верно и в случае, когда $f$ суммируема.
\end{remark}

\begin{example}
    \enewline
    \begin{itemize}
        \item Полярные координаты в $\R^2$.
            \[
                \begin{cases}
                    x = r \cos{\f} \\ 
                    y = r \sin{\f}
                \end{cases} \Lra \det{J} = r
            .\]
        \item Циллиндрические координаты в $\R^3$.
            \[
                \begin{cases}
                    x = r \cos{\f} \\
                    y = r \sin{\f} \\
                    z = z
                \end{cases} \Lra \det{J} = r
            .\]
        \item Сферические координаты в $\R^3$.
            \[
                \begin{cases}
                    x = r \cos{\f} \cos{\psi} \\
                    y = r \sin{\f} \cos{\psi} \\
                    z = r \sin{\psi}
                \end{cases} \Lra \det{J} = r^2 \cos{\f}
            .\]
        \item Сферические координаты в $\R^m$.
            \[
                \begin{cases}
                    x_1 = r \cos{\f_1} \\
                    x_2 = r \sin{\f_1} \cos{\f_2} \\
                    x_3 = r \sin{\f_1} \sin{\f_2} \cos{\f_3} \\
                    \ldots \\
                    x_{n - 1} = r \sin{\f_1} \sin{\f_2} \ldots \sin{\f_{n - 2}} \cos{\f_{n - 1}} \\
                    x_n = r \sin{\f_1} \sin{\f_2} \ldots \sin{\f_{n - 2}} \sin{\f_{n - 1}} 
                \end{cases}
            \]
            Вычислим $\det{J}$ для этого типа замены координат. Для начала сделаем замену
            по последним двум координатам:
            \[
                \begin{cases}
                    x_i = x_i,~ i = 1..{n-2} \\
                    x_{n - 1} = \r_{n - 1} \cos{\f_{n - 1}} \\
                    x_n = \r_{n - 1} \sin{\f_{n - 1}}
                \end{cases}
            .\]
            Теперь заменим $\r_{n - 1}$ и $x_{n - 2}$:
            \[
                \begin{cases}
                    x_i = x_i,~ i = 1..{n-3} \\
                    \r_{n - 1} = \r_{n - 2} \sin{\f_{n - 2}} \\
                    x_{n - 2} = \r_{n - 2} \cos{\f_{n - 2}}
                \end{cases}
            .\]
            Продолжим этот процесс по индукции, в конце получим (вместо $\r_1$ пишем $r$):
            \[
                \begin{cases}
                    \ldots \\
                    \r_2 = r \sin{\f_1} \\
                    x_1 = r \cos{\f_1}
                \end{cases}
            .\]
            Видно, что получилавь серия замен, которая совпадает со сферической заменой.
            Вычислим интеграл единицы по какому-нибудь простому множеству $\cO$, находящемуся в
            положительном октанте ($\forall i~ x_i > 0$):
            \begin{align*}
                \intl{\dd x_1 \d x_2 \ldots \d x_n} 
                &= \intl{\r_{n - 1} \dd x_1 \ldots \d x_{n - 2} \d\r_{n - 1} \d\f_{n - 1}} \\
                &= \intl{\r^2_{n - 2} \sin{\f_{n - 2}} \dd x_1 \ldots \d x_{n - 3} \d\r_{n - 2} 
                    \d\f_{n - 2} \d\f_{n - 1}} \\
                &= \intl{(\r_{n - 3} \sin{\f_{n - 3}})^2 \sin{\f_{n - 2}} \r_{n - 3} 
                    \dd x_1 \ldots \d x_{n - 4} \d\r_{n - 3} \d\f_{n - 3} \d\f_{n - 2} \d\f_{n - 1}} \\
                &= \ldots \\
                &= \intl{r^{n - 1} \sin^{n - 2}{\f_1} \sin^{n - 1}{\f_2} \ldots \sin^1{\f_{n - 2}}
                    \dd\f_1 \d\f_2 \ldots \d\f_{n - 1} \d r} 
            .\end{align*}
            Выражение, получившееся под знаком интеграла и есть якобиан преобразования (по теореме о
            единственности плотности).
    \end{itemize}
\end{example}

\begin{definition}
    При $s, t > 0$ функция, задаваемая формулой
    \[
        B(s, t) = \intl_0^1{x^{s - 1} (1 - x)^{t - 1} \dd x}
    .\]
    называется \textit{бета-функцией}.
\end{definition}

\begin{theorem}
    $\displaystyle \forall s, t > 0~ B(s, t) = \frac{\Gamma(s) \Gamma(t)}{\Gamma(s + t)}$.
\end{theorem}
\begin{proof}
    \begin{align*}
        \Gamma(s) \Gamma(t) 
        &= \parens{\intl_{0}^{+\infty}{x^{s - 1} e^{-x} \dd x}} 
        \parens{\intl_{0}^{+\infty}{y^{t - 1} e^{-y} \dd y}}
        = \intl_0^{+\infty}{x^{s - 1} e^{-x} \parens{\intl_0^{+\infty}{y^{t - 1} e^{-y} \dd y}} \dd x} \\
        &\underset{y = u - x}{=}
        \intl_0^{+\infty}{x^{s - 1} e^{-x} \parens{\intl_x^{+\infty}{
        (u - x)^{t - 1} e^{x - u} \dd y}} \dd u} \\
        &= \intl_0^{+\infty}{\dd u}{\intl_0^{u}{x^{s - 1} (u - x)^{t - 1} e^{-u} \dd x}} \\
        &\underset{x = uv}{=} \intl_0^{+\infty}{\intl_0^1{u^{s - 1}v^{s - 1} 
        u^{t - 1} (1 - v)^{t - 1} e^{-u} u \dd v}} \\
        &= \parens{\intl_0^{+\infty}{u^{s + t - 1} e^{-u} \dd u}} 
        \parens{\intl_0^1{v^{s - 1} (1 - v)^{t - 1} \dd v}} = \Gamma(s + t) B(s, t)
    .\end{align*}
\end{proof}

\begin{theorem}(Объём шара в $\Rm$)
    $\displaystyle \lambda_m{B(0, R)} = \frac{\pi^{\frac{m}{2}}}{\Gamma\parens{\frac{m}{2} + 1}}$.  
\end{theorem}
\begin{proof}
    Вычислим для начала вспомогательный интеграл:
    \begin{align*}
        \int_0^{\pi}{\sin^{n - k - 1}{\f_k} \dd\f_k} 
        &= 2 \int_0^{\frac{\pi}{2}}{\sin^{n - k - 1}{\f_k} \dd\f_k}
        \underset{\sin^2{\f_k} = t}{=} 2 \cdot \frac{1}{2} \intl_0^1
        {\frac{\d t}{\sqrt{t} \sqrt{1 - t}} t^{\frac{n - k - 1}{2}}} \\
        &= B\parens{\frac{n - k}{2}, \frac{1}{2}} = \frac{\Gamma\parens{\frac{n - k}{2}} \Gamma\parens{\frac{1}{2}}}
        {\Gamma\parens{\frac{n - k}{2} + \frac{1}{2}}}
    .\end{align*}
    \begin{align*}
        \lambda_m{B(0, R)} &= \intl_{x_1^2 + \ldots + x_m^2 \leqslant R^2}{1 \dd \lambda_m}
                           = \intl_0^R{\dd r \intl_0^{\pi}{\dd \f_1 \ldots \intl_0^{2 \pi}
                           {r^{n - 1} \sin^{n - 2}{\f_1} \ldots \sin{\f_{n - 2}} \dd \f_{n - 1}}}}
                           = \frac{\pi^{\frac{m}{2}}}{\Gamma\parens{\frac{m}{2} + 1}}R^m
    .\end{align*}
\end{proof}

