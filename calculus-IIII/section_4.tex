\section{Замена переменных в интеграле}

\begin{definition}
    
    Отображение $\Phi \colon X \to Y$ называется \textit{измеримым}, если
\[
    \forall B \in \cB~ \Phi^{-1}(B) \in \cA
.\] 
    Иначе говоря, прообраз измеримого множества измерим.
\end{definition}

\begin{lemma}
    $\Phi^{-1}(\cB)$ -- $\sigma$-алгебра.
\end{lemma}

\begin{definition}
    При фиксированном измеримом $\Phi \colon X \to Y$ отображение
    \begin{align*}
        \nu \colon &\cB \to \Rbar \\ 
                   &B \mapsto \mu(\Phi^{-1}(B))
    \end{align*}
    назовем \textit{образом меры $\mu$ при отображении} $\Phi$.
\end{definition}

\begin{lemma}
    Образ меры при отображении является мерой.
\end{lemma}

\begin{remark}
    $\displaystyle \nu(B) = \intl_{\Phi^{-1}(B)}{1 \dd\mu}$
\end{remark}

\begin{remark}
    Если функция $f \colon Y \to \Rbar$ измерима относительно $\cB$, 
    то $f \circ \Phi \colon X \to \Rbar$ измерима относительно $\cA$.
\end{remark}

\begin{definition}
    Пусть $\omega \colon X \to \Rbar$, $\omega \geqslant 0$, измерима.
    В этом контексте $\omega$ называется \textit{весовой функцией}. Тогда
    \textit{взвешенным образом меры $\mu$ с весом $\omega$} называется мера
    \begin{align*}
        \nu(B) = \intl_{\Phi^{-1}(B)}{\omega \dd\mu}    
    \end{align*}
\end{definition}

\begin{theorem}(Об интегрировании по взвешенному образу меры)
    
    Пусть $\Phi \colon X \to Y$ -- измеримое отображение, $0 \leqslant 
    \omega \colon X \to \Rbar$ -- весовая функция, измерима на $X$, 
    $\nu$ -- взвешенный образ меры $\mu$ с весом $\omega$.
    Тогда для любой измеримой $f \colon Y \to \Rbar$ верно:
    \begin{itemize}
        \item $f \circ \Phi$ измерима на $X$.
        \item $\displaystyle \intl_{Y}{f \dd\nu} = \intl_{X}{(f \circ \Phi)~ \omega \dd\mu}$
    \end{itemize} 
\end{theorem}

\begin{corollary}
    Пусть $f$ суммируема на $Y$, $B \in \cB$, тогда в условиях теоремы:
\[
    \intl_{B}{f \dd\nu} = \intl_{\Phi^{-1}(B)}{(f \circ \Phi)~ \omega \dd\mu}
.\] 
\end{corollary}

\begin{definition}
    В ситуации $X = Y$, $\cA = \cB$, $\Phi = \id$, если $\omega \geqslant 0$ измерима,
    причем $\displaystyle \nu(B) = \intl_{B}{\omega \dd\mu}$, $\omega$ называется
    \textit{плотностью меры $\nu$ относительно меры $\mu$}.
    В таком случае
\[
    \intl_{X}{f \dd\nu} = \intl_{X}{f \omega \dd\mu}
.\] 
\end{definition}

\begin{theorem}(Критерий плотности)

    Пусть $\nu$ -- мера на $\cA$, $\omega \geqslant 0$ измерима, тогда
    верно, что $\omega$ -- плотность $\nu$ относительно $\mu$ тогда
    и только тогда, когда
\[
    \forall A \in \cA~ \inf_{A}{\omega} \cdot \mu(A) \leqslant \nu(A)
    \leqslant \sup_{A}{\omega} \cdot \mu(A)
.\] 
\end{theorem}

\begin{lemma}
    Пусть $f$, $g$ -- суммируемые на $X$ функции, причем
\[
    \forall A \in \cA~ \intl_{A}{f \dd\mu} = \intl_{A}{f \dd\mu}
.\] 
    Тогда $f = g$ почти везде.
\end{lemma}

\begin{lemma}(Об образе малых кубических ячеек)
    
    Пусть $\cO$ открыто, $\Phi \colon \cO \subseteq \Rm \to \Rm$, $\ea \in \cO$, $\Phi$
    дифференцируемо в $\ea$, $\det{\Phi'(\ea)} \neq 0$, $c > |\det{\Phi'(\ea)}| > 0$.
    Тогда
\[
    \exists \delta > 0~ \forall Q \text{ -- куб}, Q \subset B(\ea, \delta)~
    \lambda \Phi(Q) < c \cdot \lambda(Q)
.\] 
\end{lemma}

\begin{lemma}
    Пусть $\cO$ открыто, $f \colon \cO \subseteq \Rm \to \R$, $f \in C(\cO)$, 
    $A \in \mathfrak{M}^m$, $A \subseteq Q$, $Q$ -- куб, причем $\Cl(Q) \subseteq \cO$. 
    Тогда
\[
    \inf_{\substack{A \subset G \\ G \text{ открыто}}}
    {\left(\lambda(G) \cdot \sup_{G}{f}\right)}
    = \lambda(A) \cdot \sup_{A}{f}
.\] 
\end{lemma}

\begin{theorem}
    Пусть $\cO$ открыто, $\Phi \colon \cO \subseteq \Rm \to \Rm$ -- диффеоморфизм,
    $A \in \mathfrak{M}^m$, $A \subseteq \cO$, тогда
\[
    \lambda \Phi(A) = \intl_{A}{|\det{\Phi'}| \dd\lambda_m}
.\] 
\end{theorem}

\begin{theorem}
    Пусть $\cO$ открыто, $\Phi \colon \cO \subseteq \Rm \to \Rm$ -- диффеоморфизм,
    $\cO^1 = \Phi(\cO)$, $f$ -- измеримая неотрицательная функция, тогда
\[
    \intl_{\cO^1}{f(y) \dd y} = \intl_{\cO}{f(\Phi(x))~ |\det{\Phi'(x)}| \dd x}
.\] 
\end{theorem}

\begin{remark}
    То же верно и в случае, когда $f$ суммируема.
\end{remark}
